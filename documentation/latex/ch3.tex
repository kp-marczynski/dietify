\chapter{Projekt}\label{ch:project}

\section{Przypadki użycia}\label{sec:usecase}
\todo{diagram przypadków użycia}
\todo{scenariusze przypadków użycia}

\begin{minipage}{\textwidth}
    \begin{figure}[H]
        \centering\includegraphics[scale=0.55]{../uml/use_case_diagrams/users.png}
        \caption{Użytkownicy - diagram przypadków użycia (opr.wł).}\label{rysunek:use-case-diagram-users}
    \end{figure}
\end{minipage}

\begin{minipage}{\textwidth}
    \begin{figure}[H]
        \centering\includegraphics[scale=0.55]{../uml/use_case_diagrams/gateway.png}
        \caption{Gateway - diagram przypadków użycia (opr.wł).}\label{rysunek:use-case-diagram-gateway}
    \end{figure}
\end{minipage}

\begin{minipage}{\textwidth}
    \begin{figure}[H]
        \centering\includegraphics[scale=0.55]{../uml/use_case_diagrams/products.png}
        \caption{Produkty - diagram przypadków użycia (opr.wł).}\label{rysunek:use-case-diagram-products}
    \end{figure}
\end{minipage}

\begin{minipage}{\textwidth}
    \begin{figure}[H]
        \centering\includegraphics[scale=0.55]{../uml/use_case_diagrams/recipes.png}
        \caption{Przepisy - diagram przypadków użycia (opr.wł).}\label{rysunek:use-case-diagram-recipes}
    \end{figure}
\end{minipage}

\begin{minipage}{\textwidth}
    \begin{figure}[H]
        \centering\includegraphics[scale=0.55]{../uml/use_case_diagrams/mealplans.png}
        \caption{Jadłospisy - diagram przypadków użycia (opr.wł).}\label{rysunek:use-case-diagram-mealplans}
    \end{figure}
\end{minipage}

\begin{minipage}{\textwidth}
    \begin{figure}[H]
        \centering\includegraphics[scale=0.55]{../uml/use_case_diagrams/appointments.png}
        \caption{Wizyty - diagram przypadków użycia (opr.wł).}\label{rysunek:use-case-diagram-appointments}
    \end{figure}
\end{minipage}

\section{Prototyp interfejsu}
\todo{mockupy}
\begin{minipage}{\textwidth}
    \begin{figure}[H]
        \centering\includegraphics[width=0.9\textwidth]{img/mockups/mockup1.png}
        \caption{Mockup1 (opr.wł).}\label{rysunek:mockup1}
    \end{figure}
\end{minipage}

\begin{minipage}{\textwidth}
    \begin{figure}[H]
        \centering\includegraphics[width=0.9\textwidth]{img/mockups/mockup2.png}
        \caption{Mockup2 (opr.wł).}\label{rysunek:mockup2}
    \end{figure}
\end{minipage}

\begin{minipage}{\textwidth}
    \begin{figure}[H]
        \centering\includegraphics[width=0.9\textwidth]{img/mockups/mockup3.png}
        \caption{Mockup3 (opr.wł).}\label{rysunek:mockup3}
    \end{figure}
\end{minipage}

\begin{minipage}{\textwidth}
    \begin{figure}[H]
        \centering\includegraphics[width=0.9\textwidth]{img/mockups/mockup4.png}
        \caption{Mockup4 (opr.wł).}\label{rysunek:mockup4}
    \end{figure}
\end{minipage}

\begin{minipage}{\textwidth}
    \begin{figure}[H]
        \centering\includegraphics[width=0.9\textwidth]{img/mockups/mockup5.png}
        \caption{Mockup5 (opr.wł).}\label{rysunek:mockup5}
    \end{figure}
\end{minipage}

\begin{minipage}{\textwidth}
    \begin{figure}[H]
        \centering\includegraphics[width=0.9\textwidth]{img/mockups/mockup6.png}
        \caption{Mockup6 (opr.wł).}\label{rysunek:mockup6}
    \end{figure}
\end{minipage}

\begin{minipage}{\textwidth}
    \begin{figure}[H]
        \centering\includegraphics[width=0.9\textwidth]{img/mockups/mockup7.png}
        \caption{Mockup7 (opr.wł).}\label{rysunek:mockup7}
    \end{figure}
\end{minipage}

\begin{minipage}{\textwidth}
    \begin{figure}[H]
        \centering\includegraphics[width=0.9\textwidth]{img/mockups/mockup8.png}
        \caption{Mockup8 (opr.wł).}\label{rysunek:mockup8}
    \end{figure}
\end{minipage}

\begin{minipage}{\textwidth}
    \begin{figure}[H]
        \centering\includegraphics[width=0.9\textwidth]{img/mockups/mockup9.png}
        \caption{Mockup9 (opr.wł).}\label{rysunek:mockup9}
    \end{figure}
\end{minipage}

\begin{minipage}{\textwidth}
    \begin{figure}[H]
        \centering\includegraphics[width=0.9\textwidth]{img/mockups/mockup10.png}
        \caption{Mockup10 (opr.wł).}\label{rysunek:mockup10}
    \end{figure}
\end{minipage}

\begin{minipage}{\textwidth}
    \begin{figure}[H]
        \centering\includegraphics[width=0.9\textwidth]{img/mockups/mockup11.png}
        \caption{Mockup11 (opr.wł).}\label{rysunek:mockup11}
    \end{figure}
\end{minipage}

\begin{minipage}{\textwidth}
    \begin{figure}[H]
        \centering\includegraphics[width=0.9\textwidth]{img/mockups/mockup12.png}
        \caption{Mockup12 (opr.wł).}\label{rysunek:mockup12}
    \end{figure}
\end{minipage}

\begin{minipage}{\textwidth}
    \begin{figure}[H]
        \centering\includegraphics[width=0.9\textwidth]{img/mockups/mockup13.png}
        \caption{Mockup13 (opr.wł).}\label{rysunek:mockup13}
    \end{figure}
\end{minipage}

\begin{minipage}{\textwidth}
    \begin{figure}[H]
        \centering\includegraphics[width=0.9\textwidth]{img/mockups/mockup14.png}
        \caption{Mockup14 (opr.wł).}\label{rysunek:mockup14}
    \end{figure}
\end{minipage}

\begin{minipage}{\textwidth}
    \begin{figure}[H]
        \centering\includegraphics[width=0.9\textwidth]{img/mockups/mockup15.png}
        \caption{Mockup15 (opr.wł).}\label{rysunek:mockup15}
    \end{figure}
\end{minipage}

\begin{minipage}{\textwidth}
    \begin{figure}[H]
        \centering\includegraphics[width=0.9\textwidth]{img/mockups/mockup16.png}
        \caption{Mockup16 (opr.wł).}\label{rysunek:mockup16}
    \end{figure}
\end{minipage}

\section{Kategorie}\label{sec:categories}
\todo{uzupełnić kategorie}

\begin{enumerate}[label={\textbf{KAT/\protect\threedigits{\theenumi}}}, wide, labelwidth=!, labelindent=0pt, labelsep=0pt, series=reqs]
    \setlength\itemsep{1em}
    \req{User} \label{kat:User} (Użytkownik)

    Opis: \lipsum[1]
    \par
    Atrybuty:
    \begin{itemize}[series=atr]
        \atr{id} \label{kat:User:id}
        \atr{login} \label{kat:User:login}
        \atr{passwordHash} \label{kat:User:passwordHash}
        \atr{firstName} \label{kat:User:firstName}
        \atr{lastName} \label{kat:User:lastName}
        \atr{email} \label{kat:User:email}
        \atr{image} \label{kat:User:image}
        \atr{activated} \label{kat:User:activated}
        \atr{langKey} \label{kat:User:langKey}
        \atr{activationKey} \label{kat:User:activationKey}
        \atr{resetKey} \label{kat:User:resetKey}
        \atr{createdDate} \label{kat:User:createdDate}
        \atr{resetDate} \label{kat:User:resetDate}
        \atr{lastModifiedDate} \label{kat:User:lastModifiedDate}
    \end{itemize}

    \req{Authority} \label{kat:Authority} (Rola)

    Opis: \lipsum[1]
    \par
    Atrybuty:
    \begin{itemize}[series=atr]
        \atr{name} \label{kat:Authority:name}
    \end{itemize}

    \req{UserExtraInfo} \label{kat:UserExtraInfo} (Dodatkowe Informacje Użytkownika)

    Opis: \lipsum[1]
    \par
    Atrybuty:
    \begin{itemize}[series=atr]
        \atr{id} \label{kat:UserExtraInfo:id}
        \atr{gender} \label{kat:UserExtraInfo:gender}
        \atr{dateOfBirth} \label{kat:UserExtraInfo:dateOfBirth}
        \atr{phoneNumber} \label{kat:UserExtraInfo:phoneNumber}
        \atr{streetAddress} \label{kat:UserExtraInfo:streetAddress}
        \atr{postalCode} \label{kat:UserExtraInfo:postalCode}
        \atr{city} \label{kat:UserExtraInfo:city}
        \atr{country} \label{kat:UserExtraInfo:country}
        \atr{personalDescription} \label{kat:UserExtraInfo:personalDescription}
    \end{itemize}

    \req{SiteContent} \label{kat:SiteContent} (Treść Strony)

    Opis: \lipsum[1]
    \par
    Atrybuty:
    \begin{itemize}[series=atr]
        \atr{id} \label{kat:SiteContent:id}
        \atr{ordinalNumber} \label{kat:SiteContent:ordinalNumber}
        \atr{siteContentType} \label{kat:SiteContent:siteContentType}
        \atr{title} \label{kat:SiteContent:title}
        \atr{description} \label{kat:SiteContent:description}
    \end{itemize}

    \req{SiteContentTranslation} \label{kat:SiteContentTranslation} (Tłumaczenie Treści Strony)

    Opis: \lipsum[1]
    \par
    Atrybuty:
    \begin{itemize}[series=atr]
        \atr{id} \label{kat:SiteContentTranslation:id}
        \atr{title} \label{kat:SiteContentTranslation:title}
        \atr{description} \label{kat:SiteContentTranslation:description}
        \atr{language} \label{kat:SiteContentTranslation:language}
    \end{itemize}

    \req{ContactInfo} \label{kat:ContactInfo} (Informacje Kontaktowe)

    Opis: \lipsum[1]
    \par
    Atrybuty:
    \begin{itemize}[series=atr]
        \atr{id} \label{kat:ContactInfo:id}
        \atr{contactInfoType} \label{kat:ContactInfo:contactInfoType}
        \atr{description} \label{kat:ContactInfo:description}
    \end{itemize}

    \req{Pricing} \label{kat:Pricing} (Cennik)

    Opis: \lipsum[1]
    \par
    Atrybuty:
    \begin{itemize}[series=atr]
        \atr{id} \label{kat:Pricing:id}
        \atr{ordinalNumber} \label{kat:Pricing:ordinalNumber}
        \atr{title} \label{kat:Pricing:title}
        \atr{description} \label{kat:Pricing:description}
        \atr{price} \label{kat:Pricing:price}
        \atr{currency} \label{kat:Pricing:currency}
    \end{itemize}

    \req{PricingTranslation} \label{kat:PricingTranslation} (Tłumaczenie Cennika)

    Opis: \lipsum[1]
    \par
    Atrybuty:
    \begin{itemize}[series=atr]
        \atr{id} \label{kat:PricingTranslation:id}
        \atr{title} \label{kat:PricingTranslation:title}
        \atr{description} \label{kat:PricingTranslation:description}
        \atr{language} \label{kat:PricingTranslation:language}
    \end{itemize}

    \req{Product} \label{kat:Product} (Produkt)

    Opis: \lipsum[1]
    \par
    Atrybuty:
    \begin{itemize}[series=atr]
        \atr{id} \label{kat:Product:id}
        \atr{source} \label{kat:Product:source}
        \atr{isPublic} \label{kat:Product:isPublic}
        \atr{language} \label{kat:Product:language}
    \end{itemize}

    \req{ProductVersion} \label{kat:ProductVersion} (Wersja Produktu)

    Opis: \lipsum[1]
    \par
    Atrybuty:
    \begin{itemize}[series=atr]
        \atr{id} \label{kat:ProductVersion:id}
        \atr{editTimestamp} \label{kat:ProductVersion:editTimestamp}
        \atr{description} \label{kat:ProductVersion:description}
    \end{itemize}

    \req{ProductBasicNutritionData} \label{kat:ProductBasicNutritionData} (Podstawowe Składniki Odżywcze Produktu)

    Opis: \lipsum[1]
    \par
    Atrybuty:
    \begin{itemize}[series=atr]
        \atr{id} \label{kat:ProductBasicNutritionData:id}
        \atr{energy} \label{kat:ProductBasicNutritionData:energy}
        \atr{protein} \label{kat:ProductBasicNutritionData:protein}
        \atr{fat} \label{kat:ProductBasicNutritionData:fat}
        \atr{carbohydrates} \label{kat:ProductBasicNutritionData:carbohydrates}
    \end{itemize}

    \req{NutritionData} \label{kat:NutritionData} (Wartość Odżywcza)

    Opis: \lipsum[1]
    \par
    Atrybuty:
    \begin{itemize}[series=atr]
        \atr{id} \label{kat:NutritionData:id}
        \atr{nutritionValue} \label{kat:NutritionData:nutritionValue}
    \end{itemize}

    \req{NutritionDefinition} \label{kat:NutritionDefinition} (Definicja Wartości Odżywczej)

    Opis: \lipsum[1]
    \par
    Atrybuty:
    \begin{itemize}[series=atr]
        \atr{id} \label{kat:NutritionDefinition:id}
        \atr{tag} \label{kat:NutritionDefinition:tag}
        \atr{description} \label{kat:NutritionDefinition:description}
        \atr{units} \label{kat:NutritionDefinition:units}
        \atr{decimalPlaces} \label{kat:NutritionDefinition:decimalPlaces}
    \end{itemize}

    \req{NutritionDefinitionTranslation} \label{kat:NutritionDefinitionTranslation} (Tłumaczenie Definicji Wartości Odżywczej)

    Opis: \lipsum[1]
    \par
    Atrybuty:
    \begin{itemize}[series=atr]
        \atr{id} \label{kat:NutritionDefinitionTranslation:id}
        \atr{translation} \label{kat:NutritionDefinitionTranslation:translation}
        \atr{language} \label{kat:NutritionDefinitionTranslation:language}
    \end{itemize}

    \req{HouseholdMeasure} \label{kat:HouseholdMeasure} (Miara Domowa)

    Opis: \lipsum[1]
    \par
    Atrybuty:
    \begin{itemize}[series=atr]
        \atr{id} \label{kat:HouseholdMeasure:id}
        \atr{description} \label{kat:HouseholdMeasure:description}
        \atr{gramsWeight} \label{kat:HouseholdMeasure:gramsWeight}
        \atr{isVisible} \label{kat:HouseholdMeasure:isVisible}
    \end{itemize}

    \req{ProductSubcategory} \label{kat:ProductSubcategory} (Podkategoria Produktu)

    Opis: \lipsum[1]
    \par
    Atrybuty:
    \begin{itemize}[series=atr]
        \atr{id} \label{kat:ProductSubcategory:id}
        \atr{description} \label{kat:ProductSubcategory:description}
    \end{itemize}

    \req{ProductCategory} \label{kat:ProductCategory} (Kategoria Produktu)

    Opis: \lipsum[1]
    \par
    Atrybuty:
    \begin{itemize}[series=atr]
        \atr{id} \label{kat:ProductCategory:id}
        \atr{description} \label{kat:ProductCategory:description}
    \end{itemize}

    \req{ProductCategoryTranslation} \label{kat:ProductCategoryTranslation} (Tłumaczenie Kategorii Produktu)

    Opis: \lipsum[1]
    \par
    Atrybuty:
    \begin{itemize}[series=atr]
        \atr{id} \label{kat:ProductCategoryTranslation:id}
        \atr{translation} \label{kat:ProductCategoryTranslation:translation}
        \atr{language} \label{kat:ProductCategoryTranslation:language}
    \end{itemize}

    \req{DietType} \label{kat:DietType} (Typ Diety)

    Opis: \lipsum[1]
    \par
    Atrybuty:
    \begin{itemize}[series=atr]
        \atr{id} \label{kat:DietType:id}
        \atr{name} \label{kat:DietType:name}
    \end{itemize}

    \req{DietTypeTranslation} \label{kat:DietTypeTranslation} (Tłumaczenie Typu Diety)

    Opis: \lipsum[1]
    \par
    Atrybuty:
    \begin{itemize}[series=atr]
        \atr{id} \label{kat:DietTypeTranslation:id}
        \atr{translation} \label{kat:DietTypeTranslation:translation}
        \atr{language} \label{kat:DietTypeTranslation:language}
    \end{itemize}

    \req{Recipe} \label{kat:Recipe} (Przepis)

    Opis: \lipsum[1]
    \par
    Atrybuty:
    \begin{itemize}[series=atr]
        \atr{id} \label{kat:Recipe:id}
        \atr{isPublic} \label{kat:Recipe:isPublic}
        \atr{language} \label{kat:Recipe:language}
    \end{itemize}

    \req{RecipeVersion} \label{kat:RecipeVersion} (Wersja Przepisu)

    Opis: \lipsum[1]
    \par
    Atrybuty:
    \begin{itemize}[series=atr]
        \atr{id} \label{kat:RecipeVersion:id}
        \atr{editTimestamp} \label{kat:RecipeVersion:editTimestamp}
        \atr{name} \label{kat:RecipeVersion:name}
        \atr{preparationTimeMinutes} \label{kat:RecipeVersion:preparationTimeMinutes}
        \atr{numberOfPortions} \label{kat:RecipeVersion:numberOfPortions}
        \atr{image} \label{kat:RecipeVersion:image}
        \atr{totalGramsWeight} \label{kat:RecipeVersion:totalGramsWeight}
    \end{itemize}

    \req{RecipeBasicNutritionData} \label{kat:RecipeBasicNutritionData} (Podstawowe Wartości Odżywcze Przepisu)

    Opis: \lipsum[1]
    \par
    Atrybuty:
    \begin{itemize}[series=atr]
        \atr{id} \label{kat:RecipeBasicNutritionData:id}
        \atr{energy} \label{kat:RecipeBasicNutritionData:energy}
        \atr{protein} \label{kat:RecipeBasicNutritionData:protein}
        \atr{fat} \label{kat:RecipeBasicNutritionData:fat}
        \atr{carbohydrates} \label{kat:RecipeBasicNutritionData:carbohydrates}
    \end{itemize}

    \req{RecipeSection} \label{kat:RecipeSection} (Sekcja Przepisu)

    Opis: \lipsum[1]
    \par
    Atrybuty:
    \begin{itemize}[series=atr]
        \atr{id} \label{kat:RecipeSection:id}
        \atr{sectionName} \label{kat:RecipeSection:sectionName}
    \end{itemize}

    \req{ProductPortion} \label{kat:ProductPortion} (Porcja Produktu)

    Opis: \lipsum[1]
    \par
    Atrybuty:
    \begin{itemize}[series=atr]
        \atr{id} \label{kat:ProductPortion:id}
        \atr{amount} \label{kat:ProductPortion:amount}
    \end{itemize}

    \req{PreparationStep} \label{kat:PreparationStep} (Krok Przygotowania)

    Opis: \lipsum[1]
    \par
    Atrybuty:
    \begin{itemize}[series=atr]
        \atr{id} \label{kat:PreparationStep:id}
        \atr{ordinalNumber} \label{kat:PreparationStep:ordinalNumber}
        \atr{stepDescription} \label{kat:PreparationStep:stepDescription}
    \end{itemize}

    \req{KitchenAppliance} \label{kat:KitchenAppliance} (Sprzęt Kuchenny)

    Opis: \lipsum[1]
    \par
    Atrybuty:
    \begin{itemize}[series=atr]
        \atr{id} \label{kat:KitchenAppliance:id}
        \atr{name} \label{kat:KitchenAppliance:name}
    \end{itemize}

    \req{KitchenApplianceTranslation} \label{kat:KitchenApplianceTranslation} (Tłumaczenie Sprzętu Kuchennego)

    Opis: \lipsum[1]
    \par
    Atrybuty:
    \begin{itemize}[series=atr]
        \atr{id} \label{kat:KitchenApplianceTranslation:id}
        \atr{translation} \label{kat:KitchenApplianceTranslation:translation}
        \atr{language} \label{kat:KitchenApplianceTranslation:language}
    \end{itemize}

    \req{DishType} \label{kat:DishType} (Typ Dania)

    Opis: \lipsum[1]
    \par
    Atrybuty:
    \begin{itemize}[series=atr]
        \atr{id} \label{kat:DishType:id}
        \atr{description} \label{kat:DishType:description}
    \end{itemize}

    \req{DishTypeTranslation} \label{kat:DishTypeTranslation} (Tłumaczenie Typu Dania)

    Opis: \lipsum[1]
    \par
    Atrybuty:
    \begin{itemize}[series=atr]
        \atr{id} \label{kat:DishTypeTranslation:id}
        \atr{translation} \label{kat:DishTypeTranslation:translation}
        \atr{language} \label{kat:DishTypeTranslation:language}
    \end{itemize}

    \req{MealType} \label{kat:MealType} (Typ Posiłku)

    Opis: \lipsum[1]
    \par
    Atrybuty:
    \begin{itemize}[series=atr]
        \atr{id} \label{kat:MealType:id}
        \atr{name} \label{kat:MealType:name}
    \end{itemize}

    \req{MealTypeTranslation} \label{kat:MealTypeTranslation} (Tłumaczenie Typu Posiłku)

    Opis: \lipsum[1]
    \par
    Atrybuty:
    \begin{itemize}[series=atr]
        \atr{id} \label{kat:MealTypeTranslation:id}
        \atr{translation} \label{kat:MealTypeTranslation:translation}
        \atr{language} \label{kat:MealTypeTranslation:language}
    \end{itemize}


    \req{MealPlan} \label{kat:MealPlan} (Jadłospis)

    Opis: \lipsum[1]
    \par
    Atrybuty:
    \begin{itemize}[series=atr]
        \atr{id} \label{kat:MealPlan:id}
        \atr{creationTimestamp} \label{kat:MealPlan:creationTimestamp}
        \atr{editTimestamp} \label{kat:MealPlan:editTimestamp}
        \atr{name} \label{kat:MealPlan:name}
        \atr{isVisible} \label{kat:MealPlan:isVisible}
        \atr{language} \label{kat:MealPlan:language}
        \atr{numberOfDays} \label{kat:MealPlan:numberOfDays}
        \atr{numberOfMealsPerDay} \label{kat:MealPlan:numberOfMealsPerDay}
        \atr{totalDailyEnergy} \label{kat:MealPlan:totalDailyEnergy}
        \atr{percentOfProtein} \label{kat:MealPlan:percentOfProtein}
        \atr{percentOfFat} \label{kat:MealPlan:percentOfFat}
        \atr{percentOfCarbohydrates} \label{kat:MealPlan:percentOfCarbohydrates}
    \end{itemize}

    \req{MealPlanDay} \label{kat:MealPlanDay} (Dzień Jadłospisu)

    Opis: \lipsum[1]
    \par
    Atrybuty:
    \begin{itemize}[series=atr]
        \atr{id} \label{kat:MealPlanDay:id}
        \atr{ordinalNumber} \label{kat:MealPlanDay:ordinalNumber}
    \end{itemize}

    \req{Meal} \label{kat:Meal} (Posiłek)

    Opis: \lipsum[1]
    \par
    Atrybuty:
    \begin{itemize}[series=atr]
        \atr{id} \label{kat:Meal:id}
        \atr{ordinalNumber} \label{kat:Meal:ordinalNumber}
    \end{itemize}

    \req{MealRecipe} \label{kat:MealRecipe} (Przepis Posiłku)

    Opis: \lipsum[1]
    \par
    Atrybuty:
    \begin{itemize}[series=atr]
        \atr{id} \label{kat:MealRecipe:id}
        \atr{amount} \label{kat:MealRecipe:amount}
    \end{itemize}

    \req{MealProduct} \label{kat:MealProduct} (Produkt Posiłku)

    Opis: \lipsum[1]
    \par
    Atrybuty:
    \begin{itemize}[series=atr]
        \atr{id} \label{kat:MealProduct:id}
        \atr{amount} \label{kat:MealProduct:amount}
    \end{itemize}

    \req{MealDefinition} \label{kat:MealDefinition} (Definicja Posiłku)

    Opis: \lipsum[1]
    \par
    Atrybuty:
    \begin{itemize}[series=atr]
        \atr{id} \label{kat:MealDefinition:id}
        \atr{ordinalNumber} \label{kat:MealDefinition:ordinalNumber}
        \atr{timeOfMeal} \label{kat:MealDefinition:timeOfMeal}
        \atr{percentOfEnergy} \label{kat:MealDefinition:percentOfEnergy}
    \end{itemize}

    \req{Appointment} \label{kat:Appointment} (Wizyta)

    Opis: \lipsum[1]
    \par
    Atrybuty:
    \begin{itemize}[series=atr]
        \atr{id} \label{kat:Appointment:id}
        \atr{appointmentDate} \label{kat:Appointment:appointmentDate}
        \atr{appointmentState} \label{kat:Appointment:appointmentState}
        \atr{generalAdvice} \label{kat:Appointment:generalAdvice}
    \end{itemize}

    \req{PatientCard} \label{kat:PatientCard} (Karta Pacjenta)

    Opis: \lipsum[1]
    \par
    Atrybuty:
    \begin{itemize}[series=atr]
        \atr{id} \label{kat:PatientCard:id}
        \atr{creationDate} \label{kat:PatientCard:creationDate}
    \end{itemize}

    \req{AppointmentEvaluation} \label{kat:AppointmentEvaluation} (Ewaluacja Wizyty)

    Opis: \lipsum[1]
    \par
    Atrybuty:
    \begin{itemize}[series=atr]
        \atr{id} \label{kat:AppointmentEvaluation:id}
        \atr{overallSatisfaction} \label{kat:AppointmentEvaluation:overallSatisfaction}
        \atr{dietitianServiceSatisfaction} \label{kat:AppointmentEvaluation:dietitianServiceSatisfaction}
        \atr{mealPlanOverallSatisfaction} \label{kat:AppointmentEvaluation:mealPlanOverallSatisfaction}
        \atr{mealCostSatisfaction} \label{kat:AppointmentEvaluation:mealCostSatisfaction}
        \atr{mealPreparationTimeSatisfaction} \label{kat:AppointmentEvaluation:mealPreparationTimeSatisfaction}
        \atr{mealComplexityLevelSatisfaction} \label{kat:AppointmentEvaluation:mealComplexityLevelSatisfaction}
        \atr{mealTastefulnessSatisfaction} \label{kat:AppointmentEvaluation:mealTastefulnessSatisfaction}
        \atr{dietaryResultSatisfaction} \label{kat:AppointmentEvaluation:dietaryResultSatisfaction}
        \atr{comment} \label{kat:AppointmentEvaluation:comment}
    \end{itemize}

    \req{BodyMeasurement} \label{kat:BodyMeasurement} (Pomiar Ciała)

    Opis: \lipsum[1]
    \par
    Atrybuty:
    \begin{itemize}[series=atr]
        \atr{id} \label{kat:BodyMeasurement:id}
        \atr{completionDate} \label{kat:BodyMeasurement:completionDate}
        \atr{height} \label{kat:BodyMeasurement:height}
        \atr{weight} \label{kat:BodyMeasurement:weight}
        \atr{waist} \label{kat:BodyMeasurement:waist}
        \atr{percentOfFatTissue} \label{kat:BodyMeasurement:percentOfFatTissue}
        \atr{percentOfWater} \label{kat:BodyMeasurement:percentOfWater}
        \atr{muscleMass} \label{kat:BodyMeasurement:muscleMass}
        \atr{physicalMark} \label{kat:BodyMeasurement:physicalMark}
        \atr{calciumInBones} \label{kat:BodyMeasurement:calciumInBones}
        \atr{basicMetabolism} \label{kat:BodyMeasurement:basicMetabolism}
        \atr{metabolicAge} \label{kat:BodyMeasurement:metabolicAge}
        \atr{visceralDatLevel} \label{kat:BodyMeasurement:visceralDatLevel}
    \end{itemize}

    \req{NutritionalInterview} \label{kat:NutritionalInterview} (Wywiad Żywieniowy)

    Opis: \lipsum[1]
    \par
    Atrybuty:
    \begin{itemize}[series=atr]
        \atr{id} \label{kat:NutritionalInterview:id}
        \atr{completionDate} \label{kat:NutritionalInterview:completionDate}
        \atr{targetWeight} \label{kat:NutritionalInterview:targetWeight}
        \atr{advicePurpose} \label{kat:NutritionalInterview:advicePurpose}
        \atr{physicalActivity} \label{kat:NutritionalInterview:physicalActivity}
        \atr{diseases} \label{kat:NutritionalInterview:diseases}
        \atr{medicines} \label{kat:NutritionalInterview:medicines}
        \atr{jobType} \label{kat:NutritionalInterview:jobType}
        \atr{likedProducts} \label{kat:NutritionalInterview:likedProducts}
        \atr{dislikedProducts} \label{kat:NutritionalInterview:dislikedProducts}
        \atr{foodAllergies} \label{kat:NutritionalInterview:foodAllergies}
        \atr{foodIntolerances} \label{kat:NutritionalInterview:foodIntolerances}
    \end{itemize}

    \req{CustomNutritionalInterviewQuestion} \label{kat:CustomNutritionalInterviewQuestion} (Niestandardowe Pytanie Wywiadu Żywieniowego)

    Opis: \lipsum[1]
    \par
    Atrybuty:
    \begin{itemize}[series=atr]
        \atr{id} \label{kat:CustomNutritionalInterviewQuestion:id}
        \atr{ordinalNumber} \label{kat:CustomNutritionalInterviewQuestion:ordinalNumber}
        \atr{question} \label{kat:CustomNutritionalInterviewQuestion:question}
        \atr{answer} \label{kat:CustomNutritionalInterviewQuestion:answer}
    \end{itemize}

    \req{CustomNutritionalInterviewQuestionTemplate} \label{kat:CustomNutritionalInterviewQuestionTemplate} (Szablon Niestandardowego Pytania Wywiadu Żywieniowego)

    Opis: \lipsum[1]
    \par
    Atrybuty:
    \begin{itemize}[series=atr]
        \atr{id} \label{kat:CustomNutritionalInterviewQuestionTemplate:id}
        \atr{question} \label{kat:CustomNutritionalInterviewQuestionTemplate:question}
        \atr{language} \label{kat:CustomNutritionalInterviewQuestionTemplate:language}
    \end{itemize}

    \req{AssignedMealPlan} \label{kat:AssignedMealPlan} (Przypisany Jadłospis)

    Opis: \lipsum[1]
    \par
    Atrybuty:
    \begin{itemize}[series=atr]
        \atr{id} \label{kat:AssignedMealPlan:id}
        \atr{assigmentTime} \label{kat:AssignedMealPlan:assigmentTime}
    \end{itemize}

\end{enumerate}

\section {Reguły funkcjonowania}\label{sec:functionalRules}
\todo{uzupełnić reguły funkcjonowania, i.e. związki pomiędzy encjami}

\begin{itemize}[label={\textbf{Reguły dla}}, wide, labelwidth=!, labelindent=0pt]
    \setlength\itemsep{1em}
    \item[\textbf{Reguły}] \textbf{ogólne}
    \begin{enumerate}[label={\textbf{REG/\protect\threedigits{\arabic{enumi}}}}, wide, labelwidth=!, align=left, leftmargin=3cm]
        \item Przedmiot kompozycji podlega takim samym zasadom dostępu co właściciel kompozycji pod warunkiem, że przedmiot kompozycji nie definuje własnych reguł dopstepu
    \end{enumerate}
    \item\ref{kat:User}
    \begin{enumerate}[label={\textbf{REG/\protect\threedigits{\arabic{enumi}}}}, wide, labelwidth=!, align=left, leftmargin=3cm, resume]
        %Relacje
        \item Użytkownik (\ref{kat:User}) nie musi mieć musi mieć żadnych dodatkowych informacji (\ref{kat:UserExtraInfo})
        \item Użytkownik (\ref{kat:User}) może mieć maksymalnie jedne dodatkowe informacje (\ref{kat:UserExtraInfo})
        \item Użytkownik (\ref{kat:User}) musi mieć musi mieć przynajmniej jedną rolę (\ref{kat:Authority})
        \item Użytkownik (\ref{kat:User}) może mieć wiele ról (\ref{kat:Authority})
        \item Użytkownik (\ref{kat:User}) nie musi mieć autora (\ref{kat:User})
        \item Użytkownik (\ref{kat:User}) może mieć maksymalnie jednego autora (\ref{kat:User})
        \item Użytkownik (\ref{kat:User}) nie musi mieć ostatniego edytora (\ref{kat:User})
        \item Użytkownik (\ref{kat:User}) może mieć maksymalnie jednego ostatniego edytora (\ref{kat:User})
        %CRUD
        \item \role{Gość} może dodawać nowego użytkownika (\ref{kat:User})
        \item \role{Użytkownik} może wyświetlać, edytować i usuwać swoje dane użytkownika (\ref{kat:User})
        \item \role{Dietetyk} może wyświetlać podstawowe dane (\ref{kat:User}) \role{Pacjenta}, którego kartotekę prowadzi
        \item \role{Administrator} może wyświetlać i usuwać dane użytkownika (\ref{kat:User})
    \end{enumerate}
    \item\ref{kat:Authority}
    \begin{enumerate}[label={\textbf{REG/\protect\threedigits{\arabic{enumi}}}}, wide, labelwidth=!, align=left, leftmargin=3cm, resume]
        %Relacje
        %CRUD
        \item \role{Administrator} może dodawać, wyświetlać, edytować i usuwać dane roli (\ref{kat:Authority})
    \end{enumerate}
    \item\ref{kat:UserExtraInfo}
    \begin{enumerate}[label={\textbf{REG/\protect\threedigits{\arabic{enumi}}}}, wide, labelwidth=!, align=left, leftmargin=3cm, resume]
        %Relacje
        \item Dodatkowe informacje (\ref{kat:UserExtraInfo}) muszą być przypisane do dokładnie jednego użytkownika (\ref{kat:User})
        %CRUD
        \item Dodatkowe informacje (\ref{kat:UserExtraInfo}) są przedmiotem kompozycji ze strony użytkownika (\ref{kat:User})
    \end{enumerate}
    \item\ref{kat:SiteContent}
    \begin{enumerate}[label={\textbf{REG/\protect\threedigits{\arabic{enumi}}}}, wide, labelwidth=!, align=left, leftmargin=3cm, resume]
        %Relacje
        \item Treść strony (\ref{kat:SiteContent}) nie musi mieć żadnego tłumaczenia (\ref{kat:SiteContentTranslation})
        \item Treść strony (\ref{kat:SiteContent}) może mieć wiele tłumaczeń (\ref{kat:SiteContentTranslation})
        %CRUD
        \item \role{Gość} może wyświetlać dane treści strony (\ref{kat:SiteContent})
        \item \role{Użytkownik} może wyświetlać dane treści strony (\ref{kat:SiteContent})
        \item \role{Administrator} może dodawać, edytować i usuwać dane treści strony (\ref{kat:SiteContent})
    \end{enumerate}
    \item\ref{kat:SiteContentTranslation}
    \begin{enumerate}[label={\textbf{REG/\protect\threedigits{\arabic{enumi}}}}, wide, labelwidth=!, align=left, leftmargin=3cm, resume]
        %Relacje
        \item Tłumaczenie treści strony (\ref{kat:SiteContentTranslation}) musi być przypisane do dokładnie jednej treści strony (\ref{kat:SiteContent})
        %CRUD
        \item Tłumaczenie treści strony (\ref{kat:SiteContentTranslation}) jest przedmiotem kompozycji ze strony treści strony (\ref{kat:SiteContent})
    \end{enumerate}
    \item\ref{kat:ContactInfo}
    \begin{enumerate}[label={\textbf{REG/\protect\threedigits{\arabic{enumi}}}}, wide, labelwidth=!, align=left, leftmargin=3cm, resume]
        %CRUD
        \item \role{Gość} może wyświetlać dane informacji kontaktowych (\ref{kat:ContactInfo})
        \item \role{Użytkownik} może wyświetlać dane informacji kontaktowych (\ref{kat:ContactInfo})
        \item \role{Administrator} może dodawać, edytować i usuwać dane informacji kontaktowych (\ref{kat:ContactInfo})
    \end{enumerate}
    \item\ref{kat:Pricing}
    \begin{enumerate}[label={\textbf{REG/\protect\threedigits{\arabic{enumi}}}}, wide, labelwidth=!, align=left, leftmargin=3cm, resume]
        %Relacje
        \item Cennik (\ref{kat:Pricing}) nie musi mieć przypisanych żadnych tłumaczeń (\ref{kat:PricingTranslation})
        \item Cennik (\ref{kat:Pricing}) może mieć przypisane wiele tłumaczeń (\ref{kat:PricingTranslation})
        %CRUD
        \item \role{Gość} może wyświetlać dane cennika (\ref{kat:Pricing})
        \item \role{Użytkownik} może wyświetlać dane cennika (\ref{kat:Pricing})
        \item \role{Administrator} może dodawać, edytować i usuwać dane cennika (\ref{kat:Pricing})
    \end{enumerate}
    \item\ref{kat:PricingTranslation}
    \begin{enumerate}[label={\textbf{REG/\protect\threedigits{\arabic{enumi}}}}, wide, labelwidth=!, align=left, leftmargin=3cm, resume]
        %Relacje
        \item Tłumaczenie cennika (\ref{kat:PricingTranslation}) musi być przypisane do dokładnie jednego cennika (\ref{kat:Pricing})
        %CRUD
        \item Tłumaczenie cennika (\ref{kat:PricingTranslation}) jest przedmiotem kompozycji ze strony cennika (\ref{kat:Pricing})
    \end{enumerate}
    \item\ref{kat:Product}
    \begin{enumerate}[label={\textbf{REG/\protect\threedigits{\arabic{enumi}}}}, wide, labelwidth=!, align=left, leftmargin=3cm, resume]
        %Relacje
        \item Produkt (\ref{kat:Product}) musi mieć przynajmniej jedną wersję (\ref{kat:ProductVersion})
        \item Produkt (\ref{kat:Product}) może mieć wiele wersji (\ref{kat:ProductVersion})
        \item Produkt (\ref{kat:Product}) nie musi mieć zdefiniowanego autora (\ref{kat:User})
        \item Produkt (\ref{kat:Product}) może mieć maksymalnie jednego autora (\ref{kat:User})
        %CRUD
        \item todo
    \end{enumerate}
    \item\ref{kat:ProductVersion}
    \begin{enumerate}[label={\textbf{REG/\protect\threedigits{\arabic{enumi}}}}, wide, labelwidth=!, align=left, leftmargin=3cm, resume]
        %Relacje
        \item Wersja produktu (\ref{kat:ProductVersion}) musi być przypisana do dokładnie jednego produktu  (\ref{kat:Product})
        \item Wersja produktu (\ref{kat:ProductVersion}) musi być przypisana do dokładnie jednych podstawowych wartości odżywczych (\ref{kat:ProductBasicNutritionData})
        \item Wersja produktu (\ref{kat:ProductVersion}) nie musi mieć zdefiniowanych żadnych wartości odżywczych (\ref{kat:NutritionData})
        \item Wersja produktu (\ref{kat:ProductVersion}) może mieć zdefiniowane wiele wartości odżywczych (\ref{kat:NutritionData})
        \item Wersja produktu (\ref{kat:ProductVersion}) nie musi mieć zdefiniowanych żadnych miar domowych (\ref{kat:HouseholdMeasure})
        \item Wersja produktu (\ref{kat:ProductVersion}) może mieć zdefiniowane wiele miar domowych (\ref{kat:HouseholdMeasure})
        \item Wersja produktu (\ref{kat:ProductVersion}) musi należeć do dokładnie jednej podkategorii (\ref{kat:ProductSubcategory})
        \item Wersja produktu (\ref{kat:ProductVersion}) nie musi mieć przypisanego żadnego odpowiedniego typu diety (\ref{kat:DietType})
        \item Wersja produktu (\ref{kat:ProductVersion}) może mieć przypisanych wiele odpowiednich typów diety (\ref{kat:DietType})
        \item Wersja produktu (\ref{kat:ProductVersion}) nie musi mieć przypisanego żadnego nieodpowiedniego typu diety (\ref{kat:DietType})
        \item Wersja produktu (\ref{kat:ProductVersion}) może mieć przypisanych wiele nieodpowiednich typów diety (\ref{kat:DietType})
        %CRUD
        \item Wersja produktu (\ref{kat:ProductVersion}) jest przedmiotem kompozycji ze strony produktu (\ref{kat:Product})
    \end{enumerate}
    \item\ref{kat:ProductBasicNutritionData}
    \begin{enumerate}[label={\textbf{REG/\protect\threedigits{\arabic{enumi}}}}, wide, labelwidth=!, align=left, leftmargin=3cm, resume]
        %Relacje
        \item Podstawowe wartości odżywcze produktu(\ref{kat:ProductBasicNutritionData}) muszą być przypisane do dokladnie jednej wersji produktu (\ref{kat:ProductVersion})
        %CRUD
        \item Podstawowe wartości odżywcze produktu (\ref{kat:ProductBasicNutritionData}) są przedmiotem kompozycji ze strony wersji produktu (\ref{kat:ProductVersion})
    \end{enumerate}
    \item\ref{kat:NutritionData}
    \begin{enumerate}[label={\textbf{REG/\protect\threedigits{\arabic{enumi}}}}, wide, labelwidth=!, align=left, leftmargin=3cm, resume]
        %Relacje
        \item Wartość odżywcza (\ref{kat:NutritionData}) musi być przypisana do dokładnie jednej wersji produktu (\ref{kat:ProductVersion})
        \item Wartość odżywcza (\ref{kat:NutritionData}) musi być przypisana do dokładnie jednej definicji wartości odżywczej (\ref{kat:NutritionDefinition})
        %CRUD
        \item Wartość odżywcza (\ref{kat:NutritionData}) jest przedmiotem kompozycji ze strony wersji produktu (\ref{kat:ProductVersion})
    \end{enumerate}
    \item\ref{kat:NutritionDefinition}
    \begin{enumerate}[label={\textbf{REG/\protect\threedigits{\arabic{enumi}}}}, wide, labelwidth=!, align=left, leftmargin=3cm, resume]
        %Relacje
        \item Definicja wartości odżywczej (\ref{kat:NutritionDefinition}) nie musi mieć zdefiniowanego żadnego tłumaczenia (\ref{kat:NutritionDefinitionTranslation})
        \item Definicja wartości odżywczej (\ref{kat:NutritionDefinition}) może mieć zdefiniowanych wiele tłumaczeń (\ref{kat:NutritionDefinitionTranslation})
        %CRUD
        \item todo
    \end{enumerate}
    \item\ref{kat:NutritionDefinitionTranslation}
    \begin{enumerate}[label={\textbf{REG/\protect\threedigits{\arabic{enumi}}}}, wide, labelwidth=!, align=left, leftmargin=3cm, resume]
        %Relacje
        \item Tłumaczenie definicji wartości odżywczej (\ref{kat:NutritionDefinitionTranslation}) musi być przypisane do dokładnie jednej definicji wartości odżywczej  (\ref{kat:NutritionDefinition})
        %CRUD
        \item Tłumaczenie definicji wartości odżywczej (\ref{kat:NutritionDefinitionTranslation}) jest przedmiotem kompozycji ze strony definicji wartości odżywczej (\ref{kat:NutritionDefinition})
    \end{enumerate}
    \item\ref{kat:HouseholdMeasure}
    \begin{enumerate}[label={\textbf{REG/\protect\threedigits{\arabic{enumi}}}}, wide, labelwidth=!, align=left, leftmargin=3cm, resume]
        %Relacje
        \item Miara domowa (\ref{kat:HouseholdMeasure}) musi być przypisana do dokładnie jednej wersji produktu (\ref{kat:ProductVersion})
        %CRUD
        \item Miara domowa (\ref{kat:HouseholdMeasure}) jest przedmiotem kompozycji ze strony wersji produktu (\ref{kat:ProductVersion})
    \end{enumerate}
    \item\ref{kat:ProductSubcategory}
    \begin{enumerate}[label={\textbf{REG/\protect\threedigits{\arabic{enumi}}}}, wide, labelwidth=!, align=left, leftmargin=3cm, resume]
        %Relacje
        \item Podkategoria produktu (\ref{kat:ProductSubcategory}) musi być przypisana do conajmniej jednej wersji produktu (\ref{kat:ProductVersion})
        \item Podkategoria produktu (\ref{kat:ProductSubcategory}) może być przypisana do wielu wersji produktu (\ref{kat:ProductVersion})
        \item Podktagoria produktu (\ref{kat:ProductSubcategory}) musi być przypisana do dokładnie jednej kategorii (\ref{kat:ProductCategory})
        %CRUD
        \item todo
    \end{enumerate}
    \item\ref{kat:ProductCategory}
    \begin{enumerate}[label={\textbf{REG/\protect\threedigits{\arabic{enumi}}}}, wide, labelwidth=!, align=left, leftmargin=3cm, resume]
        %Relacje
        \item Kategoria produktu (\ref{kat:ProductCategory}) nie musi mieć przypisanego żadnego tłumaczenia (\ref{kat:ProductCategoryTranslation})
        \item Kategoria produktu (\ref{kat:ProductCategory}) może mieć przypisanych wiele tłumaczeń (\ref{kat:ProductCategoryTranslation})
        %CRUD
        \item todo
    \end{enumerate}
    \item\ref{kat:ProductCategoryTranslation}
    \begin{enumerate}[label={\textbf{REG/\protect\threedigits{\arabic{enumi}}}}, wide, labelwidth=!, align=left, leftmargin=3cm, resume]
        %Relacje
        \item Tłumaczenie kategorii produktu (\ref{kat:ProductCategoryTranslation}) musi być przypisane do dokładnie jednej kategorii (\ref{kat:ProductCategory})
        %CRUD
        \item Tłumaczenie kategorii produktu (\ref{kat:ProductCategoryTranslation}) jest przedmiotem kompozycji ze strony kategorii (\ref{kat:ProductCategory})
    \end{enumerate}
    \item\ref{kat:DietType}
    \begin{enumerate}[label={\textbf{REG/\protect\threedigits{\arabic{enumi}}}}, wide, labelwidth=!, align=left, leftmargin=3cm, resume]
        %Relacje
        \item Typ diety (\ref{kat:DietType}) nie musi mieć zdefiniowanego żadnego tłumaczenia (\ref{kat:DietTypeTranslation})
        \item Typ diety (\ref{kat:DietType}) może mieć zdefiniowanych wiele tłumaczeń (\ref{kat:DietTypeTranslation})
        %CRUD
        \item todo
    \end{enumerate}
    \item\ref{kat:DietTypeTranslation}
    \begin{enumerate}[label={\textbf{REG/\protect\threedigits{\arabic{enumi}}}}, wide, labelwidth=!, align=left, leftmargin=3cm, resume]
        %Relacje
        \item Tłumaczenie typu diety (\ref{kat:DietTypeTranslation}) musi być przypisane do dokładnie jednego typu diety (\ref{kat:DietType})
        %CRUD
        \item Tłumaczenie typu diety (\ref{kat:DietTypeTranslation}) jest przedmiotem kompozycji ze strony typu diety (\ref{kat:DietType})
    \end{enumerate}
    \item\ref{kat:Recipe}
    \begin{enumerate}[label={\textbf{REG/\protect\threedigits{\arabic{enumi}}}}, wide, labelwidth=!, align=left, leftmargin=3cm, resume]
        %Relacje
        \item Przepis (\ref{kat:Recipe}) nie musi mieć zdefiniowanego żadnego przepisu źródłowego (\ref{kat:Recipe})
        \item Przepis (\ref{kat:Recipe}) może mieć zdefiniowany maksymalnie jeden przepis źródłowy (\ref{kat:Recipe})
        \item Przepis (\ref{kat:Recipe}) musi mieć przynajmniej jedną wersję (\ref{kat:RecipeVersion})
        \item Przepis (\ref{kat:Recipe}) może mieć wiele wersji (\ref{kat:RecipeVersion})
        \item Przepis (\ref{kat:Recipe}) nie musi mieć zdefiniowanego autora (\ref{kat:User})
        \item Przepis (\ref{kat:Recipe}) może mieć maksymalnie jednego autora (\ref{kat:User})
        %CRUD
        \item todo
    \end{enumerate}
    \item\ref{kat:RecipeVersion}
    \begin{enumerate}[label={\textbf{REG/\protect\threedigits{\arabic{enumi}}}}, wide, labelwidth=!, align=left, leftmargin=3cm, resume]
        %Relacje
        \item Wersja przepisu (\ref{kat:RecipeVersion}) musi mieć dokładnie jedne podstawowe wartości odżywcze przepisu (\ref{kat:RecipeBasicNutritionData})
        \item Wersja przepisu (\ref{kat:RecipeVersion}) musi mieć przynajmniej jedną sekcję (\ref{kat:RecipeSection})
        \item Wersja przepisu (\ref{kat:RecipeVersion}) może mieć wiele sekcji (\ref{kat:RecipeSection})
        \item Wersja przepisu (\ref{kat:RecipeVersion}) nie musi mieć przypisanego żadnego sprzętu kuchennego (\ref{kat:KitchenAppliance})
        \item Wersja przepisu (\ref{kat:RecipeVersion}) może mieć przypisanych wiele sprzętów kuchennych (\ref{kat:KitchenAppliance})
        \item Wersja przepisu (\ref{kat:RecipeVersion}) nie musi mieć przypisanego żadnego typu dania (\ref{kat:DishType})
        \item Wersja przepisu (\ref{kat:RecipeVersion}) może mieć przypisanych wiele typów dań (\ref{kat:DishType})
        \item Wersja przepisu (\ref{kat:RecipeVersion}) nie musi mieć przypisanego żadnego typu posiłku (\ref{kat:MealType})
        \item Wersja przepisu (\ref{kat:RecipeVersion}) może mieć przypisanych wiele typów posiłków (\ref{kat:MealType})
        \item Wersja przepisu (\ref{kat:RecipeVersion}) nie musi mieć przypisanego żadnego odpowiedniego typu diety (\ref{kat:DietType})
        \item Wersja przepisu (\ref{kat:RecipeVersion}) może mieć przypisanych wiele odpowiednich typów diety  (\ref{kat:DietType})
        \item Wersja przepisu (\ref{kat:RecipeVersion}) nie musi mieć przypisanego żadnego nieodpowiedniego typu diety (\ref{kat:DietType})
        \item Wersja przepisu (\ref{kat:RecipeVersion}) może mieć przypisanych wiele nieodpowiednich typów diety (\ref{kat:DietType})
        %CRUD
        \item Wersja przepisu (\ref{kat:RecipeVersion}) jest przedmiotem kompozycji ze strony przepisu (\ref{kat:Recipe})
    \end{enumerate}
    \item\ref{kat:RecipeBasicNutritionData}
    \begin{enumerate}[label={\textbf{REG/\protect\threedigits{\arabic{enumi}}}}, wide, labelwidth=!, align=left, leftmargin=3cm, resume]
        %Relacje
        \item Podstawowe wartości odżywcze przepisu (\ref{kat:RecipeBasicNutritionData}) muszą być przypisane do dokładnie jednej wersji przepisu (\ref{kat:RecipeVersion})
        %CRUD
        \item Podstawowe wartości odżywcze przepisu (\ref{kat:RecipeBasicNutritionData}) są przedmiotem kompozycji ze strony wersji przepisu (\ref{kat:RecipeVersion})
    \end{enumerate}
    \item\ref{kat:RecipeSection}
    \begin{enumerate}[label={\textbf{REG/\protect\threedigits{\arabic{enumi}}}}, wide, labelwidth=!, align=left, leftmargin=3cm, resume]
        %Relacje
        \item Sekcja przepisu (\ref{kat:RecipeSection}) musi być przypisana do dokładniej jednej wersji przepisu (\ref{kat:RecipeVersion})
        \item Sekcja przepisu (\ref{kat:RecipeSection}) musi mieć przypisaną przynajmniej jedną porcję produktu (\ref{kat:ProductPortion})
        \item Sekcja przepisu (\ref{kat:RecipeSection}) może mieć przypisanych wiele porcji produktu (\ref{kat:ProductPortion})
        \item Sekcja przepisu (\ref{kat:RecipeSection}) musi mieć przypisany przynajmniej jeden krok przygotowania (\ref{kat:PreparationStep})
        \item Sekcja przepisu (\ref{kat:RecipeSection}) może mieć zdefiniowanych wiele kroków przygotowania (\ref{kat:PreparationStep})
        %CRUD
        \item Sekcja przepisu (\ref{kat:RecipeSection}) jest przedmiotem kompozycji ze strony wersji przepisu (\ref{kat:RecipeVersion})
    \end{enumerate}
    \item\ref{kat:ProductPortion}
    \begin{enumerate}[label={\textbf{REG/\protect\threedigits{\arabic{enumi}}}}, wide, labelwidth=!, align=left, leftmargin=3cm, resume]
        %Relacje
        \item Porcja produktu (\ref{kat:ProductPortion}) musi być przypisana do dokładnie jednej sekcji przepisu (\ref{kat:RecipeSection})
        \item Porcja produktu (\ref{kat:ProductPortion}) musi mieć przypisany dokładnie jeden produkt (\ref{kat:Product})
        \item Porcja produktu (\ref{kat:ProductPortion}) nie musi mieć przypisanej miary domowej (\ref{kat:HouseholdMeasure})
        \item Porcja produktu (\ref{kat:ProductPortion}) może mieć przypisaną maksymalnie jedną miarę domową (\ref{kat:HouseholdMeasure})
        %CRUD
        \item Porcja produktu (\ref{kat:ProductPortion}) jest przedmiotem kompozycji ze strony sekcji przepisu (\ref{kat:RecipeSection})
    \end{enumerate}
    \item\ref{kat:PreparationStep}
    \begin{enumerate}[label={\textbf{REG/\protect\threedigits{\arabic{enumi}}}}, wide, labelwidth=!, align=left, leftmargin=3cm, resume]
        %Relacje
        \item Krok przygotowania (\ref{kat:PreparationStep}) musi być przypisany do dokładnie jednej sekcji przepisu (\ref{kat:RecipeSection})
        %CRUD
        \item Krok przygotowania (\ref{kat:PreparationStep}) jest przedmiotem kompozycji ze strony sekcji przepisu (\ref{kat:RecipeSection})
    \end{enumerate}
    \item\ref{kat:KitchenAppliance}
    \begin{enumerate}[label={\textbf{REG/\protect\threedigits{\arabic{enumi}}}}, wide, labelwidth=!, align=left, leftmargin=3cm, resume]
        %Relacje
        \item Sprzęt kuchenny (\ref{kat:KitchenAppliance}) nie musi mieć zdefiniowanego żadnego tłumaczenia (\ref{kat:KitchenApplianceTranslation})
        \item Sprzęt kuchenny (\ref{kat:KitchenAppliance}) może mieć zdefiniowanych wiele tłumaczeń (\ref{kat:KitchenApplianceTranslation})
        %CRUD
        \item todo
    \end{enumerate}
    \item\ref{kat:KitchenApplianceTranslation}
    \begin{enumerate}[label={\textbf{REG/\protect\threedigits{\arabic{enumi}}}}, wide, labelwidth=!, align=left, leftmargin=3cm, resume]
        %Relacje
        \item Tłumaczenie sprzetu kuchennego (\ref{kat:KitchenApplianceTranslation}) musi być przypisane do dokaldnie jednego sprzetu kuchennego (\ref{kat:KitchenAppliance})
        %CRUD
        \item Tłumaczenie sprzętu kuchennego (\ref{kat:KitchenApplianceTranslation}) jest przedmiotem kompozycji ze strony sprzętu kuchennego (\ref{kat:KitchenAppliance})
    \end{enumerate}
    \item\ref{kat:DishType}
    \begin{enumerate}[label={\textbf{REG/\protect\threedigits{\arabic{enumi}}}}, wide, labelwidth=!, align=left, leftmargin=3cm, resume]
        %Relacje
        \item Typ dania (\ref{kat:DishType}) nie musi mieć zdefiniowanego żadnego tłumaczenia (\ref{kat:DishTypeTranslation})
        \item Typ dania (\ref{kat:DishType}) może mieć zdefiniowanych wiele tłumaczeń (\ref{kat:DishTypeTranslation})
        %CRUD
        \item todo
    \end{enumerate}
    \item\ref{kat:DishTypeTranslation}
    \begin{enumerate}[label={\textbf{REG/\protect\threedigits{\arabic{enumi}}}}, wide, labelwidth=!, align=left, leftmargin=3cm, resume]
        %Relacje
        \item Tłumaczenie typu dania (\ref{kat:DishTypeTranslation}) musi być przypisane do dokładnie jednego typu dania (\ref{kat:DishType})
        %CRUD
        \item Tłumaczenie typu dania (\ref{kat:DishTypeTranslation}) jest przedmiotem kompozycji ze strony typu dania (\ref{kat:DishType})
    \end{enumerate}
    \item\ref{kat:MealType}
    \begin{enumerate}[label={\textbf{REG/\protect\threedigits{\arabic{enumi}}}}, wide, labelwidth=!, align=left, leftmargin=3cm, resume]
        %Relacje
        \item Typ posiłku (\ref{kat:MealType}) nie musi mieć zdefiniowanego żadnego tłumaczenia (\ref{kat:MealTypeTranslation})
        \item Typ posiłku (\ref{kat:MealType}) może mieć zdefiniowanych wiele tłumaczeń (\ref{kat:MealTypeTranslation})
        %CRUD
        \item todo
    \end{enumerate}
    \item\ref{kat:MealTypeTranslation}
    \begin{enumerate}[label={\textbf{REG/\protect\threedigits{\arabic{enumi}}}}, wide, labelwidth=!, align=left, leftmargin=3cm, resume]
        %Relacje
        \item Tłumaczenie typu posiłku (\ref{kat:MealTypeTranslation}) musi być przypisane do dokładnie jednego typu posiłku (\ref{kat:MealType})
        %CRUD
        \item Tłumaczenie typu posiłku (\ref{kat:MealTypeTranslation}) jest przedmiotem kompozycji ze strony typu posiłku (\ref{kat:MealType})
    \end{enumerate}
    \item\ref{kat:MealPlan}
    \begin{enumerate}[label={\textbf{REG/\protect\threedigits{\arabic{enumi}}}}, wide, labelwidth=!, align=left, leftmargin=3cm, resume]
        %Relacje
        \item Jadłospis (\ref{kat:MealPlan}) musi mieć przypisany przynajmniej jeden dzień (\ref{kat:MealPlanDay})
        \item Jadłospis (\ref{kat:MealPlan}) może mieć przypisanych maksymalnie 31 dni (\ref{kat:MealPlanDay})
        \item Jadłospis (\ref{kat:MealPlan}) musi mieć przypisaną przynajmniej jedną definicję posiłku (\ref{kat:MealDefinition})
        \item Jadłospis (\ref{kat:MealPlan}) może mieć przypisanych maksymalnie 10 definicji posiłków (\ref{kat:MealDefinition})
        \item Jadłospis (\ref{kat:MealPlan}) nie musi mieć przypisanego żadnego odpowiedniego typu diety (\ref{kat:DietType})
        \item Jadłospis (\ref{kat:MealPlan}) może mieć przypisanych wiele odpowiednich typów diety (\ref{kat:DietType})
        \item Jadłospis (\ref{kat:MealPlan}) nie musi mieć przypisanego żadnego nieodpowiedniego typu diety (\ref{kat:DietType})
        \item Jadłospis (\ref{kat:MealPlan}) może mieć przypisanych wiele nieodpowiednich typów diety (\ref{kat:DietType})
        \item Jadłospis (\ref{kat:MealPlan}) musi mieć dokładnie jednego autora (\ref{kat:User})
        %CRUD
        \item todo
    \end{enumerate}
    \item\ref{kat:MealPlanDay}
    \begin{enumerate}[label={\textbf{REG/\protect\threedigits{\arabic{enumi}}}}, wide, labelwidth=!, align=left, leftmargin=3cm, resume]
        %Relacje
        \item Dzień jadłospisu (\ref{kat:MealPlanDay}) musi być przypisany do dokładnie jednego jadłospisu (\ref{kat:MealPlan})
        \item Dzień jadłospisu (\ref{kat:MealPlanDay}) nie musi mieć przypisanego żadnego posiłku (\ref{kat:Meal})
        \item Dzień jadłospisu (\ref{kat:MealPlanDay}) może mieć przypisanych maksymalnie 10 posiłków (\ref{kat:Meal})
        %CRUD
        \item Dzień jadłospisu (\ref{kat:MealPlanDay}) jest przedmiotem kompozycji ze strony jadłospisu (\ref{kat:MealPlan})
    \end{enumerate}
    \item\ref{kat:Meal}
    \begin{enumerate}[label={\textbf{REG/\protect\threedigits{\arabic{enumi}}}}, wide, labelwidth=!, align=left, leftmargin=3cm, resume]
        %Relacje
        \item Posiłek (\ref{kat:Meal}) musi być przypisany do dokładnie jednego dnia jadłospisu (\ref{kat:MealPlanDay})
        \item Posiłek (\ref{kat:Meal}) nie musi mieć przypisanego żadnego produktu (\ref{kat:MealProduct})
        \item Posiłek (\ref{kat:Meal}) może mieć przypisanych wiele produktów (\ref{kat:MealProduct})
        \item Posiłek (\ref{kat:Meal}) nie musi mieć przypisanego żadnego przepisu (\ref{kat:MealRecipe})
        \item Posiłek (\ref{kat:Meal}) może mieć przypisanych wiele przepisów (\ref{kat:MealRecipe})
        %CRUD
        \item Posiłek (\ref{kat:Meal}) jest przedmiotem kompozycji ze strony dnia jadłospisu (\ref{kat:MealPlanDay})
    \end{enumerate}
    \item\ref{kat:MealRecipe}
    \begin{enumerate}[label={\textbf{REG/\protect\threedigits{\arabic{enumi}}}}, wide, labelwidth=!, align=left, leftmargin=3cm, resume]
        %Relacje
        \item Przepis posiłku (\ref{kat:MealRecipe}) musi być przypisany do dokladnie jednego posiłku (\ref{kat:Meal})
        \item Przepis posiłku (\ref{kat:MealRecipe}) musi mieć przypisany dokładnie jeden przepis (\ref{kat:Recipe})
        %CRUD
        \item Przepis posiłku (\ref{kat:MealRecipe}) jest przedmiotem kompozycji ze strony posiłku (\ref{kat:Meal})
    \end{enumerate}
    \item\ref{kat:MealProduct}
    \begin{enumerate}[label={\textbf{REG/\protect\threedigits{\arabic{enumi}}}}, wide, labelwidth=!, align=left, leftmargin=3cm, resume]
        %Relacje
        \item Produkt posiłku (\ref{kat:MealProduct}) musi być przypisany do dokładnie jednego posiłku (\ref{kat:Meal})
        \item Produkt posiłku (\ref{kat:MealProduct}) musi mieć przypisany dokładnie jeden produkt (\ref{kat:Product})
        \item Produkt posiłku (\ref{kat:MealProduct}) nie musi mieć przypisanej żadnej miary domowej (\ref{kat:HouseholdMeasure})
        \item Produkt posiłku (\ref{kat:MealProduct}) musi mieć przypisaną maksymalnie jedną miarę domową (\ref{kat:HouseholdMeasure})
        %CRUD
        \item Produkt posiłku (\ref{kat:MealProduct}) jest przedmiotem kompozycji ze strony posiłku (\ref{kat:Meal})
    \end{enumerate}
    \item\ref{kat:MealDefinition}
    \begin{enumerate}[label={\textbf{REG/\protect\threedigits{\arabic{enumi}}}}, wide, labelwidth=!, align=left, leftmargin=3cm, resume]
        %Relacje
        \item Definicja posiłku (\ref{kat:MealDefinition}) musi być przypisana do dokładnie jednego jadłospisu (\ref{kat:MealPlan})
        \item Definicja posiłku (\ref{kat:MealDefinition}) musi mieć przypisany dokładnie jeden typ posiłku (\ref{kat:MealType})
        %CRUD
        \item Definicja posiłku (\ref{kat:MealDefinition}) jest przedmiotem kompozycji ze strony jadłospisu (\ref{kat:MealPlan})
    \end{enumerate}
    \item\ref{kat:Appointment}
    \begin{enumerate}[label={\textbf{REG/\protect\threedigits{\arabic{enumi}}}}, wide, labelwidth=!, align=left, leftmargin=3cm, resume]
        %Relacje
        \item Wizyta (\ref{kat:Appointment}) musi być przypisana do dokładnie jednej karty pacjenta (\ref{kat:PatientCard})
        \item Wizyta (\ref{kat:Appointment}) nie musi mieć przypisanej żadnej ewaluacji (\ref{kat:AppointmentEvaluation})
        \item Wizyta (\ref{kat:Appointment}) może mieć przypisaną maksymalnie jedną ewaluację (\ref{kat:AppointmentEvaluation})
        \item Wizyta (\ref{kat:Appointment}) nie musi mieć przypisanego żadnych pomiarów ciała (\ref{kat:BodyMeasurement})
        \item Wizyta (\ref{kat:Appointment}) może mieć przypisane maksymalnie jedne pomiary ciała (\ref{kat:BodyMeasurement})
        \item Wizyta (\ref{kat:Appointment}) nie musi mieć przypisanego żadnego wywiadu żywieniowego (\ref{kat:NutritionalInterview})
        \item Wizyta (\ref{kat:Appointment}) może mieć przypisany maksymalnie jeden wywiad żywieniowy (\ref{kat:NutritionalInterview})
        \item Wizyta (\ref{kat:Appointment}) nie musi mieć przypisanego żadnego jadłospisu (\ref{kat:AssignedMealPlan})
        \item Wizyta (\ref{kat:Appointment}) może mieć przypisanych wiele jadłospisów (\ref{kat:AssignedMealPlan})
        %CRUD
        \item todo
    \end{enumerate}
    \item\ref{kat:PatientCard}
    \begin{enumerate}[label={\textbf{REG/\protect\threedigits{\arabic{enumi}}}}, wide, labelwidth=!, align=left, leftmargin=3cm, resume]
        %Relacje
        \item Karta pacjenta (\ref{kat:PatientCard}) nie musi mieć przypisanej żadnej wizyty (\ref{kat:Appointment})
        \item Karta pacjenta (\ref{kat:PatientCard}) może mieć przypisanych wiele wizyt (\ref{kat:Appointment})
        \item Karta pacjenta (\ref{kat:PatientCard}) musi mieć przypisanego dokładnie jednego pacjenta (\ref{kat:User})
        \item Karta pacjenta (\ref{kat:PatientCard}) musi mieć przypisanego dokładnie jednego dietetyka (\ref{kat:User})
        %CRUD
        \item todo
    \end{enumerate}
    \item\ref{kat:AppointmentEvaluation}
    \begin{enumerate}[label={\textbf{REG/\protect\threedigits{\arabic{enumi}}}}, wide, labelwidth=!, align=left, leftmargin=3cm, resume]
        %Relacje
        \item Ewaluacja wizyty (\ref{kat:AppointmentEvaluation}) musi być przypisana do dokładnie jednej wizyty (\ref{kat:Appointment})
        %CRUD
        \item todo
    \end{enumerate}
    \item\ref{kat:BodyMeasurement}
    \begin{enumerate}[label={\textbf{REG/\protect\threedigits{\arabic{enumi}}}}, wide, labelwidth=!, align=left, leftmargin=3cm, resume]
        %Relacje
        \item Pomiary ciała (\ref{kat:BodyMeasurement}) muszą być przypisane do dokładnie jednej wizyty (\ref{kat:Appointment})
        %CRUD
        \item Pomiary ciała (\ref{kat:BodyMeasurement}) są przedmiotem kompozycji ze strony wizyty (\ref{kat:Appointment})
    \end{enumerate}
    \item\ref{kat:NutritionalInterview}
    \begin{enumerate}[label={\textbf{REG/\protect\threedigits{\arabic{enumi}}}}, wide, labelwidth=!, align=left, leftmargin=3cm, resume]
        %Relacje
        \item Wywiad żywieniowy (\ref{kat:NutritionalInterview}) musi być przypisany do dokładnie jednej wizyty (\ref{kat:Appointment})
        \item Wywiad żywieniowy (\ref{kat:NutritionalInterview}) nie musi mieć przypisanego żadnego niestandardowego pytania (\ref{kat:CustomNutritionalInterviewQuestion})
        \item Wywiad żywiniowy (\ref{kat:NutritionalInterview}) może mieć przypisanych wiele niestandardowych pytań (\ref{kat:CustomNutritionalInterviewQuestion})
        \item Wywiad żywieniowy (\ref{kat:NutritionalInterview}) nie musi mieć przypisanych żadnych posiadanych sprzętów kuchennych (\ref{kat:KitchenAppliance})
        \item Wywiad żywieniowy (\ref{kat:NutritionalInterview}) może mieć przypisanych wiele posiadanych sprzętów kuchennych (\ref{kat:KitchenAppliance})
        %CRUD
        \item Wywiad żywieniowy (\ref{kat:NutritionalInterview}) jest przedmiotem kompozycji ze strony wizyty (\ref{kat:Appointment})
    \end{enumerate}
    \item\ref{kat:CustomNutritionalInterviewQuestion}
    \begin{enumerate}[label={\textbf{REG/\protect\threedigits{\arabic{enumi}}}}, wide, labelwidth=!, align=left, leftmargin=3cm, resume]
        %Relacje
        \item Niestandardowe pytanie żywieniowe (\ref{kat:CustomNutritionalInterviewQuestion}) musi być przypisane do dokładnie jednego wywiadu żywieniowego (\ref{kat:NutritionalInterview})
        %CRUD
        \item  Niestandardowe pytanie żywieniowe (\ref{kat:CustomNutritionalInterviewQuestion}) jest przedmiotem kompozycji ze strony wywiadu żywieniowego (\ref{kat:NutritionalInterview})
    \end{enumerate}
    \item\ref{kat:CustomNutritionalInterviewQuestionTemplate}
    \begin{enumerate}[label={\textbf{REG/\protect\threedigits{\arabic{enumi}}}}, wide, labelwidth=!, align=left, leftmargin=3cm, resume]
        %Relacje
        \item Szablon niestandardowego pytania żywieniowego (\ref{kat:CustomNutritionalInterviewQuestionTemplate}) musi mieć dokaldnie jednego autora (\ref{kat:User})
        %CRUD
        \item todo
    \end{enumerate}
    \item\ref{kat:AssignedMealPlan}
    \begin{enumerate}[label={\textbf{REG/\protect\threedigits{\arabic{enumi}}}}, wide, labelwidth=!, align=left, leftmargin=3cm, resume]
        %Relacje
        \item Przypisany jadłospis (\ref{kat:AssignedMealPlan}) musi mieć przydzieloną dokładnie jedną wizytę (\ref{kat:Appointment})
        \item Przypisany jadłospis (\ref{kat:AssignedMealPlan}) musi mieć przydzielony dokładnie jeden jadłospis (\ref{kat:MealPlan})
        %CRUD
        \item todo
    \end{enumerate}
\end{itemize}

\section{Ograniczenia dziedzinowe}\label{sec:restrictions}
\todo{uzupełnić ograniczenia dziedzionowe}

\begin{itemize}[label={\textbf{Ograniczenia dla}}, wide, labelwidth=!, labelindent=0pt]
    \setlength\itemsep{1em}
    \item[\textbf{Ograniczenia}] \textbf{ogólne}
    \begin{enumerate}[label={\textbf{OGR/\protect\threedigits{\arabic{enumi}}}}, wide, labelwidth=!, align=left, leftmargin=3cm]
        \item Wszystkie \textbf{id} muszą być być unikalne
        \item Wszystkie \textbf{id} są wymagane
        \item Wszystkie \textbf{id} są liczbami całkowitymi dodatnimi tworzonymi przez SZBD za pomocą autonumerowania
        \item Wszystkie atrybuty \textbf{language} są wymagane
        \item Wszystkie \textbf{language} są ciągami znaków o długości 2 znaków spełniającymi normę ISO 639-1
        \item Wszystkie \textbf{stemple czasowe} są w formacie YYYY:MM:DD HH:MI:SS
        \item Wszystkie \textbf{daty} są w formacie YYYY:MM:DD
    \end{enumerate}
    \item\ref{kat:User}
    \begin{enumerate}[label={\textbf{OGR/\protect\threedigits{\arabic{enumi}}}}, wide, labelwidth=!, align=left, leftmargin=3cm, resume]
        %Required
        \item Użytkownik musi mieć \ref{kat:User:login}
        \item Użytkownik musi mieć \ref{kat:User:passwordHash}
        \item Użytkownik musi mieć flagę \ref{kat:User:activated}
        \item Użytkownik musi mieć \ref{kat:User:createdDate}
        %Unique
        \item \ref{kat:User:login} musi być unikalny
        \item \ref{kat:User:email} musi być unikalny
        %Type
        \item \ref{kat:User:login} jest ciągiem znaków składającym się z liter, cyfr i dodatkowo mogącym zawierać znaki ".", "\_", "-", "@" o długości od 1 do 50 znaków
        \item \ref{kat:User:passwordHash} jest ciągiem znaków o długości 60 znaków
        \item \ref{kat:User:firstName} jest ciagiem znaków o długości do 50 znaków
        \item \ref{kat:User:lastName} jest ciagiem znaków o długości do 50 znaków
        \item \ref{kat:User:email} jest ciagiem znaków o długości od 5 do 254 znaków
        \item \ref{kat:User:activated} jest typem logicznym
        \item \ref{kat:User:image} jest ciągiem znaków o długości do 256 znaków tworzącym poprawny adres URL
        \item \ref{kat:User:activationKey} jest ciągiem znaków o długości 20 znaków
        \item \ref{kat:User:resetKey} jest ciągiem znaków o długości 20 znaków
        \item \ref{kat:User:resetDate} jest stemplem czasowym
        \item \ref{kat:User:createdDate} jest stemplem czasowym
        \item \ref{kat:User:lastModifiedDate} jest stemplem czasowym
    \end{enumerate}
    \item\ref{kat:UserExtraInfo}
    \begin{enumerate}[label={\textbf{OGR/\protect\threedigits{\arabic{enumi}}}}, wide, labelwidth=!, align=left, leftmargin=3cm, resume]
        \item \ref{kat:UserExtraInfo:gender} jest typu wyliczeniowego i może przyjmować wartości "FAMALE", "MALE", "OTHER"
        \item \ref{kat:UserExtraInfo:dateOfBirth} jest datą
        \item \ref{kat:UserExtraInfo:phoneNumber} jest ciągiem znaków o długości od 1 do 60 znaków
        \item \ref{kat:UserExtraInfo:streetAddress} jest ciągiem znaków o długości od 1 do 255 znaków
        \item \ref{kat:UserExtraInfo:postalCode} jest ciągiem znaków o długości od 1 do 20 znaków
        \item \ref{kat:UserExtraInfo:city} jest ciągiem znaków o długości od 1 do 50 znaków
        \item \ref{kat:UserExtraInfo:country} jest ciągiem znaków o długości od 1 do 50 znaków
        \item \ref{kat:UserExtraInfo:personalDescription} jest ciągiem znaków
        \item Ciągi znaków bez dodatkowych ograniczeń mogą zawierać dowolne znaki dopuszczalne w systemie kodowania UTF-8
    \end{enumerate}
    \item\ref{kat:SiteContent}
    \begin{enumerate}[label={\textbf{OGR/\protect\threedigits{\arabic{enumi}}}}, wide, labelwidth=!, align=left, leftmargin=3cm, resume]
        \item Treść strony musi mieć \ref{kat:SiteContent:ordinalNumber}
        \item Treść strony musi mieć \ref{kat:SiteContent:siteContentType}
        \item Treść strony musi mieć \ref{kat:SiteContent:description}
        \item \ref{kat:SiteContent:ordinalNumber} jest liczbą całkowitą dodatnią
        \item \ref{kat:SiteContent:siteContentType} jest typu wyliczeniowego i może przyjmować wartości "LANDING\_PAGE\_CARD", "TERMS\_OF\_SERVICE", "PRIVACY\_POLICY", "FREQUENTLY\_ASKED\_QUESTION"
        \item \ref{kat:SiteContent:title} jest ciągiem znaków o długości od 1 do 255 znaków
        \item \ref{kat:SiteContent:description} jest ciągiem znaków
    \end{enumerate}
    \item\ref{kat:SiteContentTranslation}
    \begin{enumerate}[label={\textbf{OGR/\protect\threedigits{\arabic{enumi}}}}, wide, labelwidth=!, align=left, leftmargin=3cm, resume]
        \item Tłumaczenie treści strony musi mieć \ref{kat:SiteContentTranslation:description}
        \item \ref{kat:SiteContentTranslation:title} jest ciągiem znaków o długości od 1 do 255 znaków
        \item \ref{kat:SiteContentTranslation:description} jest ciągiem znaków
    \end{enumerate}
    \item\ref{kat:ContactInfo}
    \begin{enumerate}[label={\textbf{OGR/\protect\threedigits{\arabic{enumi}}}}, wide, labelwidth=!, align=left, leftmargin=3cm, resume]
        \item Informacja kontaktowa musi mieć \ref{kat:ContactInfo:contactInfoType}
        \item Informacja kontaktowa musi mieć \ref{kat:ContactInfo:description}
        \item \ref{kat:ContactInfo:contactInfoType} jest typu wyliczeniowego i może przyjmować wartości "PHONE", "EMAIL", "ADDRESS", "FACEBOOK", "TWITTER", "INSTAGRAM", "ANDROID", "IOS", "WORKING\_HOURS", "OTHER"
        \item \ref{kat:ContactInfo:description} jest ciągiem znaków o długości od 1 do 255 znaków
    \end{enumerate}
    \item\ref{kat:Pricing}
    \begin{enumerate}[label={\textbf{OGR/\protect\threedigits{\arabic{enumi}}}}, wide, labelwidth=!, align=left, leftmargin=3cm, resume]
        \item Cennik musi mieć \ref{kat:Pricing:ordinalNumber}
        \item Cennik musi mieć \ref{kat:Pricing:description}
        \item Cennik musi mieć \ref{kat:Pricing:price}
        \item Cennik musi mieć \ref{kat:Pricing:currency}
        \item \ref{kat:Pricing:ordinalNumber} jest dodatnią liczbą całkowitą
        \item \ref{kat:Pricing:title} jest ciągiem znaków o długości od 1 do 255 znaków
        \item \ref{kat:Pricing:description} jest ciągiem znaków
        \item \ref{kat:Pricing:price} jest ciągiem znaków skladającym się z cyfr i znaku "." o długości od 1 do 8 znaków
        \item Jeżeli w \ref{kat:Pricing:price} występuje znak "." to muszą po nim wystepować dokładnie 2 cyfry
        \item \ref{kat:Pricing:currency} jest ciągiem znaków o długości 3 znaków spełniającym normę ISO 4217
    \end{enumerate}
    \item\ref{kat:PricingTranslation}
    \begin{enumerate}[label={\textbf{OGR/\protect\threedigits{\arabic{enumi}}}}, wide, labelwidth=!, align=left, leftmargin=3cm, resume]
        \item Tłumaczenie cennika musi mieć \ref{kat:PricingTranslation:description}
        \item \ref{kat:PricingTranslation:title} jest ciągiem znaków o długości od 1 do 255 znaków
        \item \ref{kat:PricingTranslation:description} jest ciągiem znaków
    \end{enumerate}
    \item\ref{kat:Product}
    \begin{enumerate}[label={\textbf{OGR/\protect\threedigits{\arabic{enumi}}}}, wide, labelwidth=!, align=left, leftmargin=3cm, resume]
        \item todo
    \end{enumerate}
    \item\ref{kat:ProductVersion}
    \begin{enumerate}[label={\textbf{OGR/\protect\threedigits{\arabic{enumi}}}}, wide, labelwidth=!, align=left, leftmargin=3cm, resume]

        \item test1

        \item todo
    \end{enumerate}
    \item\ref{kat:ProductBasicNutritionData}
    \begin{enumerate}[label={\textbf{OGR/\protect\threedigits{\arabic{enumi}}}}, wide, labelwidth=!, align=left, leftmargin=3cm, resume]

        \item test1

        \item todo
    \end{enumerate}
    \item\ref{kat:NutritionData}
    \begin{enumerate}[label={\textbf{OGR/\protect\threedigits{\arabic{enumi}}}}, wide, labelwidth=!, align=left, leftmargin=3cm, resume]

        \item test1

        \item todo
    \end{enumerate}
    \item\ref{kat:NutritionDefinition}
    \begin{enumerate}[label={\textbf{OGR/\protect\threedigits{\arabic{enumi}}}}, wide, labelwidth=!, align=left, leftmargin=3cm, resume]

        \item test1

        \item todo
    \end{enumerate}
    \item\ref{kat:NutritionDefinitionTranslation}
    \begin{enumerate}[label={\textbf{OGR/\protect\threedigits{\arabic{enumi}}}}, wide, labelwidth=!, align=left, leftmargin=3cm, resume]

        \item test1

        \item todo
    \end{enumerate}
    \item\ref{kat:HouseholdMeasure}
    \begin{enumerate}[label={\textbf{OGR/\protect\threedigits{\arabic{enumi}}}}, wide, labelwidth=!, align=left, leftmargin=3cm, resume]

        \item test1

        \item todo
    \end{enumerate}
    \item\ref{kat:ProductSubcategory}
    \begin{enumerate}[label={\textbf{OGR/\protect\threedigits{\arabic{enumi}}}}, wide, labelwidth=!, align=left, leftmargin=3cm, resume]

        \item test1

        \item todo
    \end{enumerate}
    \item\ref{kat:ProductCategory}
    \begin{enumerate}[label={\textbf{OGR/\protect\threedigits{\arabic{enumi}}}}, wide, labelwidth=!, align=left, leftmargin=3cm, resume]

        \item test1

        \item todo
    \end{enumerate}
    \item\ref{kat:ProductCategoryTranslation}
    \begin{enumerate}[label={\textbf{OGR/\protect\threedigits{\arabic{enumi}}}}, wide, labelwidth=!, align=left, leftmargin=3cm, resume]

        \item test1

        \item todo
    \end{enumerate}
    \item\ref{kat:DietType}
    \begin{enumerate}[label={\textbf{OGR/\protect\threedigits{\arabic{enumi}}}}, wide, labelwidth=!, align=left, leftmargin=3cm, resume]

        \item test1

        \item todo
    \end{enumerate}
    \item\ref{kat:DietTypeTranslation}
    \begin{enumerate}[label={\textbf{OGR/\protect\threedigits{\arabic{enumi}}}}, wide, labelwidth=!, align=left, leftmargin=3cm, resume]

        \item test1

        \item todo
    \end{enumerate}
    \item\ref{kat:Recipe}
    \begin{enumerate}[label={\textbf{OGR/\protect\threedigits{\arabic{enumi}}}}, wide, labelwidth=!, align=left, leftmargin=3cm, resume]

        \item test1

        \item todo
    \end{enumerate}
    \item\ref{kat:RecipeVersion}
    \begin{enumerate}[label={\textbf{OGR/\protect\threedigits{\arabic{enumi}}}}, wide, labelwidth=!, align=left, leftmargin=3cm, resume]

        \item test1

        \item todo
    \end{enumerate}
    \item\ref{kat:RecipeBasicNutritionData}
    \begin{enumerate}[label={\textbf{OGR/\protect\threedigits{\arabic{enumi}}}}, wide, labelwidth=!, align=left, leftmargin=3cm, resume]

        \item test1

        \item todo
    \end{enumerate}
    \item\ref{kat:RecipeSection}
    \begin{enumerate}[label={\textbf{OGR/\protect\threedigits{\arabic{enumi}}}}, wide, labelwidth=!, align=left, leftmargin=3cm, resume]

        \item test1

        \item todo
    \end{enumerate}
    \item\ref{kat:ProductPortion}
    \begin{enumerate}[label={\textbf{OGR/\protect\threedigits{\arabic{enumi}}}}, wide, labelwidth=!, align=left, leftmargin=3cm, resume]

        \item test1

        \item todo
    \end{enumerate}
    \item\ref{kat:PreparationStep}
    \begin{enumerate}[label={\textbf{OGR/\protect\threedigits{\arabic{enumi}}}}, wide, labelwidth=!, align=left, leftmargin=3cm, resume]

        \item test1

        \item todo
    \end{enumerate}
    \item\ref{kat:KitchenAppliance}
    \begin{enumerate}[label={\textbf{OGR/\protect\threedigits{\arabic{enumi}}}}, wide, labelwidth=!, align=left, leftmargin=3cm, resume]

        \item test1

        \item todo
    \end{enumerate}
    \item\ref{kat:KitchenApplianceTranslation}
    \begin{enumerate}[label={\textbf{OGR/\protect\threedigits{\arabic{enumi}}}}, wide, labelwidth=!, align=left, leftmargin=3cm, resume]

        \item test1

        \item todo
    \end{enumerate}
    \item\ref{kat:DishType}
    \begin{enumerate}[label={\textbf{OGR/\protect\threedigits{\arabic{enumi}}}}, wide, labelwidth=!, align=left, leftmargin=3cm, resume]

        \item test1

        \item todo
    \end{enumerate}
    \item\ref{kat:DishTypeTranslation}
    \begin{enumerate}[label={\textbf{OGR/\protect\threedigits{\arabic{enumi}}}}, wide, labelwidth=!, align=left, leftmargin=3cm, resume]

        \item test1

        \item todo
    \end{enumerate}
    \item\ref{kat:MealType}
    \begin{enumerate}[label={\textbf{OGR/\protect\threedigits{\arabic{enumi}}}}, wide, labelwidth=!, align=left, leftmargin=3cm, resume]

        \item test1

        \item todo
    \end{enumerate}
    \item\ref{kat:MealTypeTranslation}
    \begin{enumerate}[label={\textbf{OGR/\protect\threedigits{\arabic{enumi}}}}, wide, labelwidth=!, align=left, leftmargin=3cm, resume]

        \item test1

        \item todo
    \end{enumerate}
    \item\ref{kat:MealPlan}
    \begin{enumerate}[label={\textbf{OGR/\protect\threedigits{\arabic{enumi}}}}, wide, labelwidth=!, align=left, leftmargin=3cm, resume]

        \item test1

        \item todo
    \end{enumerate}
    \item\ref{kat:MealPlanDay}
    \begin{enumerate}[label={\textbf{OGR/\protect\threedigits{\arabic{enumi}}}}, wide, labelwidth=!, align=left, leftmargin=3cm, resume]

        \item test1

        \item todo
    \end{enumerate}
    \item\ref{kat:Meal}
    \begin{enumerate}[label={\textbf{OGR/\protect\threedigits{\arabic{enumi}}}}, wide, labelwidth=!, align=left, leftmargin=3cm, resume]

        \item test1

        \item todo
    \end{enumerate}
    \item\ref{kat:MealRecipe}
    \begin{enumerate}[label={\textbf{OGR/\protect\threedigits{\arabic{enumi}}}}, wide, labelwidth=!, align=left, leftmargin=3cm, resume]

        \item test1

        \item todo
    \end{enumerate}
    \item\ref{kat:MealProduct}
    \begin{enumerate}[label={\textbf{OGR/\protect\threedigits{\arabic{enumi}}}}, wide, labelwidth=!, align=left, leftmargin=3cm, resume]

        \item test1

        \item todo
    \end{enumerate}
    \item\ref{kat:MealDefinition}
    \begin{enumerate}[label={\textbf{OGR/\protect\threedigits{\arabic{enumi}}}}, wide, labelwidth=!, align=left, leftmargin=3cm, resume]

        \item test1

        \item todo
    \end{enumerate}
    \item\ref{kat:Appointment}
    \begin{enumerate}[label={\textbf{OGR/\protect\threedigits{\arabic{enumi}}}}, wide, labelwidth=!, align=left, leftmargin=3cm, resume]

        \item test1

        \item todo
    \end{enumerate}
    \item\ref{kat:PatientCard}
    \begin{enumerate}[label={\textbf{OGR/\protect\threedigits{\arabic{enumi}}}}, wide, labelwidth=!, align=left, leftmargin=3cm, resume]

        \item test1

        \item todo
    \end{enumerate}
    \item\ref{kat:AppointmentEvaluation}
    \begin{enumerate}[label={\textbf{OGR/\protect\threedigits{\arabic{enumi}}}}, wide, labelwidth=!, align=left, leftmargin=3cm, resume]

        \item test1

        \item todo
    \end{enumerate}
    \item\ref{kat:BodyMeasurement}
    \begin{enumerate}[label={\textbf{OGR/\protect\threedigits{\arabic{enumi}}}}, wide, labelwidth=!, align=left, leftmargin=3cm, resume]

        \item test1

        \item todo
    \end{enumerate}
    \item\ref{kat:NutritionalInterview}
    \begin{enumerate}[label={\textbf{OGR/\protect\threedigits{\arabic{enumi}}}}, wide, labelwidth=!, align=left, leftmargin=3cm, resume]

        \item test1

        \item todo
    \end{enumerate}
    \item\ref{kat:CustomNutritionalInterviewQuestion}
    \begin{enumerate}[label={\textbf{OGR/\protect\threedigits{\arabic{enumi}}}}, wide, labelwidth=!, align=left, leftmargin=3cm, resume]

        \item test1

        \item todo
    \end{enumerate}
    \item\ref{kat:CustomNutritionalInterviewQuestionTemplate}
    \begin{enumerate}[label={\textbf{OGR/\protect\threedigits{\arabic{enumi}}}}, wide, labelwidth=!, align=left, leftmargin=3cm, resume]

        \item test1

        \item todo
    \end{enumerate}
    \item\ref{kat:AssignedMealPlan}
    \begin{enumerate}[label={\textbf{OGR/\protect\threedigits{\arabic{enumi}}}}, wide, labelwidth=!, align=left, leftmargin=3cm, resume]

        \item test1

        \item todo
    \end{enumerate}
\end{itemize}

\section{Model domenowy}\label{sec:domainModel}
\todo{diagram klas}

\begin{minipage}{\textwidth}
    \begin{figure}[H]
        \centering\includegraphics[scale=0.7]{../uml/class_diagrams/dataTypes.png}
        \caption{Typy danych - diagram klas (opr.wł).}\label{rysunek:class-diagram-data-types}
    \end{figure}
\end{minipage}

\begin{minipage}{\textwidth}
    \begin{figure}[H]
        \centering\includegraphics[scale=0.7]{../uml/class_diagrams/gateway.png}
        \caption{Gateway - diagram klas (opr.wł).}\label{rysunek:class-diagram-gateway}
    \end{figure}
\end{minipage}

\begin{minipage}{\textwidth}
    \begin{figure}[H]
        \centering\includegraphics[scale=0.7]{../uml/class_diagrams/products.png}
        \caption{Produkty - diagram klas (opr.wł).}\label{rysunek:class-diagram-products}
    \end{figure}
\end{minipage}

\begin{minipage}{\textwidth}
    \begin{figure}[H]
        \centering\includegraphics[scale=0.7]{../uml/class_diagrams/recipes.png}
        \caption{Przepisy - diagram klas (opr.wł).}\label{rysunek:class-diagram-recipes}
    \end{figure}
\end{minipage}

\begin{minipage}{\textwidth}
    \begin{figure}[H]
        \centering\includegraphics[scale=0.7]{../uml/class_diagrams/mealplans.png}
        \caption{Jadłospisy - diagram klas (opr.wł).}\label{rysunek:class-diagram-mealplans}
    \end{figure}
\end{minipage}

\begin{minipage}{\textwidth}
    \begin{figure}[H]
        \centering\includegraphics[scale=0.7]{../uml/class_diagrams/appointments.png}
        \caption{Wizyty - diagram klas (opr.wł).}\label{rysunek:class-diagram-appointments}
    \end{figure}
\end{minipage}

\section{Opis podstawowej architektury systemu}\label{sec:basicArchitecture}
\todo{Opisać, że to aplikacja webowa w architekturze mikroserwisów
Wyszczególnienie modułów;
Diagram rozmieszczenia, wzorce projektowe}

%https://martinfowler.com/eaaDev/TemporalObject.html
\thispagestyle{normal}
