% !TeX spellcheck = pl_PL
\chapter{Implementacja}\label{ch:implementation}
\section{Wykorzystywane środowiska i narzędzia programistyczne}\label{sec:dev-tools}
\todo{uzupełnić, opisać, zacytować narzędzia}
Podstawowe kryteria wyboru języka są następujące:

\begin{itemize}
    \item Ścisła kontrola typów
    \item Dobre wsparcie dla paradygmatu programowania obiektowego
    \item Niezależność języka od platformy
\end{itemize}

Wybrane przeze mnie języki spełniające te kryteria to:

\begin{itemize}
    \item W warstwie backendu Java \cite{tech:java}
    \item W warstwie frontendu Typescript \cite{tech:typescript}
\end{itemize}

\begin{enumerate}
    \item Backend
    \begin{itemize}
        \item Java \cite{tech:java}
        \item Spring \cite{tech:spring}
        \item Gradle \cite{tech:gradle}
    \end{itemize}

    \item Frontend
    \begin{itemize}
        \item Typescript \cite{tech:typescript}
        \item Angular \cite{tech:angular}
        \item HTML \cite{tech:html}
        \item Sass \cite{tech:sass}
        \item Bootstrap \cite{tech:bootstrap}
        \item Bootswatch Flatly \cite{tech:bootswatch}
        \item Font Awesome \cite{tech:font-awesome}
    \end{itemize}

    \item Baza danych
    \begin{itemize}
        \item PostgreSQL \cite{tech:postgresql}
        \item Hibernate \cite{tech:hibernate}
        \item Liquibase \cite{tech:liquibase}
    \end{itemize}

    \item Testy
    \begin{itemize}
        \item JUnit\cite{tech:junit}
        \item Mockito \cite{tech:mockito}
        \item Jest \cite{tech:jest}
        \item Protractor \cite{tech:protractor}
    \end{itemize}

    \item Netflix OSS \cite{tech:netflix-oss}
    \begin{itemize}
        \item Eureka Service Discovery \cite{tech:netflix-eureka}
        \item Zuul Proxy \cite{tech:netflix-zuul}
        \item Ribbon load balancer \cite{tech:netflix-ribbon}
        \item Hystrix circuit breaker \cite{tech:netflix-hystrix}
    \end{itemize}
    \item Inne
    \begin{itemize}
        \item JHipster \cite{tech:jhipster}
        \item Gitlab \cite{tech:gitlab}
        \item Docker \cite{tech:docker}
        \item Docker Compose \cite{tech:docker-compose}
        \item Elasticsearch \cite{tech:elasticsearch}
    \end{itemize}
\end{enumerate}
\section{Architektura systemu}\label{sec:system-architecture}
\todo{opisać stack netflix oss}
\todo{diagram rozmieszczenia}
\section{Instalacja oprogramowania}\label{sec:software-installation}
\todo{opisać instalacje oprogramowania i warunki wstępne}
\subsection{Wymagania wstępne}\label{subsec:prerequirements}
\begin{itemize}
    \item Java 8 \cite{tech:java}
    \item Node.js 10 \cite{tech:nodejs}
    \item Docker \cite{tech:docker} + Docker Compose \cite{tech:docker-compose}
\end{itemize}
\section{Prezentacja aplikacji}\label{sec:app-presentation}
\todo{podstawoway opis poruszania się po aplikacji}
\section{Dokumentacja kodu}\label{sec:code-documentation}
\section{Testy}\label{sec:tests}
\thispagestyle{normal}
