\documentclass{dyplom}
\usepackage[utf8]{inputenc}
\usepackage{hyperref}
%%
\usepackage{lipsum}
\usepackage{easy-todo}

% Dane o pracy
\author{Krzysztof Marczyński}
\title{Projekt~i~implementacja systemu~do~zarządzania dietą w~oparciu~o~architekturę~mikroserwisów}
\titlen{Design~and~implementation of~diet~management~system based~on~microservice~architecture}
\promotor{dr inż. Michał Szczepanik}
%\konsultant{dr hab. inż. Kazimerz Kabacki}
\wydzial{Wydział Informatyki i Zarządzania}
\kierunek{Informatyka}
\krotkiestreszczenie{W pracy przedstawiono projekt aplikacji służącej do ukladania diet.}
\slowakluczowe{dieta, jadłospisy, aplikacja webowa, mikroserwisy}

%%%%%%%%%%%%%%%%%%%%%%%%%%%%%%%%%%%%%%%%%%%%%%%%%%%%%%%%%%%%
\begin{document}

    \maketitle
    \pagenumbering{gobble}
    \input{ch0.1_streszczenie}
    \cleardoublepage

    \tableofcontents
    \cleardoublepage

    \pagenumbering{arabic}
    \todo{uzupełnić streszczenie i słowa kluczowe na stronie tytułowej}
    \input{ch0.2_wstep}

    \chapter{Stan wiedzy i techniki w zakresie tematyki pracy}
\section{Przegląd istniejących rozwiązań konkurencyjnych}
\todo{opisać rozwiązania konkurencyjne}
\begin{itemize}
    \item Dietico
    \item TiqDiet
    \item Kcalmar PRO
    \item Dietetyk Pro
\end{itemize}

\begin{minipage}{\textwidth}
    \begin{table}[H]
        \centering\caption{Rozwiązania konkurencyjne - cechy funkcjonalne (opr.wł)\label{tabela:rozwiazania-konkurencyjne-funkcjonalne}}
        \begin{tabular}{|P{.22\textwidth}|P{.09\textwidth}|P{.09\textwidth}|P{.09\textwidth}|P{.09\textwidth}|P{.09\textwidth}|P{.09\textwidth}|}
            \hline
                                                & Tiqdiet   & Kcalmar Pro   & Dietetyk Pro & Aliant         & Dietico   & Vitme   \\ \hline
            Tworzenie jadłospisów               & TAK       & TAK           & TAK          & TAK            & TAK       & TAK     \\ \hline
            Gotowe szablony diet                & TAK       & TAK           & TAK          & TAK            & NIE       & NIE     \\ \hline
            Zapis diety do pliku                & TAK       & TAK           & TAK          & TAK            & TAK       & TAK     \\ \hline
            Wysyłanie diet do pacjenta          & TAK       & TAK           & TAK          & TAK            & NIE       & TAK     \\ \hline
            Komunikacja z pacjentem             & TAK       & TAK           & TAK          & NIE            & NIE       & TAK     \\ \hline
            Karta pacjenta                      & TAK       & TAK           & TAK          & TAK            & TAK       & TAK     \\ \hline
            Wywiad żywieniowy                   & TAK       & TAK           & TAK          & NIE            & TAK       & TAK     \\ \hline
            Lista zakupów                       & TAK       & TAK           & TAK          & TAK            & TAK       & NIE     \\ \hline
            Dodawanie własnych produktów        & TAK       & TAK           & TAK          & TAK            & TAK       & TAK     \\ \hline
            Dodawanie własnych przepisów        & TAK       & TAK           & TAK          & TAK            & TAK       & TAK     \\ \hline
        \end{tabular}
    \end{table}
\end{minipage}

\begin{minipage}{\textwidth}
    \begin{table}[H]
        \centering\caption{Rozwiązania konkurencyjne - cechy niefunkcjonalne (opr.wł)\label{tabela:rozwiazania-konkurencyjne-niefunkcjonalne}}
        \begin{tabular}{|P{.22\textwidth}|P{.09\textwidth}|P{.09\textwidth}|P{.09\textwidth}|P{.09\textwidth}|P{.09\textwidth}|P{.09\textwidth}|}
            \hline
                                                & Tiqdiet   & Kcalmar Pro   & Dietetyk Pro & Aliant         & Dietico   & Vitme   \\ \hline
            Liczba produktów w bazie            & 1000      & 1400          & 6000         & 3500           & 900       & 5000    \\ \hline
            Liczba gotowych przepisów           & 200       & 800           & 2800         & 1700           & 1900      & 400     \\ \hline
            Praca offline                       & NIE       & NIE           & NIE          & TAK            & NIE       & NIE     \\ \hline
            Praca online                        & TAK       & TAK           & TAK          & NIE            & TAK       & TAK     \\ \hline
            Aplikacja mobilna dla dietetyka     & TAK       & TAK           & TAK          & NIE            & NIE       & NIE     \\ \hline
            Aplikacja mobilna dla pacjenta      & TAK       & TAK           & NIE          & NIE            & NIE       & TAK     \\ \hline
            Dostęp dla pacjenta przez przeglądarkę internetową       & TAK       & TAK           & TAK          & NIE            & NIE       & TAK     \\ \hline
            Darmowy okres testowy               & 14dni     & 14dni         & 7dni         & bezter- minowo & 14dni     & 14dni   \\ \hline
            Cena w abonamencie rocznym          & 199       & 1188          & 246          & 699            & 546       & 219     \\ \hline
        \end{tabular}
    \end{table}
\end{minipage}

\section{Przegląd przydatnych technologii i technik}
\todo{Tutaj opisać architekturę aplikacji webowych. Porównać monolit, soa i mikroserwisy}

\section{Przegląd literatury dietetycznej}

%\todo{https://www.pum.edu.pl/__data/assets/pdf_file/0009/82881/Instrukcja-i-formularz-wywiadu-zywieniowego.pdf}
%\todo{https://www.akademiadietetyki.pl/literatura-obowiazkowa-i-uzupelniajaca/}
%\todo{http://www.uni.olsztyn.pl/wnm1/onkologia/index.php/dietetyka/rok-iii/literatura.html}
%\todo{http://www.mckp.uj.edu.pl/cm/uploads/2017/08/7.-Zywienie-literatura.pdf}

\thispagestyle{normal}

    \chapter{Stan wiedzy i techniki w zakresie tematyki pracy}
\section{Przegląd istniejących rozwiązań konkurencyjnych}
\todo{opisać rozwiązania konkurencyjne}
\begin{itemize}
    \item Dietico
    \item TiqDiet
    \item Kcalmar PRO
    \item Dietetyk Pro
\end{itemize}

\section{Przegląd przydatnych technologii i technik}
\todo{Tutaj opisać architekturę aplikacji webowych. Porównać monolit, soa i mikroserwisy}

\thispagestyle{normal}

    \chapter{Projekt}

\section{Przypadki użycia}
\todo{diagram przypadków użycia}

\begin{minipage}{\textwidth}
    \begin{figure}[H]
        \centering\includegraphics[scale=0.55]{../uml/use_case_diagrams/users.png}
        \caption{Użytkownicy - diagram przypadków użycia (opr.wł).}\label{rysunek:use-case-diagram-users}
    \end{figure}
\end{minipage}

\begin{minipage}{\textwidth}
    \begin{figure}[H]
        \centering\includegraphics[scale=0.55]{../uml/use_case_diagrams/gateway.png}
        \caption{Gateway - diagram przypadków użycia (opr.wł).}\label{rysunek:use-case-diagram-gateway}
    \end{figure}
\end{minipage}

\begin{minipage}{\textwidth}
    \begin{figure}[H]
        \centering\includegraphics[scale=0.55]{../uml/use_case_diagrams/products.png}
        \caption{Produkty - diagram przypadków użycia (opr.wł).}\label{rysunek:use-case-diagram-products}
    \end{figure}
\end{minipage}

\begin{minipage}{\textwidth}
    \begin{figure}[H]
        \centering\includegraphics[scale=0.55]{../uml/use_case_diagrams/recipes.png}
        \caption{Przepisy - diagram przypadków użycia (opr.wł).}\label{rysunek:use-case-diagram-recipes}
    \end{figure}
\end{minipage}

\begin{minipage}{\textwidth}
    \begin{figure}[H]
        \centering\includegraphics[scale=0.55]{../uml/use_case_diagrams/mealplans.png}
        \caption{Jadłospisy - diagram przypadków użycia (opr.wł).}\label{rysunek:use-case-diagram-mealplans}
    \end{figure}
\end{minipage}

\begin{minipage}{\textwidth}
    \begin{figure}[H]
        \centering\includegraphics[scale=0.55]{../uml/use_case_diagrams/appointments.png}
        \caption{Wizyty - diagram przypadków użycia (opr.wł).}\label{rysunek:use-case-diagram-appointments}
    \end{figure}
\end{minipage}

\section{Prototyp interfejsu}
\todo{mockupy}
\begin{minipage}{\textwidth}
    \begin{figure}[H]
        \centering\includegraphics[width=0.9\textwidth]{img/mockups/mockup1.png}
        \caption{Mockup1 (opr.wł).}\label{rysunek:mockup1}
    \end{figure}
\end{minipage}

\begin{minipage}{\textwidth}
    \begin{figure}[H]
        \centering\includegraphics[width=0.9\textwidth]{img/mockups/mockup2.png}
        \caption{Mockup2 (opr.wł).}\label{rysunek:mockup2}
    \end{figure}
\end{minipage}

\begin{minipage}{\textwidth}
    \begin{figure}[H]
        \centering\includegraphics[width=0.9\textwidth]{img/mockups/mockup3.png}
        \caption{Mockup3 (opr.wł).}\label{rysunek:mockup3}
    \end{figure}
\end{minipage}

\begin{minipage}{\textwidth}
    \begin{figure}[H]
        \centering\includegraphics[width=0.9\textwidth]{img/mockups/mockup4.png}
        \caption{Mockup4 (opr.wł).}\label{rysunek:mockup4}
    \end{figure}
\end{minipage}

\begin{minipage}{\textwidth}
    \begin{figure}[H]
        \centering\includegraphics[width=0.9\textwidth]{img/mockups/mockup5.png}
        \caption{Mockup5 (opr.wł).}\label{rysunek:mockup5}
    \end{figure}
\end{minipage}

\begin{minipage}{\textwidth}
    \begin{figure}[H]
        \centering\includegraphics[width=0.9\textwidth]{img/mockups/mockup6.png}
        \caption{Mockup6 (opr.wł).}\label{rysunek:mockup6}
    \end{figure}
\end{minipage}

\begin{minipage}{\textwidth}
    \begin{figure}[H]
        \centering\includegraphics[width=0.9\textwidth]{img/mockups/mockup7.png}
        \caption{Mockup7 (opr.wł).}\label{rysunek:mockup7}
    \end{figure}
\end{minipage}

\begin{minipage}{\textwidth}
    \begin{figure}[H]
        \centering\includegraphics[width=0.9\textwidth]{img/mockups/mockup8.png}
        \caption{Mockup8 (opr.wł).}\label{rysunek:mockup8}
    \end{figure}
\end{minipage}

\begin{minipage}{\textwidth}
    \begin{figure}[H]
        \centering\includegraphics[width=0.9\textwidth]{img/mockups/mockup9.png}
        \caption{Mockup9 (opr.wł).}\label{rysunek:mockup9}
    \end{figure}
\end{minipage}

\begin{minipage}{\textwidth}
    \begin{figure}[H]
        \centering\includegraphics[width=0.9\textwidth]{img/mockups/mockup10.png}
        \caption{Mockup10 (opr.wł).}\label{rysunek:mockup10}
    \end{figure}
\end{minipage}

\begin{minipage}{\textwidth}
    \begin{figure}[H]
        \centering\includegraphics[width=0.9\textwidth]{img/mockups/mockup11.png}
        \caption{Mockup11 (opr.wł).}\label{rysunek:mockup11}
    \end{figure}
\end{minipage}

\begin{minipage}{\textwidth}
    \begin{figure}[H]
        \centering\includegraphics[width=0.9\textwidth]{img/mockups/mockup12.png}
        \caption{Mockup12 (opr.wł).}\label{rysunek:mockup12}
    \end{figure}
\end{minipage}

\begin{minipage}{\textwidth}
    \begin{figure}[H]
        \centering\includegraphics[width=0.9\textwidth]{img/mockups/mockup13.png}
        \caption{Mockup13 (opr.wł).}\label{rysunek:mockup13}
    \end{figure}
\end{minipage}

\begin{minipage}{\textwidth}
    \begin{figure}[H]
        \centering\includegraphics[width=0.9\textwidth]{img/mockups/mockup14.png}
        \caption{Mockup14 (opr.wł).}\label{rysunek:mockup14}
    \end{figure}
\end{minipage}

\begin{minipage}{\textwidth}
    \begin{figure}[H]
        \centering\includegraphics[width=0.9\textwidth]{img/mockups/mockup15.png}
        \caption{Mockup15 (opr.wł).}\label{rysunek:mockup15}
    \end{figure}
\end{minipage}

\begin{minipage}{\textwidth}
    \begin{figure}[H]
        \centering\includegraphics[width=0.9\textwidth]{img/mockups/mockup16.png}
        \caption{Mockup16 (opr.wł).}\label{rysunek:mockup16}
    \end{figure}
\end{minipage}

\section{Kategorie}
\todo{uzupełnić kategorie}

\begin{enumerate}[label={\textbf{KAT/\protect\threedigits{\theenumi}}}, wide, labelwidth=!, labelindent=0pt, labelsep=0pt, series=reqs]
    \setlength\itemsep{1em}
    \req{User} \label{kat:User} (Użytkownik)

    Opis: \lipsum[1]
    \par
    Atrybuty:
    \begin{itemize}[series=atr]
        \atr{id} \label{kat:User:id}
        \atr{login} \label{kat:User:login}
        \atr{passwordHash} \label{kat:User:passwordHash}
        \atr{firstName} \label{kat:User:firstName}
        \atr{lastName} \label{kat:User:lastName}
        \atr{email} \label{kat:User:email}
        \atr{image} \label{kat:User:image}
        \atr{activated} \label{kat:User:activated}
        \atr{langKey} \label{kat:User:langKey}
        \atr{activationKey} \label{kat:User:activationKey}
        \atr{createdBy} \label{kat:User:createdBy}
        \atr{createdDate} \label{kat:User:createdDate}
        \atr{resetDate} \label{kat:User:resetDate}
        \atr{lastModifiedBy} \label{kat:User:lastModifiedBy}
        \atr{lastModifiedDate} \label{kat:User:lastModifiedDate}
    \end{itemize}

    \req{Authority} \label{kat:Authority} (Rola)

    Opis: \lipsum[1]
    \par
    Atrybuty:
    \begin{itemize}[series=atr]
        \atr{name} \label{kat:Authority:name}
    \end{itemize}

    \req{UserExtraInfo} \label{kat:UserExtraInfo} (Dodatkowe Informacje Użytkownika)

    Opis: \lipsum[1]
    \par
    Atrybuty:
    \begin{itemize}[series=atr]
        \atr{id} \label{kat:UserExtraInfo:id}
        \atr{gender} \label{kat:UserExtraInfo:gender}
        \atr{dateOfBirth} \label{kat:UserExtraInfo:dateOfBirth}
        \atr{phoneNumber} \label{kat:UserExtraInfo:phoneNumber}
        \atr{streetAddress} \label{kat:UserExtraInfo:streetAddress}
        \atr{postalCode} \label{kat:UserExtraInfo:postalCode}
        \atr{city} \label{kat:UserExtraInfo:city}
        \atr{country} \label{kat:UserExtraInfo:country}
        \atr{personalDescription} \label{kat:UserExtraInfo:personalDescription}
    \end{itemize}

    \req{SiteContent} \label{kat:SiteContent} (Treść Strony)

    Opis: \lipsum[1]
    \par
    Atrybuty:
    \begin{itemize}[series=atr]
        \atr{id} \label{kat:SiteContent:id}
        \atr{ordinalNumber} \label{kat:SiteContent:ordinalNumber}
        \atr{siteContentType} \label{kat:SiteContent:siteContentType}
        \atr{title} \label{kat:SiteContent:title}
        \atr{description} \label{kat:SiteContent:description}
    \end{itemize}

    \req{SiteContentTranslation} \label{kat:SiteContentTranslation} (Tłumaczenie Treści Strony)

    Opis: \lipsum[1]
    \par
    Atrybuty:
    \begin{itemize}[series=atr]
        \atr{id} \label{kat:SiteContentTranslation:id}
        \atr{title} \label{kat:SiteContentTranslation:title}
        \atr{description} \label{kat:SiteContentTranslation:description}
        \atr{language} \label{kat:SiteContentTranslation:language}
    \end{itemize}

    \req{ContactInfo} \label{kat:ContactInfo} (Informacje Kontaktowe)

    Opis: \lipsum[1]
    \par
    Atrybuty:
    \begin{itemize}[series=atr]
        \atr{id} \label{kat:ContactInfo:id}
        \atr{contactInfoType} \label{kat:ContactInfo:contactInfoType}
        \atr{description} \label{kat:ContactInfo:description}
    \end{itemize}

    \req{Pricing} \label{kat:Pricing} (Cennik)

    Opis: \lipsum[1]
    \par
    Atrybuty:
    \begin{itemize}[series=atr]
        \atr{id} \label{kat:Pricing:id}
        \atr{ordinalNumber} \label{kat:Pricing:ordinalNumber}
        \atr{title} \label{kat:Pricing:title}
        \atr{description} \label{kat:Pricing:description}
        \atr{price} \label{kat:Pricing:price}
        \atr{currency} \label{kat:Pricing:currency}
    \end{itemize}

    \req{PricingTranslation} \label{kat:PricingTranslation} (Tłumaczenie Cennika)

    Opis: \lipsum[1]
    \par
    Atrybuty:
    \begin{itemize}[series=atr]
        \atr{id} \label{kat:PricingTranslation:id}
        \atr{title} \label{kat:PricingTranslation:title}
        \atr{description} \label{kat:PricingTranslation:description}
        \atr{language} \label{kat:PricingTranslation:language}
    \end{itemize}

    \req{Product} \label{kat:Product} (Produkt)

    Opis: \lipsum[1]
    \par
    Atrybuty:
    \begin{itemize}[series=atr]
        \atr{id} \label{kat:Product:id}
        \atr{source} \label{kat:Product:source}
        \atr{isPublic} \label{kat:Product:isPublic}
        \atr{language} \label{kat:Product:language}
    \end{itemize}

    \req{ProductVersion} \label{kat:ProductVersion} (Wersja Produktu)

    Opis: \lipsum[1]
    \par
    Atrybuty:
    \begin{itemize}[series=atr]
        \atr{id} \label{kat:ProductVersion:id}
        \atr{editTimestamp} \label{kat:ProductVersion:editTimestamp}
        \atr{description} \label{kat:ProductVersion:description}
    \end{itemize}

    \req{ProductBasicNutritionData} \label{kat:ProductBasicNutritionData} (Podstawowe Składniki Odżywcze Produktu)

    Opis: \lipsum[1]
    \par
    Atrybuty:
    \begin{itemize}[series=atr]
        \atr{id} \label{kat:ProductBasicNutritionData:id}
        \atr{energy} \label{kat:ProductBasicNutritionData:energy}
        \atr{protein} \label{kat:ProductBasicNutritionData:protein}
        \atr{fat} \label{kat:ProductBasicNutritionData:fat}
        \atr{carbohydrates} \label{kat:ProductBasicNutritionData:carbohydrates}
    \end{itemize}

    \req{NutritionData} \label{kat:NutritionData} (Wartość Odżywcza)

    Opis: \lipsum[1]
    \par
    Atrybuty:
    \begin{itemize}[series=atr]
        \atr{id} \label{kat:NutritionData:id}
        \atr{nutritionValue} \label{kat:NutritionData:nutritionValue}
    \end{itemize}

    \req{NutritionDefinition} \label{kat:NutritionDefinition} (Definicja Wartości Odżywczej)

    Opis: \lipsum[1]
    \par
    Atrybuty:
    \begin{itemize}[series=atr]
        \atr{id} \label{kat:NutritionDefinition:id}
        \atr{tag} \label{kat:NutritionDefinition:tag}
        \atr{description} \label{kat:NutritionDefinition:description}
        \atr{units} \label{kat:NutritionDefinition:units}
        \atr{decimalPlaces} \label{kat:NutritionDefinition:decimalPlaces}
    \end{itemize}

    \req{NutritionDefinitionTranslation} \label{kat:NutritionDefinitionTranslation} (Tłumaczenie Definicji Wartości Odżywczej)

    Opis: \lipsum[1]
    \par
    Atrybuty:
    \begin{itemize}[series=atr]
        \atr{id} \label{kat:NutritionDefinitionTranslation:id}
        \atr{translation} \label{kat:NutritionDefinitionTranslation:translation}
        \atr{language} \label{kat:NutritionDefinitionTranslation:language}
    \end{itemize}

    \req{HouseholdMeasure} \label{kat:HouseholdMeasure} (Miara Domowa)

    Opis: \lipsum[1]
    \par
    Atrybuty:
    \begin{itemize}[series=atr]
        \atr{id} \label{kat:HouseholdMeasure:id}
        \atr{description} \label{kat:HouseholdMeasure:description}
        \atr{gramsWeight} \label{kat:HouseholdMeasure:gramsWeight}
        \atr{isVisible} \label{kat:HouseholdMeasure:isVisible}
    \end{itemize}

    \req{ProductSubcategory} \label{kat:ProductSubcategory} (Podkategoria Produktu)

    Opis: \lipsum[1]
    \par
    Atrybuty:
    \begin{itemize}[series=atr]
        \atr{id} \label{kat:ProductSubcategory:id}
        \atr{description} \label{kat:ProductSubcategory:description}
    \end{itemize}

    \req{ProductCategory} \label{kat:ProductCategory} (Kategoria Produktu)

    Opis: \lipsum[1]
    \par
    Atrybuty:
    \begin{itemize}[series=atr]
        \atr{id} \label{kat:ProductCategory:id}
        \atr{description} \label{kat:ProductCategory:description}
    \end{itemize}

    \req{ProductCategoryTranslation} \label{kat:ProductCategoryTranslation} (Tłumaczenie Kategorii Produktu)

    Opis: \lipsum[1]
    \par
    Atrybuty:
    \begin{itemize}[series=atr]
        \atr{id} \label{kat:ProductCategoryTranslation:id}
        \atr{translation} \label{kat:ProductCategoryTranslation:translation}
        \atr{language} \label{kat:ProductCategoryTranslation:language}
    \end{itemize}

    \req{DietType} \label{kat:DietType} (Typ Diety)

    Opis: \lipsum[1]
    \par
    Atrybuty:
    \begin{itemize}[series=atr]
        \atr{id} \label{kat:DietType:id}
        \atr{name} \label{kat:DietType:name}
    \end{itemize}

    \req{DietTypeTranslation} \label{kat:DietTypeTranslation} (Tłumaczenie Typu Diety)

    Opis: \lipsum[1]
    \par
    Atrybuty:
    \begin{itemize}[series=atr]
        \atr{id} \label{kat:DietTypeTranslation:id}
        \atr{translation} \label{kat:DietTypeTranslation:translation}
        \atr{language} \label{kat:DietTypeTranslation:language}
    \end{itemize}

    \req{Recipe} \label{kat:Recipe} (Przepis)

    Opis: \lipsum[1]
    \par
    Atrybuty:
    \begin{itemize}[series=atr]
        \atr{id} \label{kat:Recipe:id}
        \atr{isPublic} \label{kat:Recipe:isPublic}
        \atr{language} \label{kat:Recipe:language}
    \end{itemize}

    \req{RecipeVersion} \label{kat:RecipeVersion} (Wersja Przepisu)

    Opis: \lipsum[1]
    \par
    Atrybuty:
    \begin{itemize}[series=atr]
        \atr{id} \label{kat:RecipeVersion:id}
        \atr{editTimestamp} \label{kat:RecipeVersion:editTimestamp}
        \atr{name} \label{kat:RecipeVersion:name}
        \atr{preparationTimeMinutes} \label{kat:RecipeVersion:preparationTimeMinutes}
        \atr{numberOfPortions} \label{kat:RecipeVersion:numberOfPortions}
        \atr{image} \label{kat:RecipeVersion:image}
        \atr{totalGramsWeight} \label{kat:RecipeVersion:totalGramsWeight}
    \end{itemize}

    \req{RecipeBasicNutritionData} \label{kat:RecipeBasicNutritionData} (Podstawowe Wartości Odżywcze Przepisu)

    Opis: \lipsum[1]
    \par
    Atrybuty:
    \begin{itemize}[series=atr]
        \atr{id} \label{kat:RecipeBasicNutritionData:id}
        \atr{energy} \label{kat:RecipeBasicNutritionData:energy}
        \atr{protein} \label{kat:RecipeBasicNutritionData:protein}
        \atr{fat} \label{kat:RecipeBasicNutritionData:fat}
        \atr{carbohydrates} \label{kat:RecipeBasicNutritionData:carbohydrates}
    \end{itemize}

    \req{RecipeSection} \label{kat:RecipeSection} (Sekcja Przepisu)

    Opis: \lipsum[1]
    \par
    Atrybuty:
    \begin{itemize}[series=atr]
        \atr{id} \label{kat:RecipeSection:id}
        \atr{sectionName} \label{kat:RecipeSection:sectionName}
    \end{itemize}

    \req{ProductPortion} \label{kat:ProductPortion} (Porcja Produktu)

    Opis: \lipsum[1]
    \par
    Atrybuty:
    \begin{itemize}[series=atr]
        \atr{id} \label{kat:ProductPortion:id}
        \atr{amount} \label{kat:ProductPortion:amount}
    \end{itemize}

    \req{PreparationStep} \label{kat:PreparationStep} (Krok Przygotowania)

    Opis: \lipsum[1]
    \par
    Atrybuty:
    \begin{itemize}[series=atr]
        \atr{id} \label{kat:PreparationStep:id}
        \atr{ordinalNumber} \label{kat:PreparationStep:ordinalNumber}
        \atr{stepDescription} \label{kat:PreparationStep:stepDescription}
    \end{itemize}

    \req{KitchenAppliance} \label{kat:KitchenAppliance} (Sprzęt Kuchenny)

    Opis: \lipsum[1]
    \par
    Atrybuty:
    \begin{itemize}[series=atr]
        \atr{id} \label{kat:KitchenAppliance:id}
        \atr{name} \label{kat:KitchenAppliance:name}
    \end{itemize}

    \req{KitchenApplianceTranslation} \label{kat:KitchenApplianceTranslation} (Tłumaczenie Sprzętu Kuchennego)

    Opis: \lipsum[1]
    \par
    Atrybuty:
    \begin{itemize}[series=atr]
        \atr{id} \label{kat:KitchenApplianceTranslation:id}
        \atr{translation} \label{kat:KitchenApplianceTranslation:translation}
        \atr{language} \label{kat:KitchenApplianceTranslation:language}
    \end{itemize}

    \req{DishType} \label{kat:DishType} (Typ Dania)

    Opis: \lipsum[1]
    \par
    Atrybuty:
    \begin{itemize}[series=atr]
        \atr{id} \label{kat:DishType:id}
        \atr{description} \label{kat:DishType:description}
    \end{itemize}

    \req{DishTypeTranslation} \label{kat:DishTypeTranslation} (Tłumaczenie Typu Dania)

    Opis: \lipsum[1]
    \par
    Atrybuty:
    \begin{itemize}[series=atr]
        \atr{id} \label{kat:DishTypeTranslation:id}
        \atr{translation} \label{kat:DishTypeTranslation:translation}
        \atr{language} \label{kat:DishTypeTranslation:language}
    \end{itemize}

    \req{MealType} \label{kat:MealType} (Typ Posiłku)

    Opis: \lipsum[1]
    \par
    Atrybuty:
    \begin{itemize}[series=atr]
        \atr{id} \label{kat:MealType:id}
        \atr{name} \label{kat:MealType:name}
    \end{itemize}

    \req{MealTypeTranslation} \label{kat:MealTypeTranslation} (Tłumaczenie Typu Posiłku)

    Opis: \lipsum[1]
    \par
    Atrybuty:
    \begin{itemize}[series=atr]
        \atr{id} \label{kat:MealTypeTranslation:id}
        \atr{translation} \label{kat:MealTypeTranslation:translation}
        \atr{language} \label{kat:MealTypeTranslation:language}
    \end{itemize}


    \req{MealPlan} \label{kat:MealPlan} (Jadłospis)

    Opis: \lipsum[1]
    \par
    Atrybuty:
    \begin{itemize}[series=atr]
        \atr{id} \label{kat:MealPlan:id}
        \atr{creationTimestamp} \label{kat:MealPlan:creationTimestamp}
        \atr{editTimestamp} \label{kat:MealPlan:editTimestamp}
        \atr{name} \label{kat:MealPlan:name}
        \atr{isVisible} \label{kat:MealPlan:isVisible}
        \atr{language} \label{kat:MealPlan:language}
        \atr{numberOfDays} \label{kat:MealPlan:numberOfDays}
        \atr{numberOfMealsPerDay} \label{kat:MealPlan:numberOfMealsPerDay}
        \atr{totalDailyEnergy} \label{kat:MealPlan:totalDailyEnergy}
        \atr{percentOfProtein} \label{kat:MealPlan:percentOfProtein}
        \atr{percentOfFat} \label{kat:MealPlan:percentOfFat}
        \atr{percentOfCarbohydrates} \label{kat:MealPlan:percentOfCarbohydrates}
    \end{itemize}

    \req{MealPlanDay} \label{kat:MealPlanDay} (Dzień Jadłospisu)

    Opis: \lipsum[1]
    \par
    Atrybuty:
    \begin{itemize}[series=atr]
        \atr{id} \label{kat:MealPlanDay:id}
        \atr{ordinalNumber} \label{kat:MealPlanDay:ordinalNumber}
    \end{itemize}

    \req{Meal} \label{kat:Meal} (Posiłek)

    Opis: \lipsum[1]
    \par
    Atrybuty:
    \begin{itemize}[series=atr]
        \atr{id} \label{kat:Meal:id}
        \atr{ordinalNumber} \label{kat:Meal:ordinalNumber}
    \end{itemize}

    \req{MealRecipe} \label{kat:MealRecipe} (Przepis Posiłku)

    Opis: \lipsum[1]
    \par
    Atrybuty:
    \begin{itemize}[series=atr]
        \atr{id} \label{kat:MealRecipe:id}
        \atr{amount} \label{kat:MealRecipe:amount}
    \end{itemize}

    \req{MealProduct} \label{kat:MealProduct} (Produkt Posiłku)

    Opis: \lipsum[1]
    \par
    Atrybuty:
    \begin{itemize}[series=atr]
        \atr{id} \label{kat:MealProduct:id}
        \atr{amount} \label{kat:MealProduct:amount}
    \end{itemize}

    \req{MealDefinition} \label{kat:MealDefinition} (Definicja Posiłku)

    Opis: \lipsum[1]
    \par
    Atrybuty:
    \begin{itemize}[series=atr]
        \atr{id} \label{kat:MealDefinition:id}
        \atr{ordinalNumber} \label{kat:MealDefinition:ordinalNumber}
        \atr{timeOfMeal} \label{kat:MealDefinition:timeOfMeal}
        \atr{percentOfEnergy} \label{kat:MealDefinition:percentOfEnergy}
    \end{itemize}

    \req{Appointment} \label{kat:Appointment} (Wizyta)

    Opis: \lipsum[1]
    \par
    Atrybuty:
    \begin{itemize}[series=atr]
        \atr{id} \label{kat:Appointment:id}
        \atr{appointmentDate} \label{kat:Appointment:appointmentDate}
        \atr{appointmentState} \label{kat:Appointment:appointmentState}
        \atr{generalAdvice} \label{kat:Appointment:generalAdvice}
    \end{itemize}

    \req{PatientCard} \label{kat:PatientCard} (Karta Pacjenta)

    Opis: \lipsum[1]
    \par
    Atrybuty:
    \begin{itemize}[series=atr]
        \atr{id} \label{kat:PatientCard:id}
        \atr{creationDate} \label{kat:PatientCard:creationDate}
    \end{itemize}

    \req{AppointmentEvaluation} \label{kat:AppointmentEvaluation} (Ewaluacja Wizyty)

    Opis: \lipsum[1]
    \par
    Atrybuty:
    \begin{itemize}[series=atr]
        \atr{id} \label{kat:AppointmentEvaluation:id}
        \atr{overallSatisfaction} \label{kat:AppointmentEvaluation:overallSatisfaction}
        \atr{dietitianServiceSatisfaction} \label{kat:AppointmentEvaluation:dietitianServiceSatisfaction}
        \atr{mealPlanOverallSatisfaction} \label{kat:AppointmentEvaluation:mealPlanOverallSatisfaction}
        \atr{mealCostSatisfaction} \label{kat:AppointmentEvaluation:mealCostSatisfaction}
        \atr{mealPreparationTimeSatisfaction} \label{kat:AppointmentEvaluation:mealPreparationTimeSatisfaction}
        \atr{mealComplexityLevelSatisfaction} \label{kat:AppointmentEvaluation:mealComplexityLevelSatisfaction}
        \atr{mealTastefulnessSatisfaction} \label{kat:AppointmentEvaluation:mealTastefulnessSatisfaction}
        \atr{dietaryResultSatisfaction} \label{kat:AppointmentEvaluation:dietaryResultSatisfaction}
        \atr{comment} \label{kat:AppointmentEvaluation:comment}
    \end{itemize}

    \req{BodyMeasurement} \label{kat:BodyMeasurement} (Pomiar Ciała)

    Opis: \lipsum[1]
    \par
    Atrybuty:
    \begin{itemize}[series=atr]
        \atr{id} \label{kat:BodyMeasurement:id}
        \atr{completionDate} \label{kat:BodyMeasurement:completionDate}
        \atr{height} \label{kat:BodyMeasurement:height}
        \atr{weight} \label{kat:BodyMeasurement:weight}
        \atr{waist} \label{kat:BodyMeasurement:waist}
        \atr{percentOfFatTissue} \label{kat:BodyMeasurement:percentOfFatTissue}
        \atr{percentOfWater} \label{kat:BodyMeasurement:percentOfWater}
        \atr{muscleMass} \label{kat:BodyMeasurement:muscleMass}
        \atr{physicalMark} \label{kat:BodyMeasurement:physicalMark}
        \atr{calciumInBones} \label{kat:BodyMeasurement:calciumInBones}
        \atr{basicMetabolism} \label{kat:BodyMeasurement:basicMetabolism}
        \atr{metabolicAge} \label{kat:BodyMeasurement:metabolicAge}
        \atr{visceralDatLevel} \label{kat:BodyMeasurement:visceralDatLevel}
    \end{itemize}

    \req{NutritionalInterview} \label{kat:NutritionalInterview} (Wywiad Żywieniowy)

    Opis: \lipsum[1]
    \par
    Atrybuty:
    \begin{itemize}[series=atr]
        \atr{id} \label{kat:NutritionalInterview:id}
        \atr{completionDate} \label{kat:NutritionalInterview:completionDate}
        \atr{targetWeight} \label{kat:NutritionalInterview:targetWeight}
        \atr{advicePurpose} \label{kat:NutritionalInterview:advicePurpose}
        \atr{physicalActivity} \label{kat:NutritionalInterview:physicalActivity}
        \atr{diseases} \label{kat:NutritionalInterview:diseases}
        \atr{medicines} \label{kat:NutritionalInterview:medicines}
        \atr{jobType} \label{kat:NutritionalInterview:jobType}
        \atr{likedProducts} \label{kat:NutritionalInterview:likedProducts}
        \atr{dislikedProducts} \label{kat:NutritionalInterview:dislikedProducts}
        \atr{foodAllergies} \label{kat:NutritionalInterview:foodAllergies}
        \atr{foodIntolerances} \label{kat:NutritionalInterview:foodIntolerances}
    \end{itemize}

    \req{CustomNutritionalInterviewQuestion} \label{kat:CustomNutritionalInterviewQuestion} (Niestandardowe Pytanie Wywiadu Żywieniowego)

    Opis: \lipsum[1]
    \par
    Atrybuty:
    \begin{itemize}[series=atr]
        \atr{id} \label{kat:CustomNutritionalInterviewQuestion:id}
        \atr{ordinalNumber} \label{kat:CustomNutritionalInterviewQuestion:ordinalNumber}
        \atr{question} \label{kat:CustomNutritionalInterviewQuestion:question}
        \atr{answer} \label{kat:CustomNutritionalInterviewQuestion:answer}
    \end{itemize}

    \req{CustomNutritionalInterviewQuestionTemplate} \label{kat:CustomNutritionalInterviewQuestionTemplate} (Szablon Niestandardowego Pytania Wywiadu Żywieniowego)

    Opis: \lipsum[1]
    \par
    Atrybuty:
    \begin{itemize}[series=atr]
        \atr{id} \label{kat:CustomNutritionalInterviewQuestionTemplate:id}
        \atr{question} \label{kat:CustomNutritionalInterviewQuestionTemplate:question}
        \atr{language} \label{kat:CustomNutritionalInterviewQuestionTemplate:language}
    \end{itemize}

    \req{AssignedMealPlan} \label{kat:AssignedMealPlan} (Przypisany Jadłospis)

    Opis: \lipsum[1]
    \par
    Atrybuty:
    \begin{itemize}[series=atr]
        \atr{id} \label{kat:AssignedMealPlan:id}
        \atr{assigmentTime} \label{kat:AssignedMealPlan:assigmentTime}
    \end{itemize}

\end{enumerate}

\section {Reguły funkcjonowania}
\todo{uzupełnić reguły funkcjonowania, i.e. związki pomiędzy encjami}

\begin{itemize}[label={\textbf{Reguły dla}}, wide, labelwidth=!, labelindent=0pt]
    \setlength\itemsep{1em}
    \item\ref{kat:User}
    \begin{enumerate}[label={\textbf{REG/\protect\threedigits{\arabic{enumi}}}}, wide, labelwidth=!, align=left, leftmargin=3cm]
        %Relacje
        \item Użytkownik (\ref{kat:User}) nie musi mieć musi mieć żadnych dodatkowych informacji (\ref{kat:UserExtraInfo})
        \item Użytkownik (\ref{kat:User}) może mieć maksymalnie jedne dodatkowe informacje (\ref{kat:UserExtraInfo})
        \item Użytkownik (\ref{kat:User}) musi mieć musi mieć przynajmniej jedną rolę (\ref{kat:Authority})
        \item Użytkownik (\ref{kat:User}) może mieć wiele ról (\ref{kat:Authority})
        %CRUD
        \item \role{Gość} może dodawać nowego użytkownika (\ref{kat:User})
        \item \role{Użytkownik} może wyświetlać, edytować i usuwać swoje dane użytkownika (\ref{kat:User})
        \item \role{Dietetyk} może wyświetlać podstawowe dane (\ref{kat:User}) \role{Pacjenta}, którego kartotekę prowadzi
        \item \role{Administrator} może wyświetlać i usuwać dane użytkownika (\ref{kat:User})
    \end{enumerate}
    \item\ref{kat:Authority}
    \begin{enumerate}[label={\textbf{REG/\protect\threedigits{\arabic{enumi}}}}, wide, labelwidth=!, align=left, leftmargin=3cm, resume]
        %Relacje
        %CRUD
        \item \role{Administrator} może dodawać, wyświetlać, edytować i usuwać dane roli (\ref{kat:Authority})
    \end{enumerate}
    \item\ref{kat:UserExtraInfo}
    \begin{enumerate}[label={\textbf{REG/\protect\threedigits{\arabic{enumi}}}}, wide, labelwidth=!, align=left, leftmargin=3cm, resume]
        %Relacje
        \item Dodatkowe informacje (\ref{kat:UserExtraInfo}) muszą być przypisane do dokładnie jednego użytkownika (\ref{kat:User})
        %CRUD
        \item \role{Użytkownik} może dodawać, wyświetlać, edytować i usuwać swoje dodatkowe informacje (\ref{kat:UserExtraInfo})
        \item \role{Użytkownik} może wyświetlać dodatkowe informacje (\ref{kat:UserExtraInfo}) \role{Dietetyka}
        \item \role{Dietetyk} może wyświetlać dodatkowe dane (\ref{kat:UserExtraInfo}) \role{Pacjenta}, którego kartotekę prowadzi
    \end{enumerate}
    \item\ref{kat:SiteContent}
    \begin{enumerate}[label={\textbf{REG/\protect\threedigits{\arabic{enumi}}}}, wide, labelwidth=!, align=left, leftmargin=3cm, resume]
        %Relacje
        \item Treść strony (\ref{kat:SiteContent}) nie musi mieć żadnego tłumaczenia (\ref{kat:SiteContentTranslation})
        \item Treść strony (\ref{kat:SiteContent}) może mieć wiele tłumaczeń (\ref{kat:SiteContentTranslation})
        %CRUD
        \item \role{Gość} może wyświetlać dane treści strony (\ref{kat:SiteContent})
        \item \role{Użytkownik} może wyświetlać dane treści strony (\ref{kat:SiteContent})
        \item \role{Administrator} może dodawać, edytować i usuwać dane treści strony (\ref{kat:SiteContent})
    \end{enumerate}
    \item\ref{kat:SiteContentTranslation}
    \begin{enumerate}[label={\textbf{REG/\protect\threedigits{\arabic{enumi}}}}, wide, labelwidth=!, align=left, leftmargin=3cm, resume]
        %Relacje
        \item Tłumaczenie treści strony (\ref{kat:SiteContentTranslation}) musi być przypisane do dokładnie jednej treści strony (\ref{kat:SiteContent})
        %CRUD
        \item \role{Gość} może wyświetlać dane tłumaczenia treści strony (\ref{kat:SiteContentTranslation})
        \item \role{Użytkownik} może wyświetlać dane tłumaczenia treści strony (\ref{kat:SiteContentTranslation})
        \item \role{Administrator} może dodawać, edytować i usuwać dane tłumaczenia treści strony (\ref{kat:SiteContentTranslation})
    \end{enumerate}
    \item\ref{kat:ContactInfo}
    \begin{enumerate}[label={\textbf{REG/\protect\threedigits{\arabic{enumi}}}}, wide, labelwidth=!, align=left, leftmargin=3cm, resume]
        %CRUD
        \item \role{Gość} może wyświetlać dane informacji kontaktowych (\ref{kat:ContactInfo})
        \item \role{Użytkownik} może wyświetlać dane informacji kontaktowych (\ref{kat:ContactInfo})
        \item \role{Administrator} może dodawać, edytować i usuwać dane informacji kontaktowych (\ref{kat:ContactInfo})
    \end{enumerate}
    \item\ref{kat:Pricing}
    \begin{enumerate}[label={\textbf{REG/\protect\threedigits{\arabic{enumi}}}}, wide, labelwidth=!, align=left, leftmargin=3cm, resume]
        %Relacje
        \item Cennik (\ref{kat:Pricing}) nie musi mieć przypisanych żadnych tłumaczeń (\ref{kat:PricingTranslation})
        \item Cennik (\ref{kat:Pricing}) może mieć przypisane wiele tłumaczeń (\ref{kat:PricingTranslation})
        %CRUD
        \item \role{Gość} może wyświetlać dane cennika (\ref{kat:Pricing})
        \item \role{Użytkownik} może wyświetlać dane cennika (\ref{kat:Pricing})
        \item \role{Administrator} może dodawać, edytować i usuwać dane cennika (\ref{kat:Pricing})
    \end{enumerate}
    \item\ref{kat:PricingTranslation}
    \begin{enumerate}[label={\textbf{REG/\protect\threedigits{\arabic{enumi}}}}, wide, labelwidth=!, align=left, leftmargin=3cm, resume]
        %Relacje
        \item Tłumaczenie cennika (\ref{kat:PricingTranslation}) musi być przypisane do dokładnie jednego cennika (\ref{kat:Pricing})
        %CRUD
        \item \role{Gość} może wyświetlać dane tłumaczenia cennika (\ref{kat:PricingTranslation})
        \item \role{Użytkownik} może wyświetlać dane tłumaczenia cennika (\ref{kat:PricingTranslation})
        \item \role{Administrator} może dodawać, edytować i usuwać dane tłumaczenia cennika (\ref{kat:PricingTranslation})
    \end{enumerate}
    \item\ref{kat:Product}
    \begin{enumerate}[label={\textbf{REG/\protect\threedigits{\arabic{enumi}}}}, wide, labelwidth=!, align=left, leftmargin=3cm, resume]
        %Relacje
        \item Produkt (\ref{kat:Product}) musi mieć przynajmniej jedną wersję (\ref{kat:ProductVersion})
        \item Produkt (\ref{kat:Product}) może mieć wiele wersji (\ref{kat:ProductVersion})
        \item Produkt (\ref{kat:Product}) nie musi mieć zdefiniowanego autora (\ref{kat:User})
        \item Produkt (\ref{kat:Product}) może mieć maksymalnie jednego autora (\ref{kat:User})
        %CRUD
        \item todo
    \end{enumerate}
    \item\ref{kat:ProductVersion}
    \begin{enumerate}[label={\textbf{REG/\protect\threedigits{\arabic{enumi}}}}, wide, labelwidth=!, align=left, leftmargin=3cm, resume]
        %Relacje
        \item Wersja produktu (\ref{kat:ProductVersion}) musi być przypisana do dokładnie jednego produktu  (\ref{kat:Product})
        \item Wersja produktu (\ref{kat:ProductVersion}) musi być przypisana do dokładnie jednych podstawowych wartości odżywczych (\ref{kat:ProductBasicNutritionData})
        \item Wersja produktu (\ref{kat:ProductVersion}) nie musi mieć zdefiniowanych żadnych wartości odżywczych (\ref{kat:NutritionData})
        \item Wersja produktu (\ref{kat:ProductVersion}) może mieć zdefiniowane wiele wartości odżywczych (\ref{kat:NutritionData})
        \item Wersja produktu (\ref{kat:ProductVersion}) nie musi mieć zdefiniowanych żadnych miar domowych (\ref{kat:HouseholdMeasure})
        \item Wersja produktu (\ref{kat:ProductVersion}) może mieć zdefiniowane wiele miar domowych (\ref{kat:HouseholdMeasure})
        \item Wersja produktu (\ref{kat:ProductVersion}) musi należeć do dokładnie jednej podkategorii (\ref{kat:ProductSubcategory})
        \item Wersja produktu (\ref{kat:ProductVersion}) nie musi mieć przypisanego żadnego odpowiedniego typu diety (\ref{kat:DietType})
        \item Wersja produktu (\ref{kat:ProductVersion}) może mieć przypisanych wiele odpowiednich typów diety (\ref{kat:DietType})
        \item Wersja produktu (\ref{kat:ProductVersion}) nie musi mieć przypisanego żadnego nieodpowiedniego typu diety (\ref{kat:DietType})
        \item Wersja produktu (\ref{kat:ProductVersion}) może mieć przypisanych wiele nieodpowiednich typów diety (\ref{kat:DietType})
        %CRUD
        \item todo
    \end{enumerate}
    \item\ref{kat:ProductBasicNutritionData}
    \begin{enumerate}[label={\textbf{REG/\protect\threedigits{\arabic{enumi}}}}, wide, labelwidth=!, align=left, leftmargin=3cm, resume]
        %Relacje
        \item Podstawowe wartości odżywcze produktu(\ref{kat:ProductBasicNutritionData}) muszą być przypisane do dokladnie jednej wersji produktu (\ref{kat:ProductVersion})
        %CRUD
        \item todo
    \end{enumerate}
    \item\ref{kat:NutritionData}
    \begin{enumerate}[label={\textbf{REG/\protect\threedigits{\arabic{enumi}}}}, wide, labelwidth=!, align=left, leftmargin=3cm, resume]
        %Relacje
        \item Wartość odżywcza (\ref{kat:NutritionData}) musi być przypisana do dokładnie jednej wersji produktu (\ref{kat:ProductVersion})
        \item Wartość odżywcza (\ref{kat:NutritionData}) musi być przypisana do dokładnie jednej definicji wartości odżywczej (\ref{kat:NutritionDefinition})
        %CRUD
        \item todo
    \end{enumerate}
    \item\ref{kat:NutritionDefinition}
    \begin{enumerate}[label={\textbf{REG/\protect\threedigits{\arabic{enumi}}}}, wide, labelwidth=!, align=left, leftmargin=3cm, resume]
        %Relacje
        \item Definicja wartości odżywczej (\ref{kat:NutritionDefinition}) nie musi mieć zdefiniowanego żadnego tłumaczenia (\ref{kat:NutritionDefinitionTranslation})
        \item Definicja wartości odżywczej (\ref{kat:NutritionDefinition}) może mieć zdefiniowanych wiele tłumaczeń (\ref{kat:NutritionDefinitionTranslation})
        %CRUD
        \item todo
    \end{enumerate}
    \item\ref{kat:NutritionDefinitionTranslation}
    \begin{enumerate}[label={\textbf{REG/\protect\threedigits{\arabic{enumi}}}}, wide, labelwidth=!, align=left, leftmargin=3cm, resume]
        %Relacje
        \item Tłumaczenie definicji wartości odżywczej (\ref{kat:NutritionDefinitionTranslation}) musi być przypisane do dokładnie jednej definicji wartości odżywczej  (\ref{kat:NutritionDefinition})
        %CRUD
        \item todo
    \end{enumerate}
    \item\ref{kat:HouseholdMeasure}
    \begin{enumerate}[label={\textbf{REG/\protect\threedigits{\arabic{enumi}}}}, wide, labelwidth=!, align=left, leftmargin=3cm, resume]
        %Relacje
        \item Miara domowa (\ref{kat:HouseholdMeasure}) musi być przypisana do dokładnie jednej wersji produktu (\ref{kat:ProductVersion})
        %CRUD
        \item todo
    \end{enumerate}
    \item\ref{kat:ProductSubcategory}
    \begin{enumerate}[label={\textbf{REG/\protect\threedigits{\arabic{enumi}}}}, wide, labelwidth=!, align=left, leftmargin=3cm, resume]
        %Relacje
        \item Podkategoria produktu musi być przypisana do conajmniej jednej wersji produktu
        \item Podkategoria produktu może być przypisana do wielu wersji produktu
        \item Podktagoria produktu musi być przypisana do dokładnie jednej kategorii
        %CRUD
        \item todo
    \end{enumerate}
    \item\ref{kat:ProductCategory}
    \begin{enumerate}[label={\textbf{REG/\protect\threedigits{\arabic{enumi}}}}, wide, labelwidth=!, align=left, leftmargin=3cm, resume]
        %Relacje
        \item Kategoria produktu nie musi mieć przypisanego żadnego tłumaczenia
        \item Kategoria produktu może mieć przypisanych wiele tłumaczeń
        %CRUD
        \item todo
    \end{enumerate}
    \item\ref{kat:ProductCategoryTranslation}
    \begin{enumerate}[label={\textbf{REG/\protect\threedigits{\arabic{enumi}}}}, wide, labelwidth=!, align=left, leftmargin=3cm, resume]
        %Relacje
        \item Tłumaczenie kategorii produktu musi być przypisane do dokładnie jednej kategorii
        %CRUD
        \item todo
    \end{enumerate}
    \item\ref{kat:DietType}
    \begin{enumerate}[label={\textbf{REG/\protect\threedigits{\arabic{enumi}}}}, wide, labelwidth=!, align=left, leftmargin=3cm, resume]
        %Relacje
        \item Typ diety nie musi mieć zdefiniowanego żadnego tłumaczenia
        \item Typ diety może mieć zdefiniowanych wiele tłumaczeń
        %CRUD
        \item todo
    \end{enumerate}
    \item\ref{kat:DietTypeTranslation}
    \begin{enumerate}[label={\textbf{REG/\protect\threedigits{\arabic{enumi}}}}, wide, labelwidth=!, align=left, leftmargin=3cm, resume]
        %Relacje
        \item Tłumaczenie typu diety musi być przypisane do dokładnie jednego typu diety
        %CRUD
        \item todo
    \end{enumerate}
    \item\ref{kat:Recipe}
    \begin{enumerate}[label={\textbf{REG/\protect\threedigits{\arabic{enumi}}}}, wide, labelwidth=!, align=left, leftmargin=3cm, resume]
        %Relacje
        \item Przepis nie musi mieć zdefiniowanego żadnego przepisu źródłowego
        \item Przepis może mieć zdefiniowany maksymalnie jeden przepis źródłowy
        \item Przepis musi mieć przynajmniej jedną wersję
        \item Przepis może mieć wiele wersji
        \item Przepis nie musi mieć zdefiniowanego autora
        \item Przepis może mieć maksymalnie jednego autora
        %CRUD
        \item todo
    \end{enumerate}
    \item\ref{kat:RecipeVersion}
    \begin{enumerate}[label={\textbf{REG/\protect\threedigits{\arabic{enumi}}}}, wide, labelwidth=!, align=left, leftmargin=3cm, resume]
        %Relacje
        \item Wersja przepisu musi mieć dokładnie jedne podstawowe wartości odżywcze przepisu
        \item Wersja przepisu musi mieć przynajmniej jedną sekcję
        \item Wersja przepisu może mieć wiele sekcji
        \item Wersja przepisu nie musi mieć przypisanego żadnego sprzętu kuchennego
        \item Wersja przepisu może mieć przypisanych wiele sprzętów kuchennych
        \item Wersja przepisu nie musi mieć przypisanego żadnego typu dania
        \item Wersja przepisu może mieć przypisanych wiele typów dań
        \item Wersja przepisu nie musi mieć przypisanego żadnego typu posiłku
        \item Wersja przepisu może mieć przypisanych wiele typów posiłków
        \item Wersja przepisu nie musi mieć przypisanego żadnego odpowiedniego typu diety
        \item Wersja przepisu może mieć przypisanych wiele odpowiednich typów diety
        \item Wersja przepisu nie musi mieć przypisanego żadnego nieodpowiedniego typu diety
        \item Wersja przepisu może mieć przypisanych wiele nieodpowiednich typów diety
        %CRUD
        \item todo
    \end{enumerate}
    \item\ref{kat:RecipeBasicNutritionData}
    \begin{enumerate}[label={\textbf{REG/\protect\threedigits{\arabic{enumi}}}}, wide, labelwidth=!, align=left, leftmargin=3cm, resume]
        %Relacje
        \item Podstawowe wartości odżywcze przepisu muszą być przypisane do dokładnie jednej wersji przepisu
        %CRUD
        \item todo
    \end{enumerate}
    \item\ref{kat:RecipeSection}
    \begin{enumerate}[label={\textbf{REG/\protect\threedigits{\arabic{enumi}}}}, wide, labelwidth=!, align=left, leftmargin=3cm, resume]
        %Relacje
        \item Sekcja przepisu musi być przypisana do dokładniej jednej wersji przepisu
        \item Sekcja przepisu musi mieć przypisaną przynajmniej jedną porcję produktu
        \item Sekcja przepisu może mieć przypisanych wiele porcji produktu
        \item Sekcja przepisu musi mieć przypisany przynajmniej jeden krok przygotowania
        \item Sekcja przepisu może mieć zdefiniowanych wiele kroków przygotowania
        %CRUD
        \item todo
    \end{enumerate}
    \item\ref{kat:ProductPortion}
    \begin{enumerate}[label={\textbf{REG/\protect\threedigits{\arabic{enumi}}}}, wide, labelwidth=!, align=left, leftmargin=3cm, resume]
        %Relacje
        \item Porcja produktu musi być przypisana do dokładnie jednej sekcji przepisu
        \item Porcja produktu musi mieć przypisany dokładnie jeden produkt
        \item Porcja produktu nie musi mieć przypisanej miary domowej
        \item Porcja produktu może mieć przypisaną maksymalnie jedną miarę produktu
        %CRUD
        \item todo
    \end{enumerate}
    \item\ref{kat:PreparationStep}
    \begin{enumerate}[label={\textbf{REG/\protect\threedigits{\arabic{enumi}}}}, wide, labelwidth=!, align=left, leftmargin=3cm, resume]
        %Relacje
        \item Krok przygotowania musi być przypisany do dokładnie jednej sekcji przepisu
        %CRUD
        \item todo
    \end{enumerate}
    \item\ref{kat:KitchenAppliance}
    \begin{enumerate}[label={\textbf{REG/\protect\threedigits{\arabic{enumi}}}}, wide, labelwidth=!, align=left, leftmargin=3cm, resume]
        %Relacje
        \item Sprzęt kuchenny nie musi mieć zdefiniowanego żadnego tlumaczenia
        \item Sprzęt kuchenny może mieć zdefiniowanych wiele tlumaczeń
        %CRUD
        \item todo
    \end{enumerate}
    \item\ref{kat:KitchenApplianceTranslation}
    \begin{enumerate}[label={\textbf{REG/\protect\threedigits{\arabic{enumi}}}}, wide, labelwidth=!, align=left, leftmargin=3cm, resume]
        %Relacje
        \item Tłumaczenie sprzetu kuchennego musi być przypisane do dokaldnie jednego sprzetu kuchennego
        %CRUD
        \item todo
    \end{enumerate}
    \item\ref{kat:DishType}
    \begin{enumerate}[label={\textbf{REG/\protect\threedigits{\arabic{enumi}}}}, wide, labelwidth=!, align=left, leftmargin=3cm, resume]
        %Relacje
        \item Typ dania nie musi mieć zdefiniowanego żadnego tłumaczenia
        \item Typ dania może mieć zdefiniowanych wiele tłumaczeń
        %CRUD
        \item todo
    \end{enumerate}
    \item\ref{kat:DishTypeTranslation}
    \begin{enumerate}[label={\textbf{REG/\protect\threedigits{\arabic{enumi}}}}, wide, labelwidth=!, align=left, leftmargin=3cm, resume]
        %Relacje
        \item Tłumaczenie typu dania musi być przypisane do dokładnie jednego typu dania
        %CRUD
        \item todo
    \end{enumerate}
    \item\ref{kat:MealType}
    \begin{enumerate}[label={\textbf{REG/\protect\threedigits{\arabic{enumi}}}}, wide, labelwidth=!, align=left, leftmargin=3cm, resume]
        %Relacje
        \item Typ posiłku nie musi mieć zdefiniowanego żadnego tłumaczenia
        \item Typ posiłku może mieć zdefiniowanych wiele tłumaczeń
        %CRUD
        \item todo
    \end{enumerate}
    \item\ref{kat:MealTypeTranslation}
    \begin{enumerate}[label={\textbf{REG/\protect\threedigits{\arabic{enumi}}}}, wide, labelwidth=!, align=left, leftmargin=3cm, resume]
        %Relacje
        \item Tłumaczenie typu posiłku musi być przypisane do dokładnie jednego typu posiłku
        %CRUD
        \item todo
    \end{enumerate}
    \item\ref{kat:MealPlan}
    \begin{enumerate}[label={\textbf{REG/\protect\threedigits{\arabic{enumi}}}}, wide, labelwidth=!, align=left, leftmargin=3cm, resume]
        %Relacje
        \item Jadłospis musi mieć przypisany przynajmniej jeden dzień
        \item Jadłospis może mieć przypisanych maksymalnie 31 dni
        \item Jadłospis musi mieć przypisaną przynajmniej jedną definicję posiłku
        \item Jadłospis może mieć przypisanych maksymalnie 10 definicji posiłków
        \item Jadłospis nie musi mieć przypisanego żadnego odpowiedniego typu diety
        \item Jadłospis może mieć przypisanych wiele odpowiednich typów diety
        \item Jadłospis nie musi mieć przypisanego żadnego nieodpowiedniego typu diety
        \item Jadłospis może mieć przypisanych wiele nieodpowiednich typów diety
        \item Jadłospis musi mieć dokładnie jednego autora
        %CRUD
        \item todo
    \end{enumerate}
    \item\ref{kat:MealPlanDay}
    \begin{enumerate}[label={\textbf{REG/\protect\threedigits{\arabic{enumi}}}}, wide, labelwidth=!, align=left, leftmargin=3cm, resume]
        %Relacje
        \item Dzień jadłospisu musi być przypisany do dokładnie jednego jadłospisu
        \item Dzień jadłospisu nie musi mieć przypisanego żadnego posiłku
        \item Dzień jadłospisu może mieć przypisanych maksymalnie 10 posiłków
        %CRUD
        \item todo
    \end{enumerate}
    \item\ref{kat:Meal}
    \begin{enumerate}[label={\textbf{REG/\protect\threedigits{\arabic{enumi}}}}, wide, labelwidth=!, align=left, leftmargin=3cm, resume]
        %Relacje
        \item Posiłek musi być przypisany do dokładnie jednego dnia jadłospisu
        \item Posiłek nie musi mieć przypisanego żadnego produktu
        \item Posiłek może mieć przypisanych wiele produktów
        \item Posiłek nie musi mieć przypisanego żadnego przepisu
        \item Posiłek może mieć przypisanych wiele przepisów
        %CRUD
        \item todo
    \end{enumerate}
    \item\ref{kat:MealRecipe}
    \begin{enumerate}[label={\textbf{REG/\protect\threedigits{\arabic{enumi}}}}, wide, labelwidth=!, align=left, leftmargin=3cm, resume]
        %Relacje
        \item Przepis posiłku musi być przypisany do dokladnie jednego posiłku
        \item Przepis posiłku musi mieć przypisany dokładnie jeden przepis
        %CRUD
        \item todo
    \end{enumerate}
    \item\ref{kat:MealProduct}
    \begin{enumerate}[label={\textbf{REG/\protect\threedigits{\arabic{enumi}}}}, wide, labelwidth=!, align=left, leftmargin=3cm, resume]
        %Relacje
        \item Produkt posiłku musi być przypisany do dokładnie jednego posiłku
        \item Produkt posiłku musi mieć przypisany dokładnie jeden produkt
        \item Produkt posiłku nie musi mieć przypisanej żadnej miary domowej
        \item Produkt posiłku musi mieć przypisaną maksymalnie jedną miarę domową
        %CRUD
        \item todo
    \end{enumerate}
    \item\ref{kat:MealDefinition}
    \begin{enumerate}[label={\textbf{REG/\protect\threedigits{\arabic{enumi}}}}, wide, labelwidth=!, align=left, leftmargin=3cm, resume]
        %Relacje
        \item Definicja posiłku musi być przypisana do dokładnie jednego jadłospisu
        \item Definicja posiłku musi mieć przypisany dokładnie jeden typo posiłku
        %CRUD
        \item todo
    \end{enumerate}
    \item\ref{kat:Appointment}
    \begin{enumerate}[label={\textbf{REG/\protect\threedigits{\arabic{enumi}}}}, wide, labelwidth=!, align=left, leftmargin=3cm, resume]
        %Relacje
        \item Wizyta musi być przypisana do dokładnie jednej karty pacjenta
        \item Wizyta nie musi mieć przypisanej żadnej ewaluacji
        \item Wizyta może mieć przypisaną maksymalnie jedną ewaluację
        \item Wizyta nie musi mieć przypisanego żadnych pomiarów ciała
        \item Wizyta może mieć przypisane maksymalnie jedne pomiary ciała
        \item Wizyta nie musi mieć przypisanego żadnego wywiadu żywieniowego
        \item Wizyta może mieć przypisany maksymalnie jeden wywiad żywieniowy
        \item Wizyta nie musi mieć przypisanego żadnego jadłospisu
        \item Wizyta może mieć przypisanych wiele jadłospisów
        %CRUD
        \item todo
    \end{enumerate}
    \item\ref{kat:PatientCard}
    \begin{enumerate}[label={\textbf{REG/\protect\threedigits{\arabic{enumi}}}}, wide, labelwidth=!, align=left, leftmargin=3cm, resume]
        %Relacje
        \item Karta pacjenta nie musi mieć przypisanej żadnej wizyty
        \item Karta pacjenta może mieć przypisanych wiele wizyt
        \item Karta pacjenta musi mieć przypisanego dokładnie jednego pacjenta
        \item Karta pacjenta musi mieć przypisanego dokładnie jednego dietetyka
        %CRUD
        \item todo
    \end{enumerate}
    \item\ref{kat:AppointmentEvaluation}
    \begin{enumerate}[label={\textbf{REG/\protect\threedigits{\arabic{enumi}}}}, wide, labelwidth=!, align=left, leftmargin=3cm, resume]
        %Relacje
        \item Ewaluacja wizyty musi być przypisana do dokładnie jednej wizyty
        %CRUD
        \item todo
    \end{enumerate}
    \item\ref{kat:BodyMeasurement}
    \begin{enumerate}[label={\textbf{REG/\protect\threedigits{\arabic{enumi}}}}, wide, labelwidth=!, align=left, leftmargin=3cm, resume]
        %Relacje
        \item Pomiary ciała muszą być przypisane do dokładnie jednej wizyty
        %CRUD
        \item todo
    \end{enumerate}
    \item\ref{kat:NutritionalInterview}
    \begin{enumerate}[label={\textbf{REG/\protect\threedigits{\arabic{enumi}}}}, wide, labelwidth=!, align=left, leftmargin=3cm, resume]
        %Relacje
        \item Wywiad żywieniowy musi być przypisany do dokładnie jednej wizyty
        \item Wywiad żywieniowy nie musi mieć przypisanego żadnego niestandardowego pytania
        \item Wywiad żywiniowy może mieć przypisanych wiele niestandardowych pytań
        \item Wywiad żywieniowy nie musi mieć przypisanych żadnych posiadanych sprzętów kuchennych
        \item Wywiad żywieniowy może mieć przypisanych wiele posiadanych sprzętów kuchennych
        %CRUD
        \item todo
    \end{enumerate}
    \item\ref{kat:CustomNutritionalInterviewQuestion}
    \begin{enumerate}[label={\textbf{REG/\protect\threedigits{\arabic{enumi}}}}, wide, labelwidth=!, align=left, leftmargin=3cm, resume]
        %Relacje
        \item Niestandardowe pytanie żywieniowe musi być przypisane do dokładnie jednego wywiadu żywieniowego
        %CRUD
        \item todo
    \end{enumerate}
    \item\ref{kat:CustomNutritionalInterviewQuestionTemplate}
    \begin{enumerate}[label={\textbf{REG/\protect\threedigits{\arabic{enumi}}}}, wide, labelwidth=!, align=left, leftmargin=3cm, resume]
        %Relacje
        \item Szablon niestandardowego pytania żywieniowego musi mieć dokaldnie jednego autora
        %CRUD
        \item todo
    \end{enumerate}
    \item\ref{kat:AssignedMealPlan}
    \begin{enumerate}[label={\textbf{REG/\protect\threedigits{\arabic{enumi}}}}, wide, labelwidth=!, align=left, leftmargin=3cm, resume]
        %Relacje
        \item Przypisany jadłospis musi mieć przydzieloną dokładnie jedną wizytę
        \item Przypisany jadłospis musi mieć przydzielony dokładnie jeden jadłospis
        %CRUD
        \item todo
    \end{enumerate}
\end{itemize}

\section{Ograniczenia dziedzinowe}
\todo{uzupełnić ograniczenia dziedzionowe}

\begin{itemize}[label={\textbf{Ograniczenia dla}}, wide, labelwidth=!, labelindent=0pt]
    \setlength\itemsep{1em}
    \item\ref{kat:User}
    \begin{enumerate}[label={\textbf{OGR/\protect\threedigits{\arabic{enumi}}}}, wide, labelwidth=!, align=left, leftmargin=3cm]
        %Relacje
        \item test1
        %CRUD
        \item todo
    \end{enumerate}
    \item\ref{kat:UserExtraInfo}
    \begin{enumerate}[label={\textbf{OGR/\protect\threedigits{\arabic{enumi}}}}, wide, labelwidth=!, align=left, leftmargin=3cm, resume]
        %Relacje
        \item test1
        %CRUD
        \item todo
    \end{enumerate}
    \item\ref{kat:SiteContent}
    \begin{enumerate}[label={\textbf{OGR/\protect\threedigits{\arabic{enumi}}}}, wide, labelwidth=!, align=left, leftmargin=3cm, resume]
        %Relacje
        \item test1
        %CRUD
        \item todo
    \end{enumerate}
    \item\ref{kat:SiteContentTranslation}
    \begin{enumerate}[label={\textbf{OGR/\protect\threedigits{\arabic{enumi}}}}, wide, labelwidth=!, align=left, leftmargin=3cm, resume]
        %Relacje
        \item test1
        %CRUD
        \item todo
    \end{enumerate}
    \item\ref{kat:ContactInfo}
    \begin{enumerate}[label={\textbf{OGR/\protect\threedigits{\arabic{enumi}}}}, wide, labelwidth=!, align=left, leftmargin=3cm, resume]
        %Relacje
        \item test1
        %CRUD
        \item todo
    \end{enumerate}
    \item\ref{kat:Pricing}
    \begin{enumerate}[label={\textbf{OGR/\protect\threedigits{\arabic{enumi}}}}, wide, labelwidth=!, align=left, leftmargin=3cm, resume]
        %Relacje
        \item test1
        %CRUD
        \item todo
    \end{enumerate}
    \item\ref{kat:PricingTranslation}
    \begin{enumerate}[label={\textbf{OGR/\protect\threedigits{\arabic{enumi}}}}, wide, labelwidth=!, align=left, leftmargin=3cm, resume]
        %Relacje
        \item test1
        %CRUD
        \item todo
    \end{enumerate}
    \item\ref{kat:Product}
    \begin{enumerate}[label={\textbf{OGR/\protect\threedigits{\arabic{enumi}}}}, wide, labelwidth=!, align=left, leftmargin=3cm, resume]
        %Relacje
        \item test1
        %CRUD
        \item todo
    \end{enumerate}
    \item\ref{kat:ProductVersion}
    \begin{enumerate}[label={\textbf{OGR/\protect\threedigits{\arabic{enumi}}}}, wide, labelwidth=!, align=left, leftmargin=3cm, resume]
        %Relacje
        \item test1
        %CRUD
        \item todo
    \end{enumerate}
    \item\ref{kat:ProductBasicNutritionData}
    \begin{enumerate}[label={\textbf{OGR/\protect\threedigits{\arabic{enumi}}}}, wide, labelwidth=!, align=left, leftmargin=3cm, resume]
        %Relacje
        \item test1
        %CRUD
        \item todo
    \end{enumerate}
    \item\ref{kat:NutritionData}
    \begin{enumerate}[label={\textbf{OGR/\protect\threedigits{\arabic{enumi}}}}, wide, labelwidth=!, align=left, leftmargin=3cm, resume]
        %Relacje
        \item test1
        %CRUD
        \item todo
    \end{enumerate}
    \item\ref{kat:NutritionDefinition}
    \begin{enumerate}[label={\textbf{OGR/\protect\threedigits{\arabic{enumi}}}}, wide, labelwidth=!, align=left, leftmargin=3cm, resume]
        %Relacje
        \item test1
        %CRUD
        \item todo
    \end{enumerate}
    \item\ref{kat:NutritionDefinitionTranslation}
    \begin{enumerate}[label={\textbf{OGR/\protect\threedigits{\arabic{enumi}}}}, wide, labelwidth=!, align=left, leftmargin=3cm, resume]
        %Relacje
        \item test1
        %CRUD
        \item todo
    \end{enumerate}
    \item\ref{kat:HouseholdMeasure}
    \begin{enumerate}[label={\textbf{OGR/\protect\threedigits{\arabic{enumi}}}}, wide, labelwidth=!, align=left, leftmargin=3cm, resume]
        %Relacje
        \item test1
        %CRUD
        \item todo
    \end{enumerate}
    \item\ref{kat:ProductSubcategory}
    \begin{enumerate}[label={\textbf{OGR/\protect\threedigits{\arabic{enumi}}}}, wide, labelwidth=!, align=left, leftmargin=3cm, resume]
        %Relacje
        \item test1
        %CRUD
        \item todo
    \end{enumerate}
    \item\ref{kat:ProductCategory}
    \begin{enumerate}[label={\textbf{OGR/\protect\threedigits{\arabic{enumi}}}}, wide, labelwidth=!, align=left, leftmargin=3cm, resume]
        %Relacje
        \item test1
        %CRUD
        \item todo
    \end{enumerate}
    \item\ref{kat:ProductCategoryTranslation}
    \begin{enumerate}[label={\textbf{OGR/\protect\threedigits{\arabic{enumi}}}}, wide, labelwidth=!, align=left, leftmargin=3cm, resume]
        %Relacje
        \item test1
        %CRUD
        \item todo
    \end{enumerate}
    \item\ref{kat:DietType}
    \begin{enumerate}[label={\textbf{OGR/\protect\threedigits{\arabic{enumi}}}}, wide, labelwidth=!, align=left, leftmargin=3cm, resume]
        %Relacje
        \item test1
        %CRUD
        \item todo
    \end{enumerate}
    \item\ref{kat:DietTypeTranslation}
    \begin{enumerate}[label={\textbf{OGR/\protect\threedigits{\arabic{enumi}}}}, wide, labelwidth=!, align=left, leftmargin=3cm, resume]
        %Relacje
        \item test1
        %CRUD
        \item todo
    \end{enumerate}
    \item\ref{kat:Recipe}
    \begin{enumerate}[label={\textbf{OGR/\protect\threedigits{\arabic{enumi}}}}, wide, labelwidth=!, align=left, leftmargin=3cm, resume]
        %Relacje
        \item test1
        %CRUD
        \item todo
    \end{enumerate}
    \item\ref{kat:RecipeVersion}
    \begin{enumerate}[label={\textbf{OGR/\protect\threedigits{\arabic{enumi}}}}, wide, labelwidth=!, align=left, leftmargin=3cm, resume]
        %Relacje
        \item test1
        %CRUD
        \item todo
    \end{enumerate}
    \item\ref{kat:RecipeBasicNutritionData}
    \begin{enumerate}[label={\textbf{OGR/\protect\threedigits{\arabic{enumi}}}}, wide, labelwidth=!, align=left, leftmargin=3cm, resume]
        %Relacje
        \item test1
        %CRUD
        \item todo
    \end{enumerate}
    \item\ref{kat:RecipeSection}
    \begin{enumerate}[label={\textbf{OGR/\protect\threedigits{\arabic{enumi}}}}, wide, labelwidth=!, align=left, leftmargin=3cm, resume]
        %Relacje
        \item test1
        %CRUD
        \item todo
    \end{enumerate}
    \item\ref{kat:ProductPortion}
    \begin{enumerate}[label={\textbf{OGR/\protect\threedigits{\arabic{enumi}}}}, wide, labelwidth=!, align=left, leftmargin=3cm, resume]
        %Relacje
        \item test1
        %CRUD
        \item todo
    \end{enumerate}
    \item\ref{kat:PreparationStep}
    \begin{enumerate}[label={\textbf{OGR/\protect\threedigits{\arabic{enumi}}}}, wide, labelwidth=!, align=left, leftmargin=3cm, resume]
        %Relacje
        \item test1
        %CRUD
        \item todo
    \end{enumerate}
    \item\ref{kat:KitchenAppliance}
    \begin{enumerate}[label={\textbf{OGR/\protect\threedigits{\arabic{enumi}}}}, wide, labelwidth=!, align=left, leftmargin=3cm, resume]
        %Relacje
        \item test1
        %CRUD
        \item todo
    \end{enumerate}
    \item\ref{kat:KitchenApplianceTranslation}
    \begin{enumerate}[label={\textbf{OGR/\protect\threedigits{\arabic{enumi}}}}, wide, labelwidth=!, align=left, leftmargin=3cm, resume]
        %Relacje
        \item test1
        %CRUD
        \item todo
    \end{enumerate}
    \item\ref{kat:DishType}
    \begin{enumerate}[label={\textbf{OGR/\protect\threedigits{\arabic{enumi}}}}, wide, labelwidth=!, align=left, leftmargin=3cm, resume]
        %Relacje
        \item test1
        %CRUD
        \item todo
    \end{enumerate}
    \item\ref{kat:DishTypeTranslation}
    \begin{enumerate}[label={\textbf{OGR/\protect\threedigits{\arabic{enumi}}}}, wide, labelwidth=!, align=left, leftmargin=3cm, resume]
        %Relacje
        \item test1
        %CRUD
        \item todo
    \end{enumerate}
    \item\ref{kat:MealType}
    \begin{enumerate}[label={\textbf{OGR/\protect\threedigits{\arabic{enumi}}}}, wide, labelwidth=!, align=left, leftmargin=3cm, resume]
        %Relacje
        \item test1
        %CRUD
        \item todo
    \end{enumerate}
    \item\ref{kat:MealTypeTranslation}
    \begin{enumerate}[label={\textbf{OGR/\protect\threedigits{\arabic{enumi}}}}, wide, labelwidth=!, align=left, leftmargin=3cm, resume]
        %Relacje
        \item test1
        %CRUD
        \item todo
    \end{enumerate}
    \item\ref{kat:MealPlan}
    \begin{enumerate}[label={\textbf{OGR/\protect\threedigits{\arabic{enumi}}}}, wide, labelwidth=!, align=left, leftmargin=3cm, resume]
        %Relacje
        \item test1
        %CRUD
        \item todo
    \end{enumerate}
    \item\ref{kat:MealPlanDay}
    \begin{enumerate}[label={\textbf{OGR/\protect\threedigits{\arabic{enumi}}}}, wide, labelwidth=!, align=left, leftmargin=3cm, resume]
        %Relacje
        \item test1
        %CRUD
        \item todo
    \end{enumerate}
    \item\ref{kat:Meal}
    \begin{enumerate}[label={\textbf{OGR/\protect\threedigits{\arabic{enumi}}}}, wide, labelwidth=!, align=left, leftmargin=3cm, resume]
        %Relacje
        \item test1
        %CRUD
        \item todo
    \end{enumerate}
    \item\ref{kat:MealRecipe}
    \begin{enumerate}[label={\textbf{OGR/\protect\threedigits{\arabic{enumi}}}}, wide, labelwidth=!, align=left, leftmargin=3cm, resume]
        %Relacje
        \item test1
        %CRUD
        \item todo
    \end{enumerate}
    \item\ref{kat:MealProduct}
    \begin{enumerate}[label={\textbf{OGR/\protect\threedigits{\arabic{enumi}}}}, wide, labelwidth=!, align=left, leftmargin=3cm, resume]
        %Relacje
        \item test1
        %CRUD
        \item todo
    \end{enumerate}
    \item\ref{kat:MealDefinition}
    \begin{enumerate}[label={\textbf{OGR/\protect\threedigits{\arabic{enumi}}}}, wide, labelwidth=!, align=left, leftmargin=3cm, resume]
        %Relacje
        \item test1
        %CRUD
        \item todo
    \end{enumerate}
    \item\ref{kat:Appointment}
    \begin{enumerate}[label={\textbf{OGR/\protect\threedigits{\arabic{enumi}}}}, wide, labelwidth=!, align=left, leftmargin=3cm, resume]
        %Relacje
        \item test1
        %CRUD
        \item todo
    \end{enumerate}
    \item\ref{kat:PatientCard}
    \begin{enumerate}[label={\textbf{OGR/\protect\threedigits{\arabic{enumi}}}}, wide, labelwidth=!, align=left, leftmargin=3cm, resume]
        %Relacje
        \item test1
        %CRUD
        \item todo
    \end{enumerate}
    \item\ref{kat:AppointmentEvaluation}
    \begin{enumerate}[label={\textbf{OGR/\protect\threedigits{\arabic{enumi}}}}, wide, labelwidth=!, align=left, leftmargin=3cm, resume]
        %Relacje
        \item test1
        %CRUD
        \item todo
    \end{enumerate}
    \item\ref{kat:BodyMeasurement}
    \begin{enumerate}[label={\textbf{OGR/\protect\threedigits{\arabic{enumi}}}}, wide, labelwidth=!, align=left, leftmargin=3cm, resume]
        %Relacje
        \item test1
        %CRUD
        \item todo
    \end{enumerate}
    \item\ref{kat:NutritionalInterview}
    \begin{enumerate}[label={\textbf{OGR/\protect\threedigits{\arabic{enumi}}}}, wide, labelwidth=!, align=left, leftmargin=3cm, resume]
        %Relacje
        \item test1
        %CRUD
        \item todo
    \end{enumerate}
    \item\ref{kat:CustomNutritionalInterviewQuestion}
    \begin{enumerate}[label={\textbf{OGR/\protect\threedigits{\arabic{enumi}}}}, wide, labelwidth=!, align=left, leftmargin=3cm, resume]
        %Relacje
        \item test1
        %CRUD
        \item todo
    \end{enumerate}
    \item\ref{kat:CustomNutritionalInterviewQuestionTemplate}
    \begin{enumerate}[label={\textbf{OGR/\protect\threedigits{\arabic{enumi}}}}, wide, labelwidth=!, align=left, leftmargin=3cm, resume]
        %Relacje
        \item test1
        %CRUD
        \item todo
    \end{enumerate}
    \item\ref{kat:AssignedMealPlan}
    \begin{enumerate}[label={\textbf{OGR/\protect\threedigits{\arabic{enumi}}}}, wide, labelwidth=!, align=left, leftmargin=3cm, resume]
        %Relacje
        \item test1
        %CRUD
        \item todo
    \end{enumerate}
\end{itemize}

\section{Model domenowy}
\todo{diagram klas}

\begin{minipage}{\textwidth}
    \begin{figure}[H]
        \centering\includegraphics[scale=0.7]{../uml/class_diagrams/dataTypes.png}
        \caption{Typy danych - diagram klas (opr.wł).}\label{rysunek:class-diagram-data-types}
    \end{figure}
\end{minipage}

\begin{minipage}{\textwidth}
    \begin{figure}[H]
        \centering\includegraphics[scale=0.7]{../uml/class_diagrams/gateway.png}
        \caption{Gateway - diagram klas (opr.wł).}\label{rysunek:class-diagram-gateway}
    \end{figure}
\end{minipage}

\begin{minipage}{\textwidth}
    \begin{figure}[H]
        \centering\includegraphics[scale=0.7]{../uml/class_diagrams/products.png}
        \caption{Produkty - diagram klas (opr.wł).}\label{rysunek:class-diagram-products}
    \end{figure}
\end{minipage}

\begin{minipage}{\textwidth}
    \begin{figure}[H]
        \centering\includegraphics[scale=0.7]{../uml/class_diagrams/recipes.png}
        \caption{Przepisy - diagram klas (opr.wł).}\label{rysunek:class-diagram-recipes}
    \end{figure}
\end{minipage}

\begin{minipage}{\textwidth}
    \begin{figure}[H]
        \centering\includegraphics[scale=0.7]{../uml/class_diagrams/mealplans.png}
        \caption{Jadłospisy - diagram klas (opr.wł).}\label{rysunek:class-diagram-mealplans}
    \end{figure}
\end{minipage}

\begin{minipage}{\textwidth}
    \begin{figure}[H]
        \centering\includegraphics[scale=0.7]{../uml/class_diagrams/appointments.png}
        \caption{Wizyty - diagram klas (opr.wł).}\label{rysunek:class-diagram-appointments}
    \end{figure}
\end{minipage}



\thispagestyle{normal}

\section{Opis podstawowej architektury systemu}
\todo{Opisać, że to aplikacja webowa w architekturze mikroserwisów
Wyszczególnienie modułów;
Diagram rozmieszczenia, wzorce projektowe}

%https://martinfowler.com/eaaDev/TemporalObject.html
\thispagestyle{normal}

    \chapter{Projekt}

\section{Kategorie}

\begin{enumerate}[label={\textbf{KAT/\protect\threedigits{\theenumi}}}, wide, labelwidth=!, labelindent=0pt]
    \item \label{Product_Category} Product
    \item Language
    \item test22
    \item test3
\end{enumerate}
\section {Reguły funkcjonowania}

\begin{enumerate}[label={\textbf{REG/\protect\threedigits{\theenumi}}}, wide, labelwidth=!, labelindent=0pt]
    \subsection{Produkty}
    \item test1
    \item test2
    \subsection{Przepisy}

    \subsubsection{\ref{Product_Category} Product}
    \subsection{Jadłospisy}
    \item test22
    \item test3
\end{enumerate}

\section{Ograniczenia dziedzinowe}

\section{Model domenowy}


\begin{minipage}{\textwidth}
    \begin{figure}[H]
        \centering\includegraphics[width=0.9\textwidth]{img/class-diagrams/produkty.png}
        \caption{Produkty (opr.wł).}\label{rysunek:produkty}
    \end{figure}
\end{minipage}

\begin{minipage}{\textwidth}
    \begin{figure}[H]
        \centering\includegraphics[width=0.9\textwidth]{img/class-diagrams/przepisy.png}
        \caption{Przepisy (opr.wł).}\label{rysunek:przepisy}
    \end{figure}
\end{minipage}

\begin{minipage}{\textwidth}
    \begin{figure}[H]
        \centering\includegraphics[width=0.9\textwidth]{img/class-diagrams/jadospisy.png}
        \caption{Jadłospisy (opr.wł).}\label{rysunek:jadlospisy}
    \end{figure}
\end{minipage}

\begin{minipage}{\textwidth}
    \begin{figure}[H]
        \centering\includegraphics[width=0.9\textwidth]{img/class-diagrams/wizyty.png}
        \caption{Wizyty (opr.wł).}\label{rysunek:wizyty}
    \end{figure}
\end{minipage}

\section{Przypadki użycia}
\section{Prototyp interfejsu}
\begin{minipage}{\textwidth}
    \begin{figure}[H]
        \centering\includegraphics[width=0.9\textwidth]{img/mockups/mockup1.png}
        \caption{Mockup1 (opr.wł).}\label{rysunek:mockup1}
    \end{figure}
\end{minipage}

\begin{minipage}{\textwidth}
    \begin{figure}[H]
        \centering\includegraphics[width=0.9\textwidth]{img/mockups/mockup2.png}
        \caption{Mockup2 (opr.wł).}\label{rysunek:mockup2}
    \end{figure}
\end{minipage}

\begin{minipage}{\textwidth}
    \begin{figure}[H]
        \centering\includegraphics[width=0.9\textwidth]{img/mockups/mockup3.png}
        \caption{Mockup3 (opr.wł).}\label{rysunek:mockup3}
    \end{figure}
\end{minipage}

\begin{minipage}{\textwidth}
    \begin{figure}[H]
        \centering\includegraphics[width=0.9\textwidth]{img/mockups/mockup4.png}
        \caption{Mockup4 (opr.wł).}\label{rysunek:mockup4}
    \end{figure}
\end{minipage}

\begin{minipage}{\textwidth}
    \begin{figure}[H]
        \centering\includegraphics[width=0.9\textwidth]{img/mockups/mockup5.png}
        \caption{Mockup5 (opr.wł).}\label{rysunek:mockup5}
    \end{figure}
\end{minipage}

\begin{minipage}{\textwidth}
    \begin{figure}[H]
        \centering\includegraphics[width=0.9\textwidth]{img/mockups/mockup6.png}
        \caption{Mockup6 (opr.wł).}\label{rysunek:mockup6}
    \end{figure}
\end{minipage}

\begin{minipage}{\textwidth}
    \begin{figure}[H]
        \centering\includegraphics[width=0.9\textwidth]{img/mockups/mockup7.png}
        \caption{Mockup7 (opr.wł).}\label{rysunek:mockup7}
    \end{figure}
\end{minipage}

\begin{minipage}{\textwidth}
    \begin{figure}[H]
        \centering\includegraphics[width=0.9\textwidth]{img/mockups/mockup8.png}
        \caption{Mockup8 (opr.wł).}\label{rysunek:mockup8}
    \end{figure}
\end{minipage}

\begin{minipage}{\textwidth}
    \begin{figure}[H]
        \centering\includegraphics[width=0.9\textwidth]{img/mockups/mockup9.png}
        \caption{Mockup9 (opr.wł).}\label{rysunek:mockup9}
    \end{figure}
\end{minipage}

\begin{minipage}{\textwidth}
    \begin{figure}[H]
        \centering\includegraphics[width=0.9\textwidth]{img/mockups/mockup10.png}
        \caption{Mockup10 (opr.wł).}\label{rysunek:mockup10}
    \end{figure}
\end{minipage}

\begin{minipage}{\textwidth}
    \begin{figure}[H]
        \centering\includegraphics[width=0.9\textwidth]{img/mockups/mockup11.png}
        \caption{Mockup11 (opr.wł).}\label{rysunek:mockup11}
    \end{figure}
\end{minipage}

\begin{minipage}{\textwidth}
    \begin{figure}[H]
        \centering\includegraphics[width=0.9\textwidth]{img/mockups/mockup12.png}
        \caption{Mockup12 (opr.wł).}\label{rysunek:mockup12}
    \end{figure}
\end{minipage}

\begin{minipage}{\textwidth}
    \begin{figure}[H]
        \centering\includegraphics[width=0.9\textwidth]{img/mockups/mockup13.png}
        \caption{Mockup13 (opr.wł).}\label{rysunek:mockup13}
    \end{figure}
\end{minipage}

\begin{minipage}{\textwidth}
    \begin{figure}[H]
        \centering\includegraphics[width=0.9\textwidth]{img/mockups/mockup14.png}
        \caption{Mockup14 (opr.wł).}\label{rysunek:mockup14}
    \end{figure}
\end{minipage}

\begin{minipage}{\textwidth}
    \begin{figure}[H]
        \centering\includegraphics[width=0.9\textwidth]{img/mockups/mockup15.png}
        \caption{Mockup15 (opr.wł).}\label{rysunek:mockup15}
    \end{figure}
\end{minipage}

\begin{minipage}{\textwidth}
    \begin{figure}[H]
        \centering\includegraphics[width=0.9\textwidth]{img/mockups/mockup16.png}
        \caption{Mockup16 (opr.wł).}\label{rysunek:mockup16}
    \end{figure}
\end{minipage}

\section{Architektura systemu}
\subsection{Model bazy danych}

\thispagestyle{normal}

    \chapter{Implementacja}
\section{Wykorzystywane środowiska i narzędzia programistyczne}
\section{Instalacja oprogramowania}
\subsection{Wymagania wstępne}
\begin{itemize}
    \item Java 8
    \item NPM
    \item Docker + Docker Compose
\end{itemize}
\section{Instrukcja użytkowania}

\thispagestyle{normal}

%    \chapter{Testy}
\section{Testy jednostkowe}
\section{Testy integracyjne}
\section{Testy akceptacyjne}

\thispagestyle{normal}

    \input{instrukcja}
    \chapter{Ala ma kota}

ĄĆĘŁŃÓŚŹŻ ąćęłńóśźż\footnote{Przykład użycia polskich znaków diakrytycznych oraz przypisu w miejscu}. \lipsum[1]

\section{Odniesienie do pozycji z literatury (strona WWW)}

% Odniesienie do rysunku i cytowanie dokumentu. Dokumenty są definiowane w pliku literatura.bib
Reszta dokumentacji znajduje się w \cite{docker_compose_reference}. \lipsum[3]

\section{Odniesienie do książki}

Jak pisze Harel w \cite{harel_rzecz_2008}: \lipsum[7]

\section{Rysunek}

% Rysunek
\begin{figure}
    \centering\includegraphics[width=.6\textwidth]{img/swarm-network}
    \caption{Docker ma sieć \cite{docker_compose_reference}.}  \label{rys:network}
    % Źródło rysunku i etykieta przez którą odwołujemy się do rysunku.
\end{figure}

Jak widać na rys. \ref{rys:network} Docker ma wewnętrzną sieć. \lipsum[1]


\subsection{Rysunek z kotem}

Jak widać na rys.\ref{rysunek:kot} Ala ma kota. \lipsum[9-10]

\begin{figure}[h!]
    \centering\includegraphics[width=.4\textwidth]{img/kotek}
    \caption{Ala ma kota (opr.wł).}\label{rysunek:kot}
\end{figure}

\subsection{Tabela}

Co uwzględniono w tabeli \ref{tabela:coktoma}. \lipsum[13-15]

% Tabela. Nazwa tabeli u góry.
\begin{table}[h!]
    \centering\caption{Co kto ma \cite{harel_rzecz_2008} (patrz też dodatek~\ref{Dod1}) \label{tabela:coktoma}}
    \begin{tabular}{|l|l|l|}% wyrównanie kolumn tabeli -> l c r - do lewej, środka, do prawej
        \hline
        Ala & ma & kota \\
        \hline
        Ola & ma & psa \\
        \hline
        Ula & ma & małpę\\
        \hline
    \end{tabular}
\end{table}

\lipsum[19-20] Warto wspomnieć, że w \cite{aizawa_groundwater_2009} rzecz przedstawiona jest zupełnie inaczej. Poniższy wzór:

\begin{equation}
    \sum_{i=1}^{\infty}a_i
    \label{eq:mojWzor}
\end{equation}

Wzór \ref{eq:mojWzor} wskazuje że dowód podany w \cite{kaleta_experimental_2005} może zostać podważony. \lipsum[9]

\section{Kod źródłowy}

% lub {java} albo {bash} albo {text}
\begin{listing}[h!]
    \begin{minted}{c}
        int main()
        {
        int a=2*3;
        printf("**Ala ma kota\n**");
        while(!I2C_CheckEvent(I2C1, I2C_EVENT_MASTER_MODE_SELECT)); /* EV5 */
        return 0;
        }
    \end{minted}
    \caption{Przykładowy algorytm w języku C (opr. wł.)} \label{listing:moj}
\end{listing}

W moim kodzie \ref{listing:moj} zrobiłem coś wspaniałego. \lipsum[4]

\begin{table}[h]
    \begin{tabularx}{\textwidth}{|>{\setlength\hsize{1.4\hsize}\setlength\linewidth{\hsize}}X|>{\setlength\hsize{.9\hsize}\setlength\linewidth{\hsize}}X|>{\setlength\hsize{.7\hsize}\setlength\linewidth{\hsize}}X|}
        \hline
        \multicolumn{3}{|c|}{Classification of the criticel point $(0,0)$ of $x'=Ax,|\mathbf{A}|\not=0$.}\\
        \hline
        Types & Type of Critical Point & Stability \\
        \hline
        1. Real unequal eigenvalues of same sign
        \begin{itemize}
            \item $\lambda_1 > \lambda_2 > 0$
            \item $\lambda_1 < \lambda_2 < 0$
        \end{itemize} &
        \vphantom{1. Real unequal eigenvalues of same sign}
        \begin{itemize}
            \item Improper Node/Node
            \item Improper Node/Node
        \end{itemize} &
        \vphantom{1. Real unequal eigenvalues of same sign}
        \begin{itemize}
            \item Unstable
            \item Asym. Stable
        \end{itemize}\\
        \hline
        2. Real unequal eigenvalues of opposite sign
        \begin{itemize}
            \item $\lambda_2 < 0 >\lambda_1$
        \end{itemize} &
        \vphantom{2. Real unequal eigenvalues of opposite sign}
        \begin{itemize}
            \item Saddle Point
        \end{itemize} &
        \vphantom{2. Real unequal eigenvalues of opposite sign}
        \begin{itemize}
            \item Unstable
        \end{itemize}\\
        \hline
        3. Equal eigenvalues \newline Subtype 1: Two Independent vectors
        \begin{itemize}
            \item $\lambda_1 = \lambda_2 > 0$
            \item $\lambda_1 = \lambda_2 < 0$
        \end{itemize} &
        \vphantom{3. Equal eigenvalues} \vphantom{ Subtype 1: Two Independent vectors}
        \begin{itemize}
            \item Proper Node
            \item Proper Node
        \end{itemize} &
        \vphantom{3. Equal eigenvalues} \vphantom{ Subtype 1: Two Independent vectors}
        \begin{itemize}
            \item Unstable
            \item Asym. Stable
        \end{itemize}\\
        \hline
    \end{tabularx}
\end{table}
\thispagestyle{normal}


    \input{ch7_zakonczenie}

    % W pracy pojawią się tylko prace naprawdę cytowane.
    % \nocite{*}

    \bibliography{literatura}
    \bibliographystyle{dyplom}

    \newpage
    \listoffigures

    \newpage
    \listoftables

    \newpage
    \listof{listing}{Spis kodów źródłowych}

    % \appendixpage
% \addappheadtotoc

\appendix
\begin{appendices}
    \chapter{To powinien być dodatek}\label{Dod1}
    \todo{\lipsum[9-11]}
\end{appendices}
\thispagestyle{normal}


\end{document}
