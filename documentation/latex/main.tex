\documentclass{dyplom}
\usepackage[utf8]{inputenc}
\usepackage{hyperref}
%%
\usepackage{lipsum}
\usepackage{easy-todo}

% Dane o pracy
\author{Krzysztof Marczyński}
\title{Projekt~i~implementacja systemu~do~zarządzania dietą w~oparciu~o~architekturę~mikroserwisów}
\titlen{Design~and~implementation of~diet~management~system based~on~microservice~architecture}
\promotor{dr inż. Michał Szczepanik}
%\konsultant{dr hab. inż. Kazimerz Kabacki}
\wydzial{Wydział Informatyki i Zarządzania}
\kierunek{Informatyka}
\krotkiestreszczenie{W pracy przedstawiono projekt aplikacji służącej do ukladania diet.}
\slowakluczowe{dieta, jadłospisy, aplikacja webowa, mikroserwisy}

%%%%%%%%%%%%%%%%%%%%%%%%%%%%%%%%%%%%%%%%%%%%%%%%%%%%%%%%%%%%
\begin{document}
    \sloppy %for words not to leak from right side of container

    \maketitle
    \pagenumbering{gobble}
    % !TeX spellcheck = pl_PL
% --- Strona ze streszczeniem i~abstraktem ------------------------------------------------------------------
\addtocontents{toc}{\protect\setcounter{tocdepth}{-1}}
\chapter*{Streszczenie} % po polsku
Celem pracy było opracowanie systemu do zarządzania dietą w~architekturze mikroserwisów.
Aby osiągnąć ten cel przeprowadzono analizę istniejących rozwiązań konkurencyjnych,
przedstawiono niezbędną wiedzę domenową oraz porównano popularne style architektury aplikacji.
Na podstawie zgromadzonej wiedzy wyszczególniono niezbędne założenia projektowe, zaprojektowano interfejs
oraz zdefiniowano kategorie danych wraz z~regułami i~ograniczeniami ich dotyczącymi.
Następnie przedstawiono opis implementacji powstałej na podstawie opracowanego projektu.
W implementacji kluczową rolę odegrały języki Java i~TypeScript, platforma deweloperska JHipster
oraz stos technologii Netflix OSS dla architektury mikroserwisów.
Opracowane rozwiązanie może zostać wykorzystane przez dietetyków w~celu przeprowadzania kompleksowej obsługi wizyty pacjenta
z~położeniem szczególnego nacisku na układanie jadłospisów i~udostępnianie go pacjentom.


% Kilka sztuczek, żeby:
%~- Abstract pojawił się na tej samej stronie co Streszczenie
%~- Abstract nie pojawił się w~spisie treści
\addtocontents{toc}{\protect\setcounter{tocdepth}{-1}}
\begingroup
\renewcommand{\cleardoublepage}{}
\renewcommand{\clearpage}{}
\chapter*{Abstract} % ...i to samo po angielsku
The aim of this work was to develop a~diet management system based on microservice architecture.
To achieve that goal, an analysis of existing competitive solutions was performed, the necessary domain knowledge was presented,
and popular application architecture styles were compared.
Based on the accumulated knowledge, the necessary design assumptions were specified, the interface was designed,
and categories of data were defined along with the rules and restrictions concerning them.
Then a~description of the implementation based on the developed project was presented.
The key role in the implementation was played by languages Java and TypeScript,
the JHipster development platform and Netflix OSS technology stack for a~microservices architecture.
The created solution can be used by dietitians in order to conduct comprehensive service of the patient's visit
with particular emphasis on designing the meal plans and sharing them with patients.
\endgroup
\addtocontents{toc}{\protect\setcounter{tocdepth}{2}}
% --- Koniec strony ze streszczeniem i~abstraktem -----------------------------------------------------------

    \cleardoublepage

    \tableofcontents
    \cleardoublepage

    \pagenumbering{arabic}
%    \todo{uzupełnić streszczenie i słowa kluczowe na stronie tytułowej}
    % !TeX spellcheck = pl_PL
\chapter*{Wstęp}\label{ch:admission}

\section*{Opis problemu}\label{sec:problem-description}

Przed przejściem do analizy problemu, warto zdefiniować co można rozumieć pod pojęciem diety.
Według definicji z~Encyklopedii Powszechnej PWN, dieta to "system odżywiania z~ustaleniem jakości i~ilości pokarmów,
dostosowany do potrzeb organizmu"\cite{book:encyklopedia-dieta}.
Opierając się na tej definicji, można opisać system do zarządzania dietą jako system,
który będzie ułatwiał dietetykowi tworzenie jadłospisów dostosowanych do potrzeb żywieniowych organizmu pacjenta, pozwalał na zarządzanie stworzonymi jadłospisami
i udostępnianie ich pacjentowi w~formie umożliwiającej w~jak najprostszy sposób zastosowanie przez pacjenta przygotowanej dla niego diety,
a także umożliwiał kontrolę rezultatów stosowania jadłospisu przez pacjenta.

\par
Obserwując trendy występujące we współczesnym społeczeństwie można zauważyć, że zdrowy styl życia stał się modny, a~czasem nawet utożsamiany ze statusem społecznym.
W związku z~tą tendencją coraz więcej osób regularnie uprawia sport, rezygnuje z~używek, a~także dba o~dietę.
Analizując przedstawione na rysunku \ref{fig:dietetyk-trend} dane dotyczące wyszukiwania hasła "dietetyk" w latach 2004-2019 poprzez narzędzie Google Trends\cite{url:google-trends} można zauważyć,
że popularność wyszukiwania tego hasła w~latach 2016-2019 jest ponad 5~krotnie większa niż w~roku 2004.

\imagewide[\cite{url:google-trends}]{img/dietetyk-trend.jpg}{Zainteresowanie hasłem "dietetyk" w~ujęciu czasowym}{dietetyk-trend}

Zwiększone zainteresowanie usługami dietetycznymi powoduje zwiększone zapotrzebowanie na wysokiej jakości, nowoczesne narzędzia wspomagające pracę dietetyka.
W chwili pisania niniejszej pracy w~Polsce popularność zyskało jedynie kilka programów oferujących kompleksowe funkcjonalności potrzebne w~codziennej praktyce dietetyka,
można więc sądzić, że rynek aplikacji tego typu nie został jeszcze nasycony.

\par
Biorąc pod uwagę powyższe spostrzeżenia, warto w~tym miejscu podkreślić, że Światowa Organizacja Zdrowia (ang. World Health Organisation~- WHO) określiła otyłość
(i bardziej ogólnie choroby dietozależne) jako jeden z~głównych problemów zdrowia publicznego\cite{article:dietetyk-na-rynku-uslug-medycznych}.
Fakt ten podkreśla jak duże brzemię odpowiedzialności spoczywa na ramionach dietetyków,
a~co za tym idzie jak istotne jest dostarczenie specjalistom narzędzi ułatwiających niesienie pomocy pacjentom.

\section*{Cel pracy}\label{sec:thesis-goal}

Celem pracy jest projekt i~budowa platformy do zarządzania dietą w~oparciu o~architekturę mikroserwisów.
Tworzona platforma będzie obejmowała cały proces zarządzania dietą, czyli przede wszystkim:
zebranie przez dietetyka wywiadu żywieniowego od pacjenta,
stworzenie przez dietetyka jadłospisu,
udostępnienie jadłospisu pacjentowi,
ułatwienie pacjentowi stosowania diety poprzez generowanie listy zakupów
oraz kontrolę rezultatów diety.

\section*{Zakres pracy}\label{sec:scope-of-work}

Praca w~swoim zakresie zawiera opracowanie projektu systemu, w~ramach którego, między innymi,
omówieni zostaną użytkownicy systemu wraz z~ich potrzebami,
przeprowadzaona zostanie dekompozycja problemu w oparciu o poddziedizny,
przygotowane zostaną prototypy interfejsu i~diagramy UML takie jak diagram przypadków użycia i diagram klas.
Omówiona zostanie również podstawowa architektura systemu.
\par
W części praktycznej pracy analizie poddane będą wybrane narzędzia i technologie wykorzystane do implementacji projektu,
omówiony zostanie zakres implementacji, a~także przedstawiony będzie rezultat tejże implmentacji w formie prezentacji działania aplikacji.
Przedstawiona zostanie również dokumentacja kodu i~testy oraz pokrótce omówiony będzie sposób instalacji systemu z~uwzględnieniem wymagań wstępnych.

\thispagestyle{normal}


    % !TeX spellcheck = pl_PL
\chapter{Stan wiedzy i~techniki w~zakresie tematyki pracy}\label{ch:knowladge-state}
\section{Przegląd istniejących rozwiązań konkurencyjnych}\label{sec:competitive-solutions}
\todo{opisać rozwiązania konkurencyjne}
\begin{itemize}
    \item TiqDiet\cite{url:tiqdiet}
    \item Kcalmar PRO\cite{url:kcalmar}
    \item Dietetyk Pro\cite{url:dietetyk-pro}
    \item Aliant\cite{url:aliant}
    \item Dietico\cite{url:dietico}
    \item Vitme\cite{url:vitme}
\end{itemize}

\begin{minipage}{\textwidth}
    \begin{table}[H]
        \centering\caption{Rozwiązania konkurencyjne - cechy funkcjonalne (opr.wł)\label{tabela:rozwiazania-konkurencyjne-funkcjonalne}}
        \begin{tabular}{|P{.22\textwidth}|P{.09\textwidth}|P{.09\textwidth}|P{.09\textwidth}|P{.09\textwidth}|P{.09\textwidth}|P{.09\textwidth}|}
            \hline
                                                            & \cellgray{Tiqdiet}   & \cellgray{Kcalmar Pro}   & \cellgray{Dietetyk Pro} & \cellgray{Aliant}         & \cellgray{Dietico}   & \cellgray{Vitme}   \\ \hline
            \cellgray{Tworzenie jadłospisów}               & \cellgreen{TAK}       & \cellgreen{TAK}           & \cellgreen{TAK}          & \cellgreen{TAK}            & \cellgreen{TAK}       & \cellgreen{TAK}     \\ \hline
            \cellgray{Gotowe szablony diet}                & \cellgreen{TAK}       & \cellgreen{TAK}           & \cellgreen{TAK}          & \cellgreen{TAK}            & \cellred{NIE}       & \cellred{NIE}     \\ \hline
            \cellgray{Zapis diety do pliku}                & \cellgreen{TAK}       & \cellgreen{TAK}           & \cellgreen{TAK}          & \cellgreen{TAK}            & \cellgreen{TAK}       & \cellgreen{TAK}     \\ \hline
            \cellgray{Wysyłanie diet do pacjenta}          & \cellgreen{TAK}       & \cellgreen{TAK}           & \cellgreen{TAK}          & \cellgreen{TAK}            & \cellred{NIE}       & \cellgreen{TAK}     \\ \hline
            \cellgray{Komunikacja z~pacjentem}             & \cellgreen{TAK}       & \cellgreen{TAK}           & \cellgreen{TAK}          & \cellred{NIE}            & \cellred{NIE}       & \cellgreen{TAK}     \\ \hline
            \cellgray{Karta pacjenta}                      & \cellgreen{TAK}       & \cellgreen{TAK}           & \cellgreen{TAK}          & \cellgreen{TAK}            & \cellgreen{TAK}       & \cellgreen{TAK}     \\ \hline
            \cellgray{Wywiad żywieniowy}                   & \cellgreen{TAK}       & \cellgreen{TAK}           & \cellgreen{TAK}          & \cellred{NIE}            & \cellgreen{TAK}       & \cellgreen{TAK}     \\ \hline
            \cellgray{Lista zakupów}                       & \cellgreen{TAK}       & \cellgreen{TAK}           & \cellgreen{TAK}          & \cellgreen{TAK}            & \cellgreen{TAK}       & \cellred{NIE}     \\ \hline
            \cellgray{Dodawanie własnych produktów}        & \cellgreen{TAK}       & \cellgreen{TAK}           & \cellgreen{TAK}          & \cellgreen{TAK}            & \cellgreen{TAK}       & \cellgreen{TAK}     \\ \hline
            \cellgray{Dodawanie własnych przepisów}        & \cellgreen{TAK}       & \cellgreen{TAK}           & \cellgreen{TAK}          & \cellgreen{TAK}            & \cellgreen{TAK}       & \cellgreen{TAK}     \\ \hline
        \end{tabular}
    \end{table}
\end{minipage}

\begin{minipage}{\textwidth}
    \begin{table}[H]
        \centering\caption{Rozwiązania konkurencyjne - cechy niefunkcjonalne (opr.wł)\label{tabela:rozwiazania-konkurencyjne-niefunkcjonalne}}
        \begin{tabular}{|P{.22\textwidth}|P{.09\textwidth}|P{.09\textwidth}|P{.09\textwidth}|P{.09\textwidth}|P{.09\textwidth}|P{.09\textwidth}|}
            \hline
                                                            & \cellgray{Tiqdiet}   & \cellgray{Kcalmar Pro}   & \cellgray{Dietetyk Pro} & \cellgray{Aliant}         & \cellgray{Dietico}   & \cellgray{Vitme}   \\ \hline
            \cellgray{Liczba produktów w~bazie}            & 1000      & 1400          & 6000         & 3500           & 900       & 5000    \\ \hline
            \cellgray{Liczba gotowych przepisów}           & 200       & 800           & 2800         & 1700           & 1900      & 400     \\ \hline
            \cellgray{Praca offline}                       & \cellred{NIE}       & \cellred{NIE}           & \cellred{NIE}          & \cellgreen{TAK}            & \cellred{NIE}       & \cellred{NIE}     \\ \hline
            \cellgray{Praca online}                        & \cellgreen{TAK}       & \cellgreen{TAK}           & \cellgreen{TAK}          & \cellred{NIE}            & \cellgreen{TAK}       & \cellgreen{TAK}     \\ \hline
            \cellgray{Aplikacja mobilna dla dietetyka}     & \cellgreen{TAK}       & \cellgreen{TAK}           & \cellgreen{TAK}          & \cellred{NIE}            & \cellred{NIE}       & \cellred{NIE}     \\ \hline
            \cellgray{Aplikacja mobilna dla pacjenta}      & \cellgreen{TAK}       & \cellgreen{TAK}           & \cellred{NIE}          & \cellred{NIE}            & \cellred{NIE}       & \cellgreen{TAK}     \\ \hline
            \cellgray{Dostęp dla pacjenta przez przeglądarkę internetową}       & \cellgreen{TAK}       & \cellgreen{TAK}           & \cellgreen{TAK}          & \cellred{NIE}            & \cellred{NIE}       & \cellgreen{TAK}     \\ \hline
            \cellgray{Darmowy okres testowy}               & 14dni     & 14dni         & 7dni         & bezter- minowo & 14dni     & 14dni   \\ \hline
            \cellgray{Cena w~abonamencie rocznym}          & 199       & 1188          & 246          & 699            & 546       & 219     \\ \hline
        \end{tabular}
    \end{table}
\end{minipage}

\section{Przegląd przydatnych technologii i~technik}\label{sec:usefull-technologies}
\todo{Tutaj opisać architekturę aplikacji webowych. Porównać monolit, soa i~mikroserwisy}

\section{Przegląd literatury dietetycznej}\label{sec:domain-literature}
\todo{uzupełnić literaturę}
%https://www.pum.edu.pl/__data/assets/pdf_file/0009/82881/Instrukcja-i-formularz-wywiadu-zywieniowego.pdf
%https://www.akademiadietetyki.pl/literatura-obowiazkowa-i-uzupelniajaca/
%http://www.uni.olsztyn.pl/wnm1/onkologia/index.php/dietetyka/rok-iii/literatura.html
%http://www.mckp.uj.edu.pl/cm/uploads/2017/08/7.-Zywienie-literatura.pdf

\thispagestyle{normal}

    % !TeX spellcheck = pl_PL
\chapter{Założenia projektowe}\label{ch:design-assumptions}
\section{Uwagi wstępne}\label{sec:presumptions}
W niniejszym rozdziale opisano wizję systemu, który będzie wspomagał układanie diety. Aplikacja
będzie składać się z~następujących podstawowych modułów: produkty, przepisy, jadłospisy
oraz wizyty.
\par
Zalogowani dietetycy będą mogli zarządzać produktami, ich wartościami odżywczymi oraz
miarami domowymi. Korzystając ze stworzonych produktów dietetycy będą mogli tworzyć
przepisy, a~następnie, w~ramach jadłospisu, dodawać do planów posiłków przepisy
i pojedyncze produkty.
\par
Dietetycy będą mogli również zarządzać pacjentami i~ich wizytami. W~ramach wizyty dietetyk
będzie mógł przeprowadzić wywiad żywieniowy, zebrać pomiary ciała pacjenta i~przydzielić
pacjentowi jadłospis.


\section{Słownik pojęć domenowych}\label{sec:dictionary}
\todo{uzupełnić słownik}
\begin{itemize}[series=atr, wide, align=left, leftmargin=190pt]
    \atr{Administrator}- użytkownik posiadający uprawnienia do zarządzania uprawnieniami użytkowników
    \atr{BIA}- metoda impedancji bioelektrycznej wykorzystywana do analizy składu ciała%todo metoda pomiaru ciała
    \atr{BMI}- wskaźnik masy ciała% todo
    \atr{CPM}- całkowita przemiana materii %todo source
    \atr{Dieta}- sposób odżywiania% todo
    \atr{Dietetyk}- specjalista w~dziedzinie dietetyki
    \atr{Jadłospis}- plan posiłków zdefiniowany na określoną liczbę dni z~uwzględnieniem określonych wymagań
    \atr{Karta pacjenta}- karta przedstawiająca przebieg współpracy dietetyka z~pacjentem
    \atr{MET}- ekwiwalent metaboliczny%todo
    \atr{Miara domowa}- definicja pospolitej miary, takiej jak np. łyżeczka w~gramach
    \atr{Pacjent}- klient dietetyka
    \atr{PAL}- współczynnik aktywności fizycznej
    \atr{Podstawowe wartości odżywcze}- energia, białko, tłuszcz, węglowodany%todo
    \atr{Pomiary ciała}- pomiary ciała pacjenta przeprowadzane przez dietetyka
    \atr{Posiłek}- posiłek jest przydzielany do jadłospisu; zawiera produkty i~przepisy
    \atr{PPM}- podstawowa przemiana materii %todo source
    \atr{Produkt}- produkt spożywczy, dla którego specyfikowane są wartości odżywcze i~miary domowe
    \atr{Przepis}- opis składników i~kroków przygotowania dania
    \atr{Sekcja przepisu}- semantyczny podział przepisu, np. sernik może mieć sekcje związane z~przygotowaniem ciasta, nadzienia i~polewy
    \atr{USDA}- Departament Rolnictwa Stanów Zjednoczonych% todo
    \atr{Wartość odżywcza}- ilość elementu takiego jak np. węglowodanów albo białka w~100g produktu
    \atr{Wizyta}- konkretna wizyta pacjenta
    \atr{Wywiad żywieniowy}- wywiad przeprowadzany z~pacjentem uwzględniający jego nawyki żywieniowe, nietolerancje, choroby, przyjmowane leki, itp.
\end{itemize}

\section{Sformułowanie problemu}\label{sec:problem-specification}
\noindent
\begin{minipage}{\textwidth}
    \begin{table}[H]
        \centering\caption{Sformułowanie problemu (opr.wł)\label{tabela:sformulowanie-problemu}}
        \begin{tabular}{|P{.2\textwidth}|P{.7\textwidth}|}

            \hline
            \cellgray{Problem} &
            \multicolumn{1}{|l|}{Problem z~ręcznym układaniem jadłospisu}\\
            \hline

            \cellgray{Dotyczy} &
            \multicolumn{1}{|l|}{Dietetyków}\\
            \hline

            \cellgray{Wpływ problemu} &
            \begin{itemize}
                \item Dietetyk poświęca dużo czasu na wyszukiwanie informacji o~każdym produkcie, którego potrzebuje wykorzystać w~układanym jadłospisie
                \item Dietetyk poświęca dużo czasu na obliczanie wartości odżywczych w~każdym przepisie
                \item Dietetyk poświęca dużo czasu na obliczanie wartości odżywczych w~każdym jadłospisie.
                \item Dietetyk ma problem z~przeliczeniem miar domowych produktów na gramy
            \end{itemize} \\
            \hline

            \cellgray{Pomyślne rozwiązanie} &
            \begin{itemize}
                \item Będzie zwalniało dietetyka z~konieczności obliczania wartości odżywczych dla przepisów i~jadłospisów
                \item Będzie ułatwiało dietetykowi przekazywanie stworzonego jadłospisu pacjentów
            \end{itemize} \\
            \hline
        \end{tabular}
    \end{table}
\end{minipage}
\section{Pozycjonowanie produktu}\label{sec:product-positioning}
\noindent
\begin{minipage}{\textwidth}
    \begin{table}[H]
        \centering\caption{Pozycjonowanie produktu (opr.wł)\label{tabela:pozycjonowanie-produktu}}
        \begin{tabular}{|P{.2\textwidth}|P{.7\textwidth}|}

            \hline
            \cellgray{Dla} &
            \multicolumn{1}{|l|}{Dietetyka}\\
            \hline

            \cellgray{Który} &
            \multicolumn{1}{|l|}{Chce łatwiej zarządzać tworzeniem jadłospisu}\\
            \hline

            \cellgray{Nazwa produktu} &
            \multicolumn{1}{|l|}{Webowa aplikacja wspomagająca układanie jadłospisu} \\
            \hline

            \cellgray{Który} &
            \multicolumn{1}{|l|}{Skraca czas potrzebny na ułożenie i~zarządzanie jadłospisami} \\
            \hline

            \cellgray{Inaczej niż} &
            \multicolumn{1}{|l|}{Kalkulator kalorii} \\
            \hline

            \cellgray{Nasz produkt} &
            \multicolumn{1}{|l|}{Skupia się na tworzeniu i~udostępnianiu jadłospisów} \\
            \hline
        \end{tabular}
    \end{table}
\end{minipage}

%\section{Opis udziałowców i~użytkowników}
%\subsection{Podsumowanie udziałowców}

\section{Podsumowanie użytkowników systemu}\label{sec:users-summary}
\noindent
\begin{minipage}{\textwidth}
    \begin{table}[H]
        \centering\caption{Użytkownicy (opr.wł)\label{tabela:uzytkownicy}}
        \begin{tabular}{|P{.15\textwidth}|P{.25\textwidth}|P{.5\textwidth}|}

            \hline
            \cellgray{Nazwa & \cellcolor[HTML]{DDDDDD}Opis} & \cellcolor[HTML]{DDDDDD}Odpowiedzialności\\

            \hline
            Gość &
            Niezalogowany użytkownik &
            \begin{itemize}
                \item Zakłada konto użytkownika
                \item Wyświetla stronę główną
            \end{itemize} \\
            \hline
            Pacjent &
            Klient dietetyka &
            \begin{itemize}
                \item Otrzymuje ułożony jadłospis
            \end{itemize} \\
            \hline
            Dietetyk &
            Specjalista w~dziedzinie dietetyki &
            \begin{itemize}
                \item Używa założonego konta
                \item Wprowadza, edytuje i~usuwa produkty, przepisy i~jadłospisy
            \end{itemize} \\
            \hline
            Administrator &
            Osoba zarządzająca działaniem aplikacji &
            \begin{itemize}
                \item Przydzielanie i~odbieranie użytkownikom uprawnień
                \item Zarządzanie definicjami wartości odżywczych, typami diet, typami posiłków, typami dań i~wyposażeniem kuchennym
                \item Zarządzanie treścią witryny, informacjami kontaktowymi i~cennikiem
            \end{itemize} \\
            \hline
        \end{tabular}
    \end{table}
\end{minipage}

\section{Wymagania funkcjonalne}\label{sec:functional-requirements}
\noindent
\begin{minipage}{\textwidth}
    \begin{table}[H]
        \centering\caption{Wymagania funkcjonalne ogólne (opr.wł)\label{tabela:wymaganiaFunkcjonalneOgolne}}
        \begin{tabular}{|P{.3\textwidth}|P{.6\textwidth}|}
            \hline
            \cellgray{Potrzeby} & \cellcolor[HTML]{DDDDDD}Cechy \\

            \hline
            Administrator potrzebuje widzieć listę użytkowników &
            \begin{itemize}
                \item Przydzielanie i~odbieranie użytkownikom uprawnień
            \end{itemize} \\
            \hline
            Administrator potrzebuje zarządzać witryną &
            \begin{itemize}
                \item Zarządzanie treścią strony głównej
                \item Zarządzanie treścią polityki prywatności
                \item Zarządzanie treścią warunków korzystania z~usługi
                \item Zarządzanie treścią często zadawanych pytań
                \item Zarządzanie informacjami kontaktowymi
                \item Zarządzanie cennikiem
            \end{itemize} \\
            \hline
            Użytkownik potrzebuje korzystać ze swojego konta &
            \begin{itemize}
                \item Logowanie do systemu
                \item Przypomnienie hasła
                \item Zarządzanie swoimi danymi osobowymi
            \end{itemize} \\
            \hline
            Użytkownik chce przeglądać witrynę w~swoim języku &
            \begin{itemize}
                \item Obsługa witryny w~wielu językach
            \end{itemize} \\
            \hline
            Gość potrzebuje korzystać z~systemu &
            \begin{itemize}
                \item Zakładanie konta użytkownika
            \end{itemize} \\
            \hline
        \end{tabular}
    \end{table}
\end{minipage}

\begin{minipage}{\textwidth}
    \begin{table}[H]
        \centering\caption{Wymagania funkcjonalne dla produktów (opr.wł)\label{tabela:wymaganiaFunkcjonalneProdukty}}
        \begin{tabular}{|P{.3\textwidth}|P{.6\textwidth}|}
            \hline
            \cellgray{Potrzeby} & \cellcolor[HTML]{DDDDDD}Cechy \\

            \hline
            Administrator potrzebuje zarządzać definicjami potrzebnymi w~produktach &
            \begin{itemize}
                \item Zarządzanie definicjami wartości odżywczych
                \item Zarządzanie kategoriami produktów
                \item Zarządzanie rodzajami diet
            \end{itemize} \\
            \hline
            Dietetyk potrzebuje widzieć listę produktów &
            \begin{itemize}
                \item Wyszukiwanie produktów
                \item Filtrowanie produktów
                \item Dodawanie nowych produktów
            \end{itemize} \\
            \hline
            Dietetyk potrzebuje zarządzać szczegółami produktu &
            \begin{itemize}
                \item Edytowanie i~usuwanie produktów
                \item Definiowanie wartości odżywczych dla produktu
                \item Definiowanie miar domowych dla produktu
                \item Przypisywanie produktu do kategorii i~podkategorii
                \item Definiowanie do jakich typów diet produkt nadaje się, a~do jakich nie
            \end{itemize} \\
            \hline
        \end{tabular}
    \end{table}
\end{minipage}

\begin{minipage}{\textwidth}
    \begin{table}[H]
        \centering\caption{Wymagania funkcjonalne dla przepisów (opr.wł)\label{tabela:wymaganiaFunkcjonalnePrzepisy}}
        \begin{tabular}{|P{.3\textwidth}|P{.6\textwidth}|}
            \hline
            \cellgray{Potrzeby} & \cellcolor[HTML]{DDDDDD}Cechy \\

            \hline
            Administrator potrzebuje zarządzać definicjami potrzebnymi w~przepisach &
            \begin{itemize}
                \item Zarządzanie typami posiłków
                \item Zarządzanie typami dań
                \item Zarządzanie definicjami wyposażenia kuchennego
            \end{itemize} \\
            \hline
            Dietetyk potrzebuje widzieć listę przepisów &
            \begin{itemize}
                \item Wyszukiwanie przepisów
                \item Filtrowanie przepisów
                \item Dodawanie nowych przepisów
            \end{itemize} \\
            \hline
            Dietetyk potrzebuje zarządzać szczegółami przepisu &
            \begin{itemize}
                \item Edytowanie i~usuwanie przepisów
                \item Dodawanie wielu sekcji do przepisu
                \item Dodawanie do każdej sekcji listy składników
                \item Dodawanie do każdej sekcji sposobu przygotowania
                \item Dodawanie zdjęcia dania do przepisu
                \item Definiowanie czasu przygotowania posiłku
            \end{itemize} \\
            \hline
        \end{tabular}
    \end{table}
\end{minipage}

\begin{minipage}{\textwidth}
    \begin{table}[H]
        \centering\caption{Wymagania funkcjonalne dla jadłospisów (opr.wł)\label{tabela:wymaganiaFunkcjonalneJadlospisy}}
        \begin{tabular}{|P{.3\textwidth}|P{.6\textwidth}|}
            \hline
            \cellgray{Potrzeby} & \cellcolor[HTML]{DDDDDD}Cechy \\

            \hline
            Dietetyk potrzebuje widzieć listę jadłospisów &
            \begin{itemize}
                \item Wyszukiwanie jadłospisów
                \item Filtrowanie jadłospisów
                \item Dodawanie nowych jadłospisów
            \end{itemize} \\
            \hline
            Dietetyk potrzebuje zarządzać szczegółami jadłospisu &
            \begin{itemize}
                \item Dodawanie, edytowanie i~usuwanie jadłospisów
                \item Definiowanie liczby dni na które będzie układany jadłospis
                \item Definiowanie liczby posiłków dziennie
                \item Definiowanie planowanego czasu każdego z~posiłków
                \item Definiowanie procentowego udziału podstawowych wartości odżywczych w~każdym posiłku
                \item Definiowanie posiłków w~jadłospisie
                \item Dodawanie produktów i~przepisów do posiłków
            \end{itemize} \\
            \hline
        \end{tabular}
    \end{table}
\end{minipage}

\begin{minipage}{\textwidth}
    \begin{table}[H]
        \centering\caption{Wymagania funkcjonalne dla wizyt (opr.wł)\label{tabela:wymaganiaFunkcjonalneWizyty}}
        \begin{tabular}{|P{.3\textwidth}|P{.6\textwidth}|}
            \hline
            \cellgray{Potrzeby} & \cellcolor[HTML]{DDDDDD}Cechy \\

            \hline
            Dietetyk potrzebuje wyświetlać listę swoich pacjentów &
            \begin{itemize}
                \item Wyszukiwanie pacjentów
                \item Wyświetlanie listy znalezionych pacjentów
                \item Wyświetlanie listy umówionych wizyt
                \item Wyświetlanie listy oczekujących porad
                \item Dodawanie nowych pacjentów
            \end{itemize} \\
            \hline
            Dietetyk potrzebuje zarządzać kartą pacjenta &
            \begin{itemize}
                \item Wyświetlanie i~edytowanie podstawowych informacji pacjenta
                \item Wyświetlanie listy wizyt pacjenta
                \item Wyświetlanie listy oczekujących porad pacjenta
                \item Dodawanie nowej wizyty pacjenta
            \end{itemize} \\
            \hline
            Dietetyk potrzebuje wyświetlać szczegóły wizyty pacjenta &
            \begin{itemize}
                \item Wyświetlanie i~edytowanie szczegółów wizyty pacjenta
                \item Zarządzanie pomiarami ciała pacjenta przypisanymi do wizyty
                \item Zarządzanie wywiadem żywieniowym przypisanym do wizyty
                \item Zarządzanie jadłospisem przydzielonym do wizyty
            \end{itemize} \\
            \hline
            Pacjent potrzebuje otrzymywać dietę &
            \begin{itemize}
                \item Udostępnianie pacjentowi jadłospisu
            \end{itemize} \\
            \hline
            Pacjent chce mieć wgląd w~swoją kartę &
            \begin{itemize}
                \item Logowanie do konta utworzonego w~serwisie
                \item Dodawanie kart pacjenta do swojego konta po udostępnieniu ich przez dietetyka
            \end{itemize} \\
            \hline
            Pacjent chce wyrazić opinię o~wizycie &
            \begin{itemize}
                \item Ocenianie odbytej wizyty
            \end{itemize} \\
            \hline
            Pacjent chce znaleźć dietetyka &
            \begin{itemize}
                \item Wyświetlanie listy dietetyków
                \item Wyświetlanie profilu dietetyka
                \item Wyświetlanie list opinii o~wybranym dietetyku
                \item Kontakt z~wybranym dietetykiem
            \end{itemize} \\
            %            \hline
            %            Pacjent zarządza dietą za pomocą asystenta głosowego &
            %            \begin{itemize}
            %                \item Wydawanie w~języku naturalnym poleceń dotyczących diety
            %            \end{itemize} \\
            \hline
        \end{tabular}
    \end{table}
\end{minipage}

\newpage

\section{Wymagania niefunkcjonalne}\label{sec:nonfunctional-requirements}
\begin{itemize}
    \item System działa poprawnie w~przeglądarkach Google Chrome 76, Mozilla Firefox 69, Safari 12, Opera 63, Microsoft Edge 17
    \item System działa na urządzenia mobilnych korzystających z~systemu Android i~iOS
    \item System jest dostępny w~polskiej i~angielskiej wersji językowej
    \item System ma czytelny i~minimalistyczny interfejs
    \item Aplikacja webowa jest w~pełni responsywna i~wygodna do używania na ekranach od 5~do 30 cali
    %    \item Aplikacja webowa udostępnia część funkcji offline
    \item Aplikacja ma być oparta na architekturze mikroserwisów
\end{itemize}
\thispagestyle{normal}


    \chapter{Projekt}\label{ch:project}

\section{Przypadki użycia}\label{sec:usecase}
\todo{diagram przypadków użycia}
\todo{scenariusze przypadków użycia}

\begin{minipage}{\textwidth}
    \begin{figure}[H]
        \centering\includegraphics[scale=0.55]{../uml/use_case_diagrams/users.png}
        \caption{Użytkownicy - diagram przypadków użycia (opr.wł).}\label{rysunek:use-case-diagram-users}
    \end{figure}
\end{minipage}

\begin{minipage}{\textwidth}
    \begin{figure}[H]
        \centering\includegraphics[scale=0.55]{../uml/use_case_diagrams/gateway.png}
        \caption{Gateway - diagram przypadków użycia (opr.wł).}\label{rysunek:use-case-diagram-gateway}
    \end{figure}
\end{minipage}

\begin{minipage}{\textwidth}
    \begin{figure}[H]
        \centering\includegraphics[scale=0.55]{../uml/use_case_diagrams/products.png}
        \caption{Produkty - diagram przypadków użycia (opr.wł).}\label{rysunek:use-case-diagram-products}
    \end{figure}
\end{minipage}

\begin{minipage}{\textwidth}
    \begin{figure}[H]
        \centering\includegraphics[scale=0.55]{../uml/use_case_diagrams/recipes.png}
        \caption{Przepisy - diagram przypadków użycia (opr.wł).}\label{rysunek:use-case-diagram-recipes}
    \end{figure}
\end{minipage}

\begin{minipage}{\textwidth}
    \begin{figure}[H]
        \centering\includegraphics[scale=0.55]{../uml/use_case_diagrams/mealplans.png}
        \caption{Jadłospisy - diagram przypadków użycia (opr.wł).}\label{rysunek:use-case-diagram-mealplans}
    \end{figure}
\end{minipage}

\begin{minipage}{\textwidth}
    \begin{figure}[H]
        \centering\includegraphics[scale=0.55]{../uml/use_case_diagrams/appointments.png}
        \caption{Wizyty - diagram przypadków użycia (opr.wł).}\label{rysunek:use-case-diagram-appointments}
    \end{figure}
\end{minipage}

\section{Prototyp interfejsu}\label{sec:mockups}
\todo{Podpisy i opisy mockupów}

\begin{minipage}{\textwidth}
    \begin{figure}[H]
        \centering\includegraphics[scale=0.55]{../mockup/0home.png}
        \caption{Prototyp interfejsu - 0home (opr.wł)}\label{rysunek:0home}
    \end{figure}
\end{minipage}
\begin{minipage}{\textwidth}
    \begin{figure}[H]
        \centering\includegraphics[scale=0.55]{../mockup/0home_1sign-up.png}
        \caption{Prototyp interfejsu - 0home:1sign-up (opr.wł)}\label{rysunek:0home_1sign-up}
    \end{figure}
\end{minipage}
\begin{minipage}{\textwidth}
    \begin{figure}[H]
        \centering\includegraphics[scale=0.55]{../mockup/0home_2login.png}
        \caption{Prototyp interfejsu - 0home:2login (opr.wł)}\label{rysunek:0home_2login}
    \end{figure}
\end{minipage}
\begin{minipage}{\textwidth}
    \begin{figure}[H]
        \centering\includegraphics[scale=0.55]{../mockup/1products.png}
        \caption{Prototyp interfejsu - 1products (opr.wł)}\label{rysunek:1products}
    \end{figure}
\end{minipage}
\begin{minipage}{\textwidth}
    \begin{figure}[H]
        \centering\includegraphics[scale=0.55]{../mockup/1products_1new.png}
        \caption{Prototyp interfejsu - 1products:1new (opr.wł)}\label{rysunek:1products_1new}
    \end{figure}
\end{minipage}
\begin{minipage}{\textwidth}
    \begin{figure}[H]
        \centering\includegraphics[scale=0.55]{../mockup/1products_2details.png}
        \caption{Prototyp interfejsu - 1products:2details (opr.wł)}\label{rysunek:1products_2details}
    \end{figure}
\end{minipage}
\begin{minipage}{\textwidth}
    \begin{figure}[H]
        \centering\includegraphics[scale=0.55]{../mockup/2recipes.png}
        \caption{Prototyp interfejsu - 2recipes (opr.wł)}\label{rysunek:2recipes}
    \end{figure}
\end{minipage}
\begin{minipage}{\textwidth}
    \begin{figure}[H]
        \centering\includegraphics[scale=0.55]{../mockup/2recipes_1new.png}
        \caption{Prototyp interfejsu - 2recipes:1new (opr.wł)}\label{rysunek:2recipes_1new}
    \end{figure}
\end{minipage}
\begin{minipage}{\textwidth}
    \begin{figure}[H]
        \centering\includegraphics[scale=0.55]{../mockup/2recipes_1new_1add-product.png}
        \caption{Prototyp interfejsu - 2recipes:1new:1add-product (opr.wł)}\label{rysunek:2recipes_1new_1add-product}
    \end{figure}
\end{minipage}
\begin{minipage}{\textwidth}
    \begin{figure}[H]
        \centering\includegraphics[scale=0.55]{../mockup/2recipes_2details.png}
        \caption{Prototyp interfejsu - 2recipes:2details (opr.wł)}\label{rysunek:2recipes_2details}
    \end{figure}
\end{minipage}
\begin{minipage}{\textwidth}
    \begin{figure}[H]
        \centering\includegraphics[scale=0.55]{../mockup/3mealplans.png}
        \caption{Prototyp interfejsu - 3mealplans (opr.wł)}\label{rysunek:3mealplans}
    \end{figure}
\end{minipage}
\begin{minipage}{\textwidth}
    \begin{figure}[H]
        \centering\includegraphics[scale=0.55]{../mockup/3mealplans_1new_1settings.png}
        \caption{Prototyp interfejsu - 3mealplans:1new:1settings (opr.wł)}\label{rysunek:3mealplans_1new_1settings}
    \end{figure}
\end{minipage}
\begin{minipage}{\textwidth}
    \begin{figure}[H]
        \centering\includegraphics[scale=0.55]{../mockup/3mealplans_1new_2calendar.png}
        \caption{Prototyp interfejsu - 3mealplans:1new:2calendar (opr.wł)}\label{rysunek:3mealplans_1new_2calendar}
    \end{figure}
\end{minipage}
\begin{minipage}{\textwidth}
    \begin{figure}[H]
        \centering\includegraphics[scale=0.55]{../mockup/3mealplans_1new_2calendar_1meal.png}
        \caption{Prototyp interfejsu - 3mealplans:1new:2calendar:1meal (opr.wł)}\label{rysunek:3mealplans_1new_2calendar_1meal}
    \end{figure}
\end{minipage}
\begin{minipage}{\textwidth}
    \begin{figure}[H]
        \centering\includegraphics[scale=0.55]{../mockup/3mealplans_1new_2calendar_2add-product.png}
        \caption{Prototyp interfejsu - 3mealplans:1new:2calendar:2add-product (opr.wł)}\label{rysunek:3mealplans_1new_2calendar_2add-product}
    \end{figure}
\end{minipage}
\begin{minipage}{\textwidth}
    \begin{figure}[H]
        \centering\includegraphics[scale=0.55]{../mockup/3mealplans_1new_2calendar_3add-recipe.png}
        \caption{Prototyp interfejsu - 3mealplans:1new:2calendar:3add-recipe (opr.wł)}\label{rysunek:3mealplans_1new_2calendar_3add-recipe}
    \end{figure}
\end{minipage}
\begin{minipage}{\textwidth}
    \begin{figure}[H]
        \centering\includegraphics[scale=0.55]{../mockup/3mealplans_2details_1settings.png}
        \caption{Prototyp interfejsu - 3mealplans:2details:1settings (opr.wł)}\label{rysunek:3mealplans_2details_1settings}
    \end{figure}
\end{minipage}
\begin{minipage}{\textwidth}
    \begin{figure}[H]
        \centering\includegraphics[scale=0.55]{../mockup/3mealplans_2details_2calendar.png}
        \caption{Prototyp interfejsu - 3mealplans:2details:2calendar (opr.wł)}\label{rysunek:3mealplans_2details_2calendar}
    \end{figure}
\end{minipage}
\begin{minipage}{\textwidth}
    \begin{figure}[H]
        \centering\includegraphics[scale=0.55]{../mockup/3mealplans_2details_2calendar_1meal.png}
        \caption{Prototyp interfejsu - 3mealplans:2details:2calendar:1meal (opr.wł)}\label{rysunek:3mealplans_2details_2calendar_1meal}
    \end{figure}
\end{minipage}
\begin{minipage}{\textwidth}
    \begin{figure}[H]
        \centering\includegraphics[scale=0.55]{../mockup/4appointments.png}
        \caption{Prototyp interfejsu - 4appointments (opr.wł)}\label{rysunek:4appointments}
    \end{figure}
\end{minipage}
\begin{minipage}{\textwidth}
    \begin{figure}[H]
        \centering\includegraphics[scale=0.55]{../mockup/4appointments_1new-patient-card.png}
        \caption{Prototyp interfejsu - 4appointments:1new-patient-card (opr.wł)}\label{rysunek:4appointments_1new-patient-card}
    \end{figure}
\end{minipage}
\begin{minipage}{\textwidth}
    \begin{figure}[H]
        \centering\includegraphics[scale=0.55]{../mockup/4appointments_2patient-card-details.png}
        \caption{Prototyp interfejsu - 4appointments:2patient-card-details (opr.wł)}\label{rysunek:4appointments_2patient-card-details}
    \end{figure}
\end{minipage}
\begin{minipage}{\textwidth}
    \begin{figure}[H]
        \centering\includegraphics[scale=0.55]{../mockup/4appointments_3new-appointment.png}
        \caption{Prototyp interfejsu - 4appointments:3new-appointment (opr.wł)}\label{rysunek:4appointments_3new-appointment}
    \end{figure}
\end{minipage}
\begin{minipage}{\textwidth}
    \begin{figure}[H]
        \centering\includegraphics[scale=0.55]{../mockup/4appointments_4appointment-details.png}
        \caption{Prototyp interfejsu - 4appointments:4appointment-details (opr.wł)}\label{rysunek:4appointments_4appointment-details}
    \end{figure}
\end{minipage}
\begin{minipage}{\textwidth}
    \begin{figure}[H]
        \centering\includegraphics[scale=0.55]{../mockup/4appointments_4appointment-details_1nutritional-interview.png}
        \caption{Prototyp interfejsu - 4appointments:4appointment-details:1nutritional-interview (opr.wł)}\label{rysunek:4appointments_4appointment-details_1nutritional-interview}
    \end{figure}
\end{minipage}
\begin{minipage}{\textwidth}
    \begin{figure}[H]
        \centering\includegraphics[scale=0.55]{../mockup/4appointments_4appointment-details_2body-measurement.png}
        \caption{Prototyp interfejsu - 4appointments:4appointment-details:2body-measurement (opr.wł)}\label{rysunek:4appointments_4appointment-details_2body-measurement}
    \end{figure}
\end{minipage}
\begin{minipage}{\textwidth}
    \begin{figure}[H]
        \centering\includegraphics[scale=0.55]{../mockup/4appointments_4appointment-details_3mealplan.png}
        \caption{Prototyp interfejsu - 4appointments:4appointment-details:3mealplan (opr.wł)}\label{rysunek:4appointments_4appointment-details_3mealplan}
    \end{figure}
\end{minipage}
\begin{minipage}{\textwidth}
    \begin{figure}[H]
        \centering\includegraphics[scale=0.55]{../mockup/5administration.png}
        \caption{Prototyp interfejsu - 5administration (opr.wł)}\label{rysunek:5administration}
    \end{figure}
\end{minipage}

\section{Kategorie}\label{sec:categories}
\todo{uzupełnić kategorie}

\begin{enumerate}[label={\textbf{KAT/\protect\threedigits{\theenumi}}}, wide, labelwidth=!, labelindent=0pt, labelsep=0pt, series=reqs]
    \setlength\itemsep{1em}
    \req{User} \label{kat:User} (Użytkownik)

    \textbf{Opis}: Konto użytkownika aplikacji. Każdy zalogowany użytkownik musi mieć konto użytkownika
    \par
    \textbf{Atrybuty}:
    \begin{itemize}[series=atr, wide, align=left, leftmargin=190pt]
        \atr{id} \label{kat:User:id} - identyfikator
        \atr{login} \label{kat:User:login} - login użytkownika
        \atr{passwordHash} \label{kat:User:passwordHash} - reprezentacja hasła stworzona przez nałożenie na hasło funkcji skrótu
        \atr{firstName} \label{kat:User:firstName} - imię użytkownika
        \atr{lastName} \label{kat:User:lastName} - nazwisko użytkownika
        \atr{email} \label{kat:User:email} - adres email
        \atr{image} \label{kat:User:image} - zdjęcie profilowe użytkownika
        \atr{activated} \label{kat:User:activated} - flaga pokazująca czy konto użytkownika zostało aktywowane
        \atr{language} \label{kat:User:language} - język użytkownika w postaci kodu ISO 639-1
        \atr{activationKey} \label{kat:User:activationKey} - klucz wymagany podczas aktywacji konta użytkownika
        \atr{resetKey} \label{kat:User:resetKey} - klucz wymagany podczas resetowania hasła do konta użytkownika
        \atr{createdDate} \label{kat:User:createdDate} - data utworzenia konta
        \atr{resetDate} \label{kat:User:resetDate} - data ostatniego resetowania hasła do konta
        \atr{lastModifiedDate} \label{kat:User:lastModifiedDate} - data ostatniej modyfikacji konta
    \end{itemize}

    \req{Authority} \label{kat:Authority} (Rola)

    \textbf{Opis}: Rola użytkownika od której zależy zakres uprawnień użytkownika
    \par
    \textbf{Atrybuty}:
    \begin{itemize}[series=atr, wide, align=left, leftmargin=190pt]
        \atr{name} \label{kat:Authority:name} - nazwa roli
    \end{itemize}

    \req{UserExtraInfo} \label{kat:UserExtraInfo} (Dodatkowe Informacje Użytkownika)

    \textbf{Opis}: Dodatkowe informacje o użytkowniku
    \par
    \textbf{Atrybuty}:
    \begin{itemize}[series=atr, wide, align=left, leftmargin=190pt]
        \atr{id} \label{kat:UserExtraInfo:id} - identyfikator
        \atr{gender} \label{kat:UserExtraInfo:gender} - płeć
        \atr{dateOfBirth} \label{kat:UserExtraInfo:dateOfBirth} - data urodzenia
        \atr{phoneNumber} \label{kat:UserExtraInfo:phoneNumber} - numer telefonu, najlepiej w formacie (+00) 000-000-000
        \atr{streetAddress} \label{kat:UserExtraInfo:streetAddress} - adres zamieszkania
        \atr{postalCode} \label{kat:UserExtraInfo:postalCode} - kod pocztowy
        \atr{city} \label{kat:UserExtraInfo:city} - miasto
        \atr{country} \label{kat:UserExtraInfo:country} - państwo
        \atr{personalDescription} \label{kat:UserExtraInfo:personalDescription} - krótki opis osobisty. W przypadku dietyka może zawierać dodatkowe informacje o prowadzonej praktyce dietetycznej
    \end{itemize}

    \req{SiteContent} \label{kat:SiteContent} (Treść Strony)

    \textbf{Opis}: Treść strony definiowana przez administratora
    \par
    \textbf{Atrybuty}:
    \begin{itemize}[series=atr, wide, align=left, leftmargin=190pt]
        \atr{id} \label{kat:SiteContent:id} - identyfikator
        \atr{ordinalNumber} \label{kat:SiteContent:ordinalNumber} - numer porządkowy definiujący kolejność w jakim dane powinny być wyświetlane
        \atr{siteContentType} \label{kat:SiteContent:siteContentType} - typ treści strony
        \atr{title} \label{kat:SiteContent:title} - tytuł treści stron
        \atr{description} \label{kat:SiteContent:description} - opis treści strony
        \atr{image} \label{kat:SiteContent:image} - opcjonalny obrazek, który powinien zostać wyświetlony obok treści strony
    \end{itemize}

    \req{SiteContentTranslation} \label{kat:SiteContentTranslation} (Tłumaczenie Treści Strony)

    \textbf{Opis}: Tłumaczenie treści strony
    \par
    \textbf{Atrybuty}:
    \begin{itemize}[series=atr, wide, align=left, leftmargin=190pt]
        \atr{id} \label{kat:SiteContentTranslation:id} - identyfikator
        \atr{title} \label{kat:SiteContentTranslation:title} - tłumaczenie tytułu
        \atr{description} \label{kat:SiteContentTranslation:description} - tłumaczenie opisu
        \atr{language} \label{kat:SiteContentTranslation:language} - język tłumaczenia w postaci kodu ISO 639-1
    \end{itemize}

    \req{ContactInfo} \label{kat:ContactInfo} (Informacje Kontaktowe)

    \textbf{Opis}: Informacje kontaktowe witryny
    \par
    \textbf{Atrybuty}:
    \begin{itemize}[series=atr, wide, align=left, leftmargin=190pt]
        \atr{id} \label{kat:ContactInfo:id} - identyfikator
        \atr{contactInfoType} \label{kat:ContactInfo:contactInfoType} - typ informacji kontaktowej
        \atr{description} \label{kat:ContactInfo:description} - opis
    \end{itemize}

    \req{Pricing} \label{kat:Pricing} (Cennik)

    \textbf{Opis}: Pozycja cennika dostępu do usług oferowanych przez system
    \par
    \textbf{Atrybuty}:
    \begin{itemize}[series=atr, wide, align=left, leftmargin=190pt]
        \atr{id} \label{kat:Pricing:id} - identyfikator
        \atr{ordinalNumber} \label{kat:Pricing:ordinalNumber} - numer porządkowy warunkujący kolejność wyswietlania pozycji w cenniku
        \atr{title} \label{kat:Pricing:title} - nazwa pozycji cennika
        \atr{description} \label{kat:Pricing:description} - opis pozycji cennika
        \atr{price} \label{kat:Pricing:price} - cena
        \atr{currency} \label{kat:Pricing:currency} - waluta przedstawiona jako kod ISO 4217
    \end{itemize}

    \req{PricingTranslation} \label{kat:PricingTranslation} (Tłumaczenie Cennika)

    \textbf{Opis}: Tłumaczenie pozycji cennika
    \par
    \textbf{Atrybuty}:
    \begin{itemize}[series=atr, wide, align=left, leftmargin=190pt]
        \atr{id} \label{kat:PricingTranslation:id} - identyfikator
        \atr{title} \label{kat:PricingTranslation:title} - tłumaczenie nazwy
        \atr{description} \label{kat:PricingTranslation:description} - tłumaczenie opisu
        \atr{language} \label{kat:PricingTranslation:language} - język tłumaczenia w postaci kodu ISO 639-1
    \end{itemize}

    \req{Product} \label{kat:Product} (Produkt)

    \textbf{Opis}: A food product. Data initially retrieved form USDA Standard Reference database
    \par
    \textbf{Atrybuty}:
    \begin{itemize}[series=atr, wide, align=left, leftmargin=190pt]
        \atr{id} \label{kat:Product:id} - identyfikator
        \atr{source} \label{kat:Product:source} - Specifying source if product is imported, preferably url address if possible
        \atr{isPublic} \label{kat:Product:isPublic} - Flag specifying if product is public
        \atr{language} \label{kat:Product:language} - Language tag of a product as ISO 639-1 code
    \end{itemize}

    \req{ProductVersion} \label{kat:ProductVersion} (Wersja Produktu)

    \textbf{Opis}: A version of food product. Data initially retrieved form USDA Standard Reference database.
    \par
    \textbf{Atrybuty}:
    \begin{itemize}[series=atr, wide, align=left, leftmargin=190pt]
        \atr{id} \label{kat:ProductVersion:id} - identyfikator
        \atr{createdDate} \label{kat:ProductVersion:createdDate} - Timestamp of version creation
        \atr{description} \label{kat:ProductVersion:description} - Short description of Product in a language of a product
    \end{itemize}

    \req{ProductBasicNutritionData} \label{kat:ProductBasicNutritionData} (Podstawowe Składniki Odżywcze Produktu)

    \textbf{Opis}: Basic nutrition data
    \par
    \textbf{Atrybuty}:
    \begin{itemize}[series=atr, wide, align=left, leftmargin=190pt]
        \atr{id} \label{kat:ProductBasicNutritionData:id} - identyfikator
        \atr{energy} \label{kat:ProductBasicNutritionData:energy} - Energy in kcal per 100 gram of product
        \atr{protein} \label{kat:ProductBasicNutritionData:protein} - Protein in grams per 100 gram of product
        \atr{fat} \label{kat:ProductBasicNutritionData:fat} - Fat in grams per 100 gram of product
        \atr{carbohydrates} \label{kat:ProductBasicNutritionData:carbohydrates} - Carbohydrates in grams per 100 gram of product
    \end{itemize}

    \req{NutritionData} \label{kat:NutritionData} (Wartość Odżywcza)

    \textbf{Opis}: A value of nutrition definition for concrete Product. Data initially retrieved form USDA Standard Reference database.
    \par
    \textbf{Atrybuty}:
    \begin{itemize}[series=atr, wide, align=left, leftmargin=190pt]
        \atr{id} \label{kat:NutritionData:id} - identyfikator
        \atr{nutritionValue} \label{kat:NutritionData:nutritionValue} - Nutrition value in units specified in NutritionDefinition
    \end{itemize}

    \req{NutritionDefinition} \label{kat:NutritionDefinition} (Definicja Wartości Odżywczej)

    \textbf{Opis}: A definition of nutrition. Data retrieved form USDA Standard Reference database. Data set is not planned to be expanded.
    \par
    \textbf{Atrybuty}:
    \begin{itemize}[series=atr, wide, align=left, leftmargin=190pt]
        \atr{id} \label{kat:NutritionDefinition:id} - identyfikator
        \atr{tag} \label{kat:NutritionDefinition:tag} - Short tag name of nutrient
        \atr{description} \label{kat:NutritionDefinition:description} - Short description of nutrient in English
        \atr{units} \label{kat:NutritionDefinition:units} - Unit used for nutrient measurement, e.g. "g", "kcal", "ml"
        \atr{decimalPlaces} \label{kat:NutritionDefinition:decimalPlaces} - Decimal places to which nutrient value should be rounded
    \end{itemize}

    \req{NutritionDefinitionTranslation} \label{kat:NutritionDefinitionTranslation} (Tłumaczenie Definicji Wartości Odżywczej)

    \textbf{Opis}: Nutrition definition translation
    \par
    \textbf{Atrybuty}:
    \begin{itemize}[series=atr, wide, align=left, leftmargin=190pt]
        \atr{id} \label{kat:NutritionDefinitionTranslation:id} - identyfikator
        \atr{translation} \label{kat:NutritionDefinitionTranslation:translation} - Translated description of nutrition definition
        \atr{language} \label{kat:NutritionDefinitionTranslation:language} - Language of translation as ISO 639-1 code
    \end{itemize}

    \req{HouseholdMeasure} \label{kat:HouseholdMeasure} (Miara Domowa)

    \textbf{Opis}: A household measures of product with weight in grams. Data initially retrieved form USDA Standard Reference database.
    \par
    \textbf{Atrybuty}:
    \begin{itemize}[series=atr, wide, align=left, leftmargin=190pt]
        \atr{id} \label{kat:HouseholdMeasure:id} - identyfikator
        \atr{description} \label{kat:HouseholdMeasure:description} - Short description of measure in language of a product, e.g. "cup" or "tea spoon"
        \atr{gramsWeight} \label{kat:HouseholdMeasure:gramsWeight} - Grams weight of 1 unit of specified measure
        \atr{isVisible} \label{kat:HouseholdMeasure:isVisible} - Flag specifying if measure is visible on presentation layer. By default it is initially set to false for data imported from external sources
    \end{itemize}

    \req{ProductSubcategory} \label{kat:ProductSubcategory} (Podkategoria Produktu)

    \textbf{Opis}: A subcategory of product. Data initially retrieved form USDA Standard Reference database
    \par
    \textbf{Atrybuty}:
    \begin{itemize}[series=atr, wide, align=left, leftmargin=190pt]
        \atr{id} \label{kat:ProductSubcategory:id} - identyfikator
        \atr{description} \label{kat:ProductSubcategory:description} - Short description of subcategory in language of a product
    \end{itemize}

    \req{ProductCategory} \label{kat:ProductCategory} (Kategoria Produktu)

    \textbf{Opis}: A main category of product. Data initially retrieved form USDA Standard Reference database
    \par
    \textbf{Atrybuty}:
    \begin{itemize}[series=atr, wide, align=left, leftmargin=190pt]
        \atr{id} \label{kat:ProductCategory:id} - identyfikator
        \atr{description} \label{kat:ProductCategory:description} - Short description of category in English
    \end{itemize}

    \req{ProductCategoryTranslation} \label{kat:ProductCategoryTranslation} (Tłumaczenie Kategorii Produktu)

    \textbf{Opis}: Product category translation
    \par
    \textbf{Atrybuty}:
    \begin{itemize}[series=atr, wide, align=left, leftmargin=190pt]
        \atr{id} \label{kat:ProductCategoryTranslation:id} - identyfikator
        \atr{translation} \label{kat:ProductCategoryTranslation:translation} - Translated name of product category
        \atr{language} \label{kat:ProductCategoryTranslation:language} - Language of translation as ISO 639-1 code
    \end{itemize}

    \req{DietType} \label{kat:DietType} (Typ Diety)

    \textbf{Opis}: A tag specifying characteristic feature of object to which it is applied
    \par
    \textbf{Atrybuty}:
    \begin{itemize}[series=atr, wide, align=left, leftmargin=190pt]
        \atr{id} \label{kat:DietType:id} - identyfikator
        \atr{name} \label{kat:DietType:name} - Short description of diet type in English
    \end{itemize}

    \req{DietTypeTranslation} \label{kat:DietTypeTranslation} (Tłumaczenie Typu Diety)

    \textbf{Opis}: Diet type translation
    \par
    \textbf{Atrybuty}:
    \begin{itemize}[series=atr, wide, align=left, leftmargin=190pt]
        \atr{id} \label{kat:DietTypeTranslation:id} - identyfikator
        \atr{translation} \label{kat:DietTypeTranslation:translation} - Translated name of diet type
        \atr{language} \label{kat:DietTypeTranslation:language} - Language of translation as ISO 639-1 code
    \end{itemize}

    \req{Recipe} \label{kat:Recipe} (Przepis)

    \textbf{Opis}: A recipe
    \par
    \textbf{Atrybuty}:
    \begin{itemize}[series=atr, wide, align=left, leftmargin=190pt]
        \atr{id} \label{kat:Recipe:id} - identyfikator
        \atr{isPublic} \label{kat:Recipe:isPublic} - Flag specifying if recipe is publicly visible
        \atr{language} \label{kat:Recipe:language} - Language tag of a recipe as ISO 639-1 code
    \end{itemize}

    \req{RecipeVersion} \label{kat:RecipeVersion} (Wersja Przepisu)

    \textbf{Opis}: A version of recipe
    \par
    \textbf{Atrybuty}:
    \begin{itemize}[series=atr, wide, align=left, leftmargin=190pt]
        \atr{id} \label{kat:RecipeVersion:id} - identyfikator
        \atr{editTimestamp} \label{kat:RecipeVersion:editTimestamp} - Timestamp of version creation
        \atr{name} \label{kat:RecipeVersion:name} - Name of recipe in language of recipe
        \atr{preparationTimeMinutes} \label{kat:RecipeVersion:preparationTimeMinutes} - Average time needed for overall recipe preparation, defined in minutes
        \atr{numberOfPortions} \label{kat:RecipeVersion:numberOfPortions} - Number of portions for which all quantities are specified
        \atr{image} \label{kat:RecipeVersion:image} - Optional image of recipe
        \atr{totalGramsWeight} \label{kat:RecipeVersion:totalGramsWeight} - Total weight in grams of meal prepared from recipe
    \end{itemize}

    \req{RecipeBasicNutritionData} \label{kat:RecipeBasicNutritionData} (Podstawowe Wartości Odżywcze Przepisu)

    \textbf{Opis}: Basic nutrition data
    \par
    \textbf{Atrybuty}:
    \begin{itemize}[series=atr, wide, align=left, leftmargin=190pt]
        \atr{id} \label{kat:RecipeBasicNutritionData:id} - identyfikator
        \atr{energy} \label{kat:RecipeBasicNutritionData:energy} - Energy in kcal per 100 gram of recipe meal calculated from products added to recipe
        \atr{protein} \label{kat:RecipeBasicNutritionData:protein} - Protein in grams per 100 gram of recipe meal calculated from products added to recipe
        \atr{fat} \label{kat:RecipeBasicNutritionData:fat} - Fat in grams per 100 gram of recipe meal calculated from products added to recipe
        \atr{carbohydrates} \label{kat:RecipeBasicNutritionData:carbohydrates} - Carbohydrates in grams per 100 gram of recipe meal calculated from products added to recipe
    \end{itemize}

    \req{RecipeSection} \label{kat:RecipeSection} (Sekcja Przepisu)

    \textbf{Opis}: A recipe section. e.g. recipe for cheesecake might have 3 separate sections for dough, filling and topping.
    \par
    \textbf{Atrybuty}:
    \begin{itemize}[series=atr, wide, align=left, leftmargin=190pt]
        \atr{id} \label{kat:RecipeSection:id} - identyfikator
        \atr{sectionName} \label{kat:RecipeSection:sectionName} - Name of recipe section in language of a recipe
    \end{itemize}

    \req{ProductPortion} \label{kat:ProductPortion} (Porcja Produktu)

    \textbf{Opis}: A portion of product used in recipe
    \par
    \textbf{Atrybuty}:
    \begin{itemize}[series=atr, wide, align=left, leftmargin=190pt]
        \atr{id} \label{kat:ProductPortion:id} - identyfikator
        \atr{amount} \label{kat:ProductPortion:amount} - Amount of product in household measure units. If household measure is null then amount is in grams
    \end{itemize}

    \req{PreparationStep} \label{kat:PreparationStep} (Krok Przygotowania)

    \textbf{Opis}: A preparation step in recipe
    \par
    \textbf{Atrybuty}:
    \begin{itemize}[series=atr, wide, align=left, leftmargin=190pt]
        \atr{id} \label{kat:PreparationStep:id} - identyfikator
        \atr{ordinalNumber} \label{kat:PreparationStep:ordinalNumber} - Ordinal number of preparation step
        \atr{stepDescription} \label{kat:PreparationStep:stepDescription} - Preferably short step description
    \end{itemize}

    \req{KitchenAppliance} \label{kat:KitchenAppliance} (Sprzęt Kuchenny)

    \textbf{Opis}: Kitchen appliance definition
    \par
    \textbf{Atrybuty}:
    \begin{itemize}[series=atr, wide, align=left, leftmargin=190pt]
        \atr{id} \label{kat:KitchenAppliance:id} - identyfikator
        \atr{name} \label{kat:KitchenAppliance:name} - English name of kitchen appliance
    \end{itemize}

    \req{KitchenApplianceTranslation} \label{kat:KitchenApplianceTranslation} (Tłumaczenie Sprzętu Kuchennego)

    \textbf{Opis}: Kitchen appliance translation
    \par
    \textbf{Atrybuty}:
    \begin{itemize}[series=atr, wide, align=left, leftmargin=190pt]
        \atr{id} \label{kat:KitchenApplianceTranslation:id} - identyfikator
        \atr{translation} \label{kat:KitchenApplianceTranslation:translation} - Translated name of kitchen appliance
        \atr{language} \label{kat:KitchenApplianceTranslation:language} - Language of translation as ISO 639-1 code
    \end{itemize}

    \req{DishType} \label{kat:DishType} (Typ Dania)

    \textbf{Opis}: A dish type, e.g. salad or soup
    \par
    \textbf{Atrybuty}:
    \begin{itemize}[series=atr, wide, align=left, leftmargin=190pt]
        \atr{id} \label{kat:DishType:id} - identyfikator
        \atr{description} \label{kat:DishType:description} - English description of dish type
    \end{itemize}

    \req{DishTypeTranslation} \label{kat:DishTypeTranslation} (Tłumaczenie Typu Dania)

    \textbf{Opis}: Dish type translation
    \par
    \textbf{Atrybuty}:
    \begin{itemize}[series=atr, wide, align=left, leftmargin=190pt]
        \atr{id} \label{kat:DishTypeTranslation:id} - identyfikator
        \atr{translation} \label{kat:DishTypeTranslation:translation} - Translated name of dish type
        \atr{language} \label{kat:DishTypeTranslation:language} - Language of translation as ISO 639-1 code
    \end{itemize}

    \req{MealType} \label{kat:MealType} (Typ Posiłku)

    \textbf{Opis}: A meal type, e.g. breakfast or dinner
    \par
    \textbf{Atrybuty}:
    \begin{itemize}[series=atr, wide, align=left, leftmargin=190pt]
        \atr{id} \label{kat:MealType:id} - identyfikator
        \atr{name} \label{kat:MealType:name} - English name of meal type
    \end{itemize}

    \req{MealTypeTranslation} \label{kat:MealTypeTranslation} (Tłumaczenie Typu Posiłku)

    \textbf{Opis}: Meal type translation
    \par
    \textbf{Atrybuty}:
    \begin{itemize}[series=atr, wide, align=left, leftmargin=190pt]
        \atr{id} \label{kat:MealTypeTranslation:id} - identyfikator
        \atr{translation} \label{kat:MealTypeTranslation:translation} - Translated name of meal type
        \atr{language} \label{kat:MealTypeTranslation:language} - Language of translation as ISO 639-1 code
    \end{itemize}


    \req{MealPlan} \label{kat:MealPlan} (Jadłospis)

    \textbf{Opis}: A Meal plan
    \par
    \textbf{Atrybuty}:
    \begin{itemize}[series=atr, wide, align=left, leftmargin=190pt]
        \atr{id} \label{kat:MealPlan:id} - identyfikator
        \atr{creationTimestamp} \label{kat:MealPlan:creationTimestamp} - Creation date of the plan
        \atr{editTimestamp} \label{kat:MealPlan:editTimestamp} - Last edit date of the plan
        \atr{name} \label{kat:MealPlan:name} - Plan name
        \atr{isVisible} \label{kat:MealPlan:isVisible} - Flag specifying if meal plan is visible in author's list of meal plans
        \atr{language} \label{kat:MealPlan:language} - Language tag of a meal plan as ISO 639-1 code
        \atr{numberOfDays} \label{kat:MealPlan:numberOfDays} - Number of days in plan
        \atr{numberOfMealsPerDay} \label{kat:MealPlan:numberOfMealsPerDay} - Number of meals per day
        \atr{totalDailyEnergy} \label{kat:MealPlan:totalDailyEnergy} - Amount of total energy per day in kcal
        \atr{percentOfProtein} \label{kat:MealPlan:percentOfProtein} - Percent of proteins in total daily energy
        \atr{percentOfFat} \label{kat:MealPlan:percentOfFat} - Percent of fats in total daily energy
        \atr{percentOfCarbohydrates} \label{kat:MealPlan:percentOfCarbohydrates} - Percent of carbohydrates in total daily energy
    \end{itemize}

    \req{MealPlanDay} \label{kat:MealPlanDay} (Dzień Jadłospisu)

    \textbf{Opis}: A Day in meal plan
    \par
    \textbf{Atrybuty}:
    \begin{itemize}[series=atr, wide, align=left, leftmargin=190pt]
        \atr{id} \label{kat:MealPlanDay:id} - identyfikator
        \atr{ordinalNumber} \label{kat:MealPlanDay:ordinalNumber} - Ordinal number of day
    \end{itemize}

    \req{Meal} \label{kat:Meal} (Posiłek)

    \textbf{Opis}: A Meal
    \par
    \textbf{Atrybuty}:
    \begin{itemize}[series=atr, wide, align=left, leftmargin=190pt]
        \atr{id} \label{kat:Meal:id} - identyfikator
        \atr{ordinalNumber} \label{kat:Meal:ordinalNumber} - Ordinal number of meal
    \end{itemize}

    \req{MealRecipe} \label{kat:MealRecipe} (Przepis Posiłku)

    \textbf{Opis}: A Recipe assigned to a meal
    \par
    \textbf{Atrybuty}:
    \begin{itemize}[series=atr, wide, align=left, leftmargin=190pt]
        \atr{id} \label{kat:MealRecipe:id} - identyfikator
        \atr{amount} \label{kat:MealRecipe:amount} - Amount of recipe in grams
    \end{itemize}

    \req{MealProduct} \label{kat:MealProduct} (Produkt Posiłku)

    \textbf{Opis}: A Product assigned to a meal
    \par
    \textbf{Atrybuty}:
    \begin{itemize}[series=atr, wide, align=left, leftmargin=190pt]
        \atr{id} \label{kat:MealProduct:id} - identyfikator
        \atr{amount} \label{kat:MealProduct:amount} - Amount of Product in household measure units. If household measure is null then amount is in grams
    \end{itemize}

    \req{MealDefinition} \label{kat:MealDefinition} (Definicja Posiłku)

    \textbf{Opis}: A Meal Definition used for specifying basic properties of each daily meal
    \par
    \textbf{Atrybuty}:
    \begin{itemize}[series=atr, wide, align=left, leftmargin=190pt]
        \atr{id} \label{kat:MealDefinition:id} - identyfikator
        \atr{ordinalNumber} \label{kat:MealDefinition:ordinalNumber} - Daily ordinal number of meal
        \atr{timeOfMeal} \label{kat:MealDefinition:timeOfMeal} - Usual time of meal in 24h format: HH:mm
        \atr{percentOfEnergy} \label{kat:MealDefinition:percentOfEnergy} - Part of daily total energy expressed in percent
    \end{itemize}

    \req{Appointment} \label{kat:Appointment} (Wizyta)

    \textbf{Opis}: An appointment
    \par
    \textbf{Atrybuty}:
    \begin{itemize}[series=atr, wide, align=left, leftmargin=190pt]
        \atr{id} \label{kat:Appointment:id} - identyfikator
        \atr{appointmentDate} \label{kat:Appointment:appointmentDate} - Date and time of the appointment
        \atr{appointmentState} \label{kat:Appointment:appointmentState} - Current appointment state
        \atr{generalAdvice} \label{kat:Appointment:generalAdvice} - General advice after appointment
    \end{itemize}

    \req{PatientCard} \label{kat:PatientCard} (Karta Pacjenta)

    \textbf{Opis}: A Patient's card
    \par
    \textbf{Atrybuty}:
    \begin{itemize}[series=atr, wide, align=left, leftmargin=190pt]
        \atr{id} \label{kat:PatientCard:id} - identyfikator
        \atr{creationDate} \label{kat:PatientCard:creationDate} - Date when patient registered to dietitian
    \end{itemize}

    \req{AppointmentEvaluation} \label{kat:AppointmentEvaluation} (Ewaluacja Wizyty)

    \textbf{Opis}: Patient's appointment evaluation
    \par
    \textbf{Atrybuty}:
    \begin{itemize}[series=atr, wide, align=left, leftmargin=190pt]
        \atr{id} \label{kat:AppointmentEvaluation:id} - identyfikator
        \atr{overallSatisfaction} \label{kat:AppointmentEvaluation:overallSatisfaction} - Overall visit satisfaction
        \atr{dietitianServiceSatisfaction} \label{kat:AppointmentEvaluation:dietitianServiceSatisfaction} - Dietitian service satisfaction
        \atr{mealPlanOverallSatisfaction} \label{kat:AppointmentEvaluation:mealPlanOverallSatisfaction} - Overall meal plan satisfaction
        \atr{mealCostSatisfaction} \label{kat:AppointmentEvaluation:mealCostSatisfaction} - Meals cost satisfaction
        \atr{mealPreparationTimeSatisfaction} \label{kat:AppointmentEvaluation:mealPreparationTimeSatisfaction} - Meals preparation time satisfaction
        \atr{mealComplexityLevelSatisfaction} \label{kat:AppointmentEvaluation:mealComplexityLevelSatisfaction} - Meal complexity level satisfaction
        \atr{mealTastefulnessSatisfaction} \label{kat:AppointmentEvaluation:mealTastefulnessSatisfaction} - Meal tastefulness satisfaction
        \atr{dietaryResultSatisfaction} \label{kat:AppointmentEvaluation:dietaryResultSatisfaction} - Dietary result satisfaction
        \atr{comment} \label{kat:AppointmentEvaluation:comment} - Optional comment to visit
    \end{itemize}

    \req{BodyMeasurement} \label{kat:BodyMeasurement} (Pomiar Ciała)

    \textbf{Opis}: A body measurement
    \par
    \textbf{Atrybuty}:
    \begin{itemize}[series=atr, wide, align=left, leftmargin=190pt]
        \atr{id} \label{kat:BodyMeasurement:id} - identyfikator
        \atr{completionDate} \label{kat:BodyMeasurement:completionDate} - Date of measurement completion
        \atr{height} \label{kat:BodyMeasurement:height} - Patient's height. Alongside with weight it is used to calculate BMI factor
        \atr{weight} \label{kat:BodyMeasurement:weight} - Patient's weight. Alongside with height it is used to calculate BMI factor
        \atr{waist} \label{kat:BodyMeasurement:waist} - Patient's waist measure
        \atr{percentOfFatTissue} \label{kat:BodyMeasurement:percentOfFatTissue} - percent of fat tissue in patient's body. Norm for women: 16-20. Norm for men: 15-18
        \atr{percentOfWater} \label{kat:BodyMeasurement:percentOfWater} - Percent of water in patient's body. Norm for women: 45-60. Norm for men: 50-65
        \atr{muscleMass} \label{kat:BodyMeasurement:muscleMass} - Mass of patient's muscle tissue in kilograms
        \atr{physicalMark} \label{kat:BodyMeasurement:physicalMark} - Physical mark. Norm: 5
        \atr{calciumInBones} \label{kat:BodyMeasurement:calciumInBones} - Level of calcium in patient's bones in kilograms. Norm: ~2.4kg
        \atr{basicMetabolism} \label{kat:BodyMeasurement:basicMetabolism} - Basic metabolism in kcal
        \atr{metabolicAge} \label{kat:BodyMeasurement:metabolicAge} - Metabolic age in years
        \atr{visceralFatLevel} \label{kat:BodyMeasurement:visceralFatLevel} - Level of visceral fat. Norm: 1-12
    \end{itemize}

    \req{NutritionalInterview} \label{kat:NutritionalInterview} (Wywiad Żywieniowy)

    \textbf{Opis}: A nutritional interview
    \par
    \textbf{Atrybuty}:
    \begin{itemize}[series=atr, wide, align=left, leftmargin=190pt]
        \atr{id} \label{kat:NutritionalInterview:id} - identyfikator
        \atr{completionDate} \label{kat:NutritionalInterview:completionDate} - Timestamp of interview completion
        \atr{targetWeight} \label{kat:NutritionalInterview:targetWeight} - Patient's target weight in kilograms
        \atr{advicePurpose} \label{kat:NutritionalInterview:advicePurpose} - Advice purpose summarising what patient wish to accomplish with diet
        \atr{physicalActivity} \label{kat:NutritionalInterview:physicalActivity} - Patient's usual daily activity level
        \atr{diseases} \label{kat:NutritionalInterview:diseases} - Patient's diseases
        \atr{medicines} \label{kat:NutritionalInterview:medicines} - Patient's medicines
        \atr{jobType} \label{kat:NutritionalInterview:jobType} - Patient's job type
        \atr{likedProducts} \label{kat:NutritionalInterview:likedProducts} - Food products that patient likes
        \atr{dislikedProducts} \label{kat:NutritionalInterview:dislikedProducts} - Food products that patient dislikes
        \atr{foodAllergies} \label{kat:NutritionalInterview:foodAllergies} - Food products that patient is allergic to
        \atr{foodIntolerances} \label{kat:NutritionalInterview:foodIntolerances} - Patient's food intolerances
    \end{itemize}

    \req{CustomNutritionalInterviewQuestion} \label{kat:CustomNutritionalInterviewQuestion} (Niestandardowe Pytanie Wywiadu Żywieniowego)

    \textbf{Opis}: A custom nutritional interview question
    \par
    \textbf{Atrybuty}:
    \begin{itemize}[series=atr, wide, align=left, leftmargin=190pt]
        \atr{id} \label{kat:CustomNutritionalInterviewQuestion:id} - identyfikator
        \atr{ordinalNumber} \label{kat:CustomNutritionalInterviewQuestion:ordinalNumber} - Ordinal number of custom question
        \atr{question} \label{kat:CustomNutritionalInterviewQuestion:question} - Custom question extending Nutritional Interview
        \atr{answer} \label{kat:CustomNutritionalInterviewQuestion:answer} - Answer for question
    \end{itemize}

    \req{CustomNutritionalInterviewQuestionTemplate} \label{kat:CustomNutritionalInterviewQuestionTemplate} (Szablon Niestandardowego Pytania Wywiadu Żywieniowego)

    \textbf{Opis}: A template for custom nutritional interview question
    \par
    \textbf{Atrybuty}:
    \begin{itemize}[series=atr, wide, align=left, leftmargin=190pt]
        \atr{id} \label{kat:CustomNutritionalInterviewQuestionTemplate:id} - identyfikator
        \atr{question} \label{kat:CustomNutritionalInterviewQuestionTemplate:question} - Custom question extending Nutritional Interview
        \atr{language} \label{kat:CustomNutritionalInterviewQuestionTemplate:language} - Language of translation as ISO 639-1 code
    \end{itemize}

    \req{AssignedMealPlan} \label{kat:AssignedMealPlan} (Przypisany Jadłospis)

    \textbf{Opis}: An assigned meal plan
    \par
    \textbf{Atrybuty}:
    \begin{itemize}[series=atr, wide, align=left, leftmargin=190pt]
        \atr{id} \label{kat:AssignedMealPlan:id} - identyfikator
        \atr{assigmentTime} \label{kat:AssignedMealPlan:assigmentTime} - Time of assigment
    \end{itemize}

\end{enumerate}

\section {Reguły funkcjonowania}\label{sec:functionalRules}
\todo{uzupełnić uprawnienia użytkowników w regułach}

\begin{itemize}[label={\textbf{Reguły dla}}, wide, labelwidth=!, labelindent=0pt]
    \setlength\itemsep{1em}
    \item[\textbf{Reguły}] \textbf{ogólne}
    \begin{enumerate}[label={\textbf{REG/\protect\threedigits{\arabic{enumi}}}}, wide, labelwidth=!, align=left, leftmargin=3cm]
        \item Przedmiot kompozycji podlega takim samym zasadom dostępu co właściciel kompozycji pod warunkiem, że przedmiot kompozycji nie definuje własnych reguł dopstepu
    \end{enumerate}
    \item\ref{kat:User}
    \begin{enumerate}[label={\textbf{REG/\protect\threedigits{\arabic{enumi}}}}, wide, labelwidth=!, align=left, leftmargin=3cm, resume]
        %Relacje
        \item Użytkownik (\ref{kat:User}) nie musi mieć musi mieć żadnych dodatkowych informacji (\ref{kat:UserExtraInfo})
        \item Użytkownik (\ref{kat:User}) może mieć maksymalnie jedne dodatkowe informacje (\ref{kat:UserExtraInfo})
        \item Użytkownik (\ref{kat:User}) musi mieć musi mieć przynajmniej jedną rolę (\ref{kat:Authority})
        \item Użytkownik (\ref{kat:User}) może mieć wiele ról (\ref{kat:Authority})
        \item Użytkownik (\ref{kat:User}) nie musi mieć autora (\ref{kat:User})
        \item Użytkownik (\ref{kat:User}) może mieć maksymalnie jednego autora (\ref{kat:User})
        \item Użytkownik (\ref{kat:User}) nie musi mieć ostatniego edytora (\ref{kat:User})
        \item Użytkownik (\ref{kat:User}) może mieć maksymalnie jednego ostatniego edytora (\ref{kat:User})
        %CRUD
        \item \role{Gość} może dodawać nowego użytkownika (\ref{kat:User})
        \item \role{Użytkownik} może wyświetlać, edytować i usuwać swoje dane użytkownika (\ref{kat:User})
        \item \role{Dietetyk} może wyświetlać podstawowe dane (\ref{kat:User}) \role{Pacjenta}, którego kartotekę prowadzi
        \item \role{Administrator} może wyświetlać i usuwać dane użytkownika (\ref{kat:User})
    \end{enumerate}
    \item\ref{kat:Authority}
    \begin{enumerate}[label={\textbf{REG/\protect\threedigits{\arabic{enumi}}}}, wide, labelwidth=!, align=left, leftmargin=3cm, resume]
        %Relacje
        %CRUD
        \item \role{Administrator} może dodawać, wyświetlać, edytować i usuwać dane roli (\ref{kat:Authority})
    \end{enumerate}
    \item\ref{kat:UserExtraInfo}
    \begin{enumerate}[label={\textbf{REG/\protect\threedigits{\arabic{enumi}}}}, wide, labelwidth=!, align=left, leftmargin=3cm, resume]
        %Relacje
        \item Dodatkowe informacje (\ref{kat:UserExtraInfo}) muszą być przypisane do dokładnie jednego użytkownika (\ref{kat:User})
        %CRUD
        \item Dodatkowe informacje (\ref{kat:UserExtraInfo}) są przedmiotem kompozycji ze strony użytkownika (\ref{kat:User})
    \end{enumerate}
    \item\ref{kat:SiteContent}
    \begin{enumerate}[label={\textbf{REG/\protect\threedigits{\arabic{enumi}}}}, wide, labelwidth=!, align=left, leftmargin=3cm, resume]
        %Relacje
        \item Treść strony (\ref{kat:SiteContent}) nie musi mieć żadnego tłumaczenia (\ref{kat:SiteContentTranslation})
        \item Treść strony (\ref{kat:SiteContent}) może mieć wiele tłumaczeń (\ref{kat:SiteContentTranslation})
        %CRUD
        \item \role{Gość} może wyświetlać dane treści strony (\ref{kat:SiteContent})
        \item \role{Użytkownik} może wyświetlać dane treści strony (\ref{kat:SiteContent})
        \item \role{Administrator} może dodawać, edytować i usuwać dane treści strony (\ref{kat:SiteContent})
    \end{enumerate}
    \item\ref{kat:SiteContentTranslation}
    \begin{enumerate}[label={\textbf{REG/\protect\threedigits{\arabic{enumi}}}}, wide, labelwidth=!, align=left, leftmargin=3cm, resume]
        %Relacje
        \item Tłumaczenie treści strony (\ref{kat:SiteContentTranslation}) musi być przypisane do dokładnie jednej treści strony (\ref{kat:SiteContent})
        %CRUD
        \item Tłumaczenie treści strony (\ref{kat:SiteContentTranslation}) jest przedmiotem kompozycji ze strony treści strony (\ref{kat:SiteContent})
    \end{enumerate}
    \item\ref{kat:ContactInfo}
    \begin{enumerate}[label={\textbf{REG/\protect\threedigits{\arabic{enumi}}}}, wide, labelwidth=!, align=left, leftmargin=3cm, resume]
        %CRUD
        \item \role{Gość} może wyświetlać dane informacji kontaktowych (\ref{kat:ContactInfo})
        \item \role{Użytkownik} może wyświetlać dane informacji kontaktowych (\ref{kat:ContactInfo})
        \item \role{Administrator} może dodawać, edytować i usuwać dane informacji kontaktowych (\ref{kat:ContactInfo})
    \end{enumerate}
    \item\ref{kat:Pricing}
    \begin{enumerate}[label={\textbf{REG/\protect\threedigits{\arabic{enumi}}}}, wide, labelwidth=!, align=left, leftmargin=3cm, resume]
        %Relacje
        \item Cennik (\ref{kat:Pricing}) nie musi mieć przypisanych żadnych tłumaczeń (\ref{kat:PricingTranslation})
        \item Cennik (\ref{kat:Pricing}) może mieć przypisane wiele tłumaczeń (\ref{kat:PricingTranslation})
        %CRUD
        \item \role{Gość} może wyświetlać dane cennika (\ref{kat:Pricing})
        \item \role{Użytkownik} może wyświetlać dane cennika (\ref{kat:Pricing})
        \item \role{Administrator} może dodawać, edytować i usuwać dane cennika (\ref{kat:Pricing})
    \end{enumerate}
    \item\ref{kat:PricingTranslation}
    \begin{enumerate}[label={\textbf{REG/\protect\threedigits{\arabic{enumi}}}}, wide, labelwidth=!, align=left, leftmargin=3cm, resume]
        %Relacje
        \item Tłumaczenie cennika (\ref{kat:PricingTranslation}) musi być przypisane do dokładnie jednego cennika (\ref{kat:Pricing})
        %CRUD
        \item Tłumaczenie cennika (\ref{kat:PricingTranslation}) jest przedmiotem kompozycji ze strony cennika (\ref{kat:Pricing})
    \end{enumerate}
    \item\ref{kat:Product}
    \begin{enumerate}[label={\textbf{REG/\protect\threedigits{\arabic{enumi}}}}, wide, labelwidth=!, align=left, leftmargin=3cm, resume]
        %Relacje
        \item Produkt (\ref{kat:Product}) musi mieć przynajmniej jedną wersję (\ref{kat:ProductVersion})
        \item Produkt (\ref{kat:Product}) może mieć wiele wersji (\ref{kat:ProductVersion})
        \item Produkt (\ref{kat:Product}) nie musi mieć zdefiniowanego autora (\ref{kat:User})
        \item Produkt (\ref{kat:Product}) może mieć maksymalnie jednego autora (\ref{kat:User})
        %CRUD
        \item todo
    \end{enumerate}
    \item\ref{kat:ProductVersion}
    \begin{enumerate}[label={\textbf{REG/\protect\threedigits{\arabic{enumi}}}}, wide, labelwidth=!, align=left, leftmargin=3cm, resume]
        %Relacje
        \item Wersja produktu (\ref{kat:ProductVersion}) musi być przypisana do dokładnie jednego produktu  (\ref{kat:Product})
        \item Wersja produktu (\ref{kat:ProductVersion}) musi być przypisana do dokładnie jednych podstawowych wartości odżywczych (\ref{kat:ProductBasicNutritionData})
        \item Wersja produktu (\ref{kat:ProductVersion}) nie musi mieć zdefiniowanych żadnych wartości odżywczych (\ref{kat:NutritionData})
        \item Wersja produktu (\ref{kat:ProductVersion}) może mieć zdefiniowane wiele wartości odżywczych (\ref{kat:NutritionData})
        \item Wersja produktu (\ref{kat:ProductVersion}) nie musi mieć zdefiniowanych żadnych miar domowych (\ref{kat:HouseholdMeasure})
        \item Wersja produktu (\ref{kat:ProductVersion}) może mieć zdefiniowane wiele miar domowych (\ref{kat:HouseholdMeasure})
        \item Wersja produktu (\ref{kat:ProductVersion}) musi należeć do dokładnie jednej podkategorii (\ref{kat:ProductSubcategory})
        \item Wersja produktu (\ref{kat:ProductVersion}) nie musi mieć przypisanego żadnego odpowiedniego typu diety (\ref{kat:DietType})
        \item Wersja produktu (\ref{kat:ProductVersion}) może mieć przypisanych wiele odpowiednich typów diety (\ref{kat:DietType})
        \item Wersja produktu (\ref{kat:ProductVersion}) nie musi mieć przypisanego żadnego nieodpowiedniego typu diety (\ref{kat:DietType})
        \item Wersja produktu (\ref{kat:ProductVersion}) może mieć przypisanych wiele nieodpowiednich typów diety (\ref{kat:DietType})
        %CRUD
        \item Wersja produktu (\ref{kat:ProductVersion}) jest przedmiotem kompozycji ze strony produktu (\ref{kat:Product})
    \end{enumerate}
    \item\ref{kat:ProductBasicNutritionData}
    \begin{enumerate}[label={\textbf{REG/\protect\threedigits{\arabic{enumi}}}}, wide, labelwidth=!, align=left, leftmargin=3cm, resume]
        %Relacje
        \item Podstawowe wartości odżywcze produktu(\ref{kat:ProductBasicNutritionData}) muszą być przypisane do dokladnie jednej wersji produktu (\ref{kat:ProductVersion})
        %CRUD
        \item Podstawowe wartości odżywcze produktu (\ref{kat:ProductBasicNutritionData}) są przedmiotem kompozycji ze strony wersji produktu (\ref{kat:ProductVersion})
    \end{enumerate}
    \item\ref{kat:NutritionData}
    \begin{enumerate}[label={\textbf{REG/\protect\threedigits{\arabic{enumi}}}}, wide, labelwidth=!, align=left, leftmargin=3cm, resume]
        %Relacje
        \item Wartość odżywcza (\ref{kat:NutritionData}) musi być przypisana do dokładnie jednej wersji produktu (\ref{kat:ProductVersion})
        \item Wartość odżywcza (\ref{kat:NutritionData}) musi być przypisana do dokładnie jednej definicji wartości odżywczej (\ref{kat:NutritionDefinition})
        %CRUD
        \item Wartość odżywcza (\ref{kat:NutritionData}) jest przedmiotem kompozycji ze strony wersji produktu (\ref{kat:ProductVersion})
    \end{enumerate}
    \item\ref{kat:NutritionDefinition}
    \begin{enumerate}[label={\textbf{REG/\protect\threedigits{\arabic{enumi}}}}, wide, labelwidth=!, align=left, leftmargin=3cm, resume]
        %Relacje
        \item Definicja wartości odżywczej (\ref{kat:NutritionDefinition}) nie musi mieć zdefiniowanego żadnego tłumaczenia (\ref{kat:NutritionDefinitionTranslation})
        \item Definicja wartości odżywczej (\ref{kat:NutritionDefinition}) może mieć zdefiniowanych wiele tłumaczeń (\ref{kat:NutritionDefinitionTranslation})
        %CRUD
        \item todo
    \end{enumerate}
    \item\ref{kat:NutritionDefinitionTranslation}
    \begin{enumerate}[label={\textbf{REG/\protect\threedigits{\arabic{enumi}}}}, wide, labelwidth=!, align=left, leftmargin=3cm, resume]
        %Relacje
        \item Tłumaczenie definicji wartości odżywczej (\ref{kat:NutritionDefinitionTranslation}) musi być przypisane do dokładnie jednej definicji wartości odżywczej  (\ref{kat:NutritionDefinition})
        %CRUD
        \item Tłumaczenie definicji wartości odżywczej (\ref{kat:NutritionDefinitionTranslation}) jest przedmiotem kompozycji ze strony definicji wartości odżywczej (\ref{kat:NutritionDefinition})
    \end{enumerate}
    \item\ref{kat:HouseholdMeasure}
    \begin{enumerate}[label={\textbf{REG/\protect\threedigits{\arabic{enumi}}}}, wide, labelwidth=!, align=left, leftmargin=3cm, resume]
        %Relacje
        \item Miara domowa (\ref{kat:HouseholdMeasure}) musi być przypisana do dokładnie jednej wersji produktu (\ref{kat:ProductVersion})
        %CRUD
        \item Miara domowa (\ref{kat:HouseholdMeasure}) jest przedmiotem kompozycji ze strony wersji produktu (\ref{kat:ProductVersion})
    \end{enumerate}
    \item\ref{kat:ProductSubcategory}
    \begin{enumerate}[label={\textbf{REG/\protect\threedigits{\arabic{enumi}}}}, wide, labelwidth=!, align=left, leftmargin=3cm, resume]
        %Relacje
        \item Podkategoria produktu (\ref{kat:ProductSubcategory}) musi być przypisana do conajmniej jednej wersji produktu (\ref{kat:ProductVersion})
        \item Podkategoria produktu (\ref{kat:ProductSubcategory}) może być przypisana do wielu wersji produktu (\ref{kat:ProductVersion})
        \item Podktagoria produktu (\ref{kat:ProductSubcategory}) musi być przypisana do dokładnie jednej kategorii (\ref{kat:ProductCategory})
        %CRUD
        \item todo
    \end{enumerate}
    \item\ref{kat:ProductCategory}
    \begin{enumerate}[label={\textbf{REG/\protect\threedigits{\arabic{enumi}}}}, wide, labelwidth=!, align=left, leftmargin=3cm, resume]
        %Relacje
        \item Kategoria produktu (\ref{kat:ProductCategory}) nie musi mieć przypisanego żadnego tłumaczenia (\ref{kat:ProductCategoryTranslation})
        \item Kategoria produktu (\ref{kat:ProductCategory}) może mieć przypisanych wiele tłumaczeń (\ref{kat:ProductCategoryTranslation})
        %CRUD
        \item todo
    \end{enumerate}
    \item\ref{kat:ProductCategoryTranslation}
    \begin{enumerate}[label={\textbf{REG/\protect\threedigits{\arabic{enumi}}}}, wide, labelwidth=!, align=left, leftmargin=3cm, resume]
        %Relacje
        \item Tłumaczenie kategorii produktu (\ref{kat:ProductCategoryTranslation}) musi być przypisane do dokładnie jednej kategorii (\ref{kat:ProductCategory})
        %CRUD
        \item Tłumaczenie kategorii produktu (\ref{kat:ProductCategoryTranslation}) jest przedmiotem kompozycji ze strony kategorii (\ref{kat:ProductCategory})
    \end{enumerate}
    \item\ref{kat:DietType}
    \begin{enumerate}[label={\textbf{REG/\protect\threedigits{\arabic{enumi}}}}, wide, labelwidth=!, align=left, leftmargin=3cm, resume]
        %Relacje
        \item Typ diety (\ref{kat:DietType}) nie musi mieć zdefiniowanego żadnego tłumaczenia (\ref{kat:DietTypeTranslation})
        \item Typ diety (\ref{kat:DietType}) może mieć zdefiniowanych wiele tłumaczeń (\ref{kat:DietTypeTranslation})
        %CRUD
        \item todo
    \end{enumerate}
    \item\ref{kat:DietTypeTranslation}
    \begin{enumerate}[label={\textbf{REG/\protect\threedigits{\arabic{enumi}}}}, wide, labelwidth=!, align=left, leftmargin=3cm, resume]
        %Relacje
        \item Tłumaczenie typu diety (\ref{kat:DietTypeTranslation}) musi być przypisane do dokładnie jednego typu diety (\ref{kat:DietType})
        %CRUD
        \item Tłumaczenie typu diety (\ref{kat:DietTypeTranslation}) jest przedmiotem kompozycji ze strony typu diety (\ref{kat:DietType})
    \end{enumerate}
    \item\ref{kat:Recipe}
    \begin{enumerate}[label={\textbf{REG/\protect\threedigits{\arabic{enumi}}}}, wide, labelwidth=!, align=left, leftmargin=3cm, resume]
        %Relacje
        \item Przepis (\ref{kat:Recipe}) nie musi mieć zdefiniowanego żadnego przepisu źródłowego (\ref{kat:Recipe})
        \item Przepis (\ref{kat:Recipe}) może mieć zdefiniowany maksymalnie jeden przepis źródłowy (\ref{kat:Recipe})
        \item Przepis (\ref{kat:Recipe}) musi mieć przynajmniej jedną wersję (\ref{kat:RecipeVersion})
        \item Przepis (\ref{kat:Recipe}) może mieć wiele wersji (\ref{kat:RecipeVersion})
        \item Przepis (\ref{kat:Recipe}) nie musi mieć zdefiniowanego autora (\ref{kat:User})
        \item Przepis (\ref{kat:Recipe}) może mieć maksymalnie jednego autora (\ref{kat:User})
        %CRUD
        \item todo
    \end{enumerate}
    \item\ref{kat:RecipeVersion}
    \begin{enumerate}[label={\textbf{REG/\protect\threedigits{\arabic{enumi}}}}, wide, labelwidth=!, align=left, leftmargin=3cm, resume]
        %Relacje
        \item Wersja przepisu (\ref{kat:RecipeVersion}) musi mieć dokładnie jedne podstawowe wartości odżywcze przepisu (\ref{kat:RecipeBasicNutritionData})
        \item Wersja przepisu (\ref{kat:RecipeVersion}) musi mieć przynajmniej jedną sekcję (\ref{kat:RecipeSection})
        \item Wersja przepisu (\ref{kat:RecipeVersion}) może mieć wiele sekcji (\ref{kat:RecipeSection})
        \item Wersja przepisu (\ref{kat:RecipeVersion}) nie musi mieć przypisanego żadnego sprzętu kuchennego (\ref{kat:KitchenAppliance})
        \item Wersja przepisu (\ref{kat:RecipeVersion}) może mieć przypisanych wiele sprzętów kuchennych (\ref{kat:KitchenAppliance})
        \item Wersja przepisu (\ref{kat:RecipeVersion}) nie musi mieć przypisanego żadnego typu dania (\ref{kat:DishType})
        \item Wersja przepisu (\ref{kat:RecipeVersion}) może mieć przypisanych wiele typów dań (\ref{kat:DishType})
        \item Wersja przepisu (\ref{kat:RecipeVersion}) nie musi mieć przypisanego żadnego typu posiłku (\ref{kat:MealType})
        \item Wersja przepisu (\ref{kat:RecipeVersion}) może mieć przypisanych wiele typów posiłków (\ref{kat:MealType})
        \item Wersja przepisu (\ref{kat:RecipeVersion}) nie musi mieć przypisanego żadnego odpowiedniego typu diety (\ref{kat:DietType})
        \item Wersja przepisu (\ref{kat:RecipeVersion}) może mieć przypisanych wiele odpowiednich typów diety  (\ref{kat:DietType})
        \item Wersja przepisu (\ref{kat:RecipeVersion}) nie musi mieć przypisanego żadnego nieodpowiedniego typu diety (\ref{kat:DietType})
        \item Wersja przepisu (\ref{kat:RecipeVersion}) może mieć przypisanych wiele nieodpowiednich typów diety (\ref{kat:DietType})
        %CRUD
        \item Wersja przepisu (\ref{kat:RecipeVersion}) jest przedmiotem kompozycji ze strony przepisu (\ref{kat:Recipe})
    \end{enumerate}
    \item\ref{kat:RecipeBasicNutritionData}
    \begin{enumerate}[label={\textbf{REG/\protect\threedigits{\arabic{enumi}}}}, wide, labelwidth=!, align=left, leftmargin=3cm, resume]
        %Relacje
        \item Podstawowe wartości odżywcze przepisu (\ref{kat:RecipeBasicNutritionData}) muszą być przypisane do dokładnie jednej wersji przepisu (\ref{kat:RecipeVersion})
        %CRUD
        \item Podstawowe wartości odżywcze przepisu (\ref{kat:RecipeBasicNutritionData}) są przedmiotem kompozycji ze strony wersji przepisu (\ref{kat:RecipeVersion})
    \end{enumerate}
    \item\ref{kat:RecipeSection}
    \begin{enumerate}[label={\textbf{REG/\protect\threedigits{\arabic{enumi}}}}, wide, labelwidth=!, align=left, leftmargin=3cm, resume]
        %Relacje
        \item Sekcja przepisu (\ref{kat:RecipeSection}) musi być przypisana do dokładniej jednej wersji przepisu (\ref{kat:RecipeVersion})
        \item Sekcja przepisu (\ref{kat:RecipeSection}) musi mieć przypisaną przynajmniej jedną porcję produktu (\ref{kat:ProductPortion})
        \item Sekcja przepisu (\ref{kat:RecipeSection}) może mieć przypisanych wiele porcji produktu (\ref{kat:ProductPortion})
        \item Sekcja przepisu (\ref{kat:RecipeSection}) musi mieć przypisany przynajmniej jeden krok przygotowania (\ref{kat:PreparationStep})
        \item Sekcja przepisu (\ref{kat:RecipeSection}) może mieć zdefiniowanych wiele kroków przygotowania (\ref{kat:PreparationStep})
        %CRUD
        \item Sekcja przepisu (\ref{kat:RecipeSection}) jest przedmiotem kompozycji ze strony wersji przepisu (\ref{kat:RecipeVersion})
    \end{enumerate}
    \item\ref{kat:ProductPortion}
    \begin{enumerate}[label={\textbf{REG/\protect\threedigits{\arabic{enumi}}}}, wide, labelwidth=!, align=left, leftmargin=3cm, resume]
        %Relacje
        \item Porcja produktu (\ref{kat:ProductPortion}) musi być przypisana do dokładnie jednej sekcji przepisu (\ref{kat:RecipeSection})
        \item Porcja produktu (\ref{kat:ProductPortion}) musi mieć przypisany dokładnie jeden produkt (\ref{kat:Product})
        \item Porcja produktu (\ref{kat:ProductPortion}) nie musi mieć przypisanej miary domowej (\ref{kat:HouseholdMeasure})
        \item Porcja produktu (\ref{kat:ProductPortion}) może mieć przypisaną maksymalnie jedną miarę domową (\ref{kat:HouseholdMeasure})
        %CRUD
        \item Porcja produktu (\ref{kat:ProductPortion}) jest przedmiotem kompozycji ze strony sekcji przepisu (\ref{kat:RecipeSection})
    \end{enumerate}
    \item\ref{kat:PreparationStep}
    \begin{enumerate}[label={\textbf{REG/\protect\threedigits{\arabic{enumi}}}}, wide, labelwidth=!, align=left, leftmargin=3cm, resume]
        %Relacje
        \item Krok przygotowania (\ref{kat:PreparationStep}) musi być przypisany do dokładnie jednej sekcji przepisu (\ref{kat:RecipeSection})
        %CRUD
        \item Krok przygotowania (\ref{kat:PreparationStep}) jest przedmiotem kompozycji ze strony sekcji przepisu (\ref{kat:RecipeSection})
    \end{enumerate}
    \item\ref{kat:KitchenAppliance}
    \begin{enumerate}[label={\textbf{REG/\protect\threedigits{\arabic{enumi}}}}, wide, labelwidth=!, align=left, leftmargin=3cm, resume]
        %Relacje
        \item Sprzęt kuchenny (\ref{kat:KitchenAppliance}) nie musi mieć zdefiniowanego żadnego tłumaczenia (\ref{kat:KitchenApplianceTranslation})
        \item Sprzęt kuchenny (\ref{kat:KitchenAppliance}) może mieć zdefiniowanych wiele tłumaczeń (\ref{kat:KitchenApplianceTranslation})
        %CRUD
        \item todo
    \end{enumerate}
    \item\ref{kat:KitchenApplianceTranslation}
    \begin{enumerate}[label={\textbf{REG/\protect\threedigits{\arabic{enumi}}}}, wide, labelwidth=!, align=left, leftmargin=3cm, resume]
        %Relacje
        \item Tłumaczenie sprzetu kuchennego (\ref{kat:KitchenApplianceTranslation}) musi być przypisane do dokaldnie jednego sprzetu kuchennego (\ref{kat:KitchenAppliance})
        %CRUD
        \item Tłumaczenie sprzętu kuchennego (\ref{kat:KitchenApplianceTranslation}) jest przedmiotem kompozycji ze strony sprzętu kuchennego (\ref{kat:KitchenAppliance})
    \end{enumerate}
    \item\ref{kat:DishType}
    \begin{enumerate}[label={\textbf{REG/\protect\threedigits{\arabic{enumi}}}}, wide, labelwidth=!, align=left, leftmargin=3cm, resume]
        %Relacje
        \item Typ dania (\ref{kat:DishType}) nie musi mieć zdefiniowanego żadnego tłumaczenia (\ref{kat:DishTypeTranslation})
        \item Typ dania (\ref{kat:DishType}) może mieć zdefiniowanych wiele tłumaczeń (\ref{kat:DishTypeTranslation})
        %CRUD
        \item todo
    \end{enumerate}
    \item\ref{kat:DishTypeTranslation}
    \begin{enumerate}[label={\textbf{REG/\protect\threedigits{\arabic{enumi}}}}, wide, labelwidth=!, align=left, leftmargin=3cm, resume]
        %Relacje
        \item Tłumaczenie typu dania (\ref{kat:DishTypeTranslation}) musi być przypisane do dokładnie jednego typu dania (\ref{kat:DishType})
        %CRUD
        \item Tłumaczenie typu dania (\ref{kat:DishTypeTranslation}) jest przedmiotem kompozycji ze strony typu dania (\ref{kat:DishType})
    \end{enumerate}
    \item\ref{kat:MealType}
    \begin{enumerate}[label={\textbf{REG/\protect\threedigits{\arabic{enumi}}}}, wide, labelwidth=!, align=left, leftmargin=3cm, resume]
        %Relacje
        \item Typ posiłku (\ref{kat:MealType}) nie musi mieć zdefiniowanego żadnego tłumaczenia (\ref{kat:MealTypeTranslation})
        \item Typ posiłku (\ref{kat:MealType}) może mieć zdefiniowanych wiele tłumaczeń (\ref{kat:MealTypeTranslation})
        %CRUD
        \item todo
    \end{enumerate}
    \item\ref{kat:MealTypeTranslation}
    \begin{enumerate}[label={\textbf{REG/\protect\threedigits{\arabic{enumi}}}}, wide, labelwidth=!, align=left, leftmargin=3cm, resume]
        %Relacje
        \item Tłumaczenie typu posiłku (\ref{kat:MealTypeTranslation}) musi być przypisane do dokładnie jednego typu posiłku (\ref{kat:MealType})
        %CRUD
        \item Tłumaczenie typu posiłku (\ref{kat:MealTypeTranslation}) jest przedmiotem kompozycji ze strony typu posiłku (\ref{kat:MealType})
    \end{enumerate}
    \item\ref{kat:MealPlan}
    \begin{enumerate}[label={\textbf{REG/\protect\threedigits{\arabic{enumi}}}}, wide, labelwidth=!, align=left, leftmargin=3cm, resume]
        %Relacje
        \item Jadłospis (\ref{kat:MealPlan}) musi mieć przypisany przynajmniej jeden dzień (\ref{kat:MealPlanDay})
        \item Jadłospis (\ref{kat:MealPlan}) może mieć przypisanych maksymalnie 31 dni (\ref{kat:MealPlanDay})
        \item Jadłospis (\ref{kat:MealPlan}) musi mieć przypisaną przynajmniej jedną definicję posiłku (\ref{kat:MealDefinition})
        \item Jadłospis (\ref{kat:MealPlan}) może mieć przypisanych maksymalnie 10 definicji posiłków (\ref{kat:MealDefinition})
        \item Jadłospis (\ref{kat:MealPlan}) nie musi mieć przypisanego żadnego odpowiedniego typu diety (\ref{kat:DietType})
        \item Jadłospis (\ref{kat:MealPlan}) może mieć przypisanych wiele odpowiednich typów diety (\ref{kat:DietType})
        \item Jadłospis (\ref{kat:MealPlan}) nie musi mieć przypisanego żadnego nieodpowiedniego typu diety (\ref{kat:DietType})
        \item Jadłospis (\ref{kat:MealPlan}) może mieć przypisanych wiele nieodpowiednich typów diety (\ref{kat:DietType})
        \item Jadłospis (\ref{kat:MealPlan}) musi mieć dokładnie jednego autora (\ref{kat:User})
        %CRUD
        \item todo
    \end{enumerate}
    \item\ref{kat:MealPlanDay}
    \begin{enumerate}[label={\textbf{REG/\protect\threedigits{\arabic{enumi}}}}, wide, labelwidth=!, align=left, leftmargin=3cm, resume]
        %Relacje
        \item Dzień jadłospisu (\ref{kat:MealPlanDay}) musi być przypisany do dokładnie jednego jadłospisu (\ref{kat:MealPlan})
        \item Dzień jadłospisu (\ref{kat:MealPlanDay}) nie musi mieć przypisanego żadnego posiłku (\ref{kat:Meal})
        \item Dzień jadłospisu (\ref{kat:MealPlanDay}) może mieć przypisanych maksymalnie 10 posiłków (\ref{kat:Meal})
        %CRUD
        \item Dzień jadłospisu (\ref{kat:MealPlanDay}) jest przedmiotem kompozycji ze strony jadłospisu (\ref{kat:MealPlan})
    \end{enumerate}
    \item\ref{kat:Meal}
    \begin{enumerate}[label={\textbf{REG/\protect\threedigits{\arabic{enumi}}}}, wide, labelwidth=!, align=left, leftmargin=3cm, resume]
        %Relacje
        \item Posiłek (\ref{kat:Meal}) musi być przypisany do dokładnie jednego dnia jadłospisu (\ref{kat:MealPlanDay})
        \item Posiłek (\ref{kat:Meal}) nie musi mieć przypisanego żadnego produktu (\ref{kat:MealProduct})
        \item Posiłek (\ref{kat:Meal}) może mieć przypisanych wiele produktów (\ref{kat:MealProduct})
        \item Posiłek (\ref{kat:Meal}) nie musi mieć przypisanego żadnego przepisu (\ref{kat:MealRecipe})
        \item Posiłek (\ref{kat:Meal}) może mieć przypisanych wiele przepisów (\ref{kat:MealRecipe})
        %CRUD
        \item Posiłek (\ref{kat:Meal}) jest przedmiotem kompozycji ze strony dnia jadłospisu (\ref{kat:MealPlanDay})
    \end{enumerate}
    \item\ref{kat:MealRecipe}
    \begin{enumerate}[label={\textbf{REG/\protect\threedigits{\arabic{enumi}}}}, wide, labelwidth=!, align=left, leftmargin=3cm, resume]
        %Relacje
        \item Przepis posiłku (\ref{kat:MealRecipe}) musi być przypisany do dokladnie jednego posiłku (\ref{kat:Meal})
        \item Przepis posiłku (\ref{kat:MealRecipe}) musi mieć przypisany dokładnie jeden przepis (\ref{kat:Recipe})
        %CRUD
        \item Przepis posiłku (\ref{kat:MealRecipe}) jest przedmiotem kompozycji ze strony posiłku (\ref{kat:Meal})
    \end{enumerate}
    \item\ref{kat:MealProduct}
    \begin{enumerate}[label={\textbf{REG/\protect\threedigits{\arabic{enumi}}}}, wide, labelwidth=!, align=left, leftmargin=3cm, resume]
        %Relacje
        \item Produkt posiłku (\ref{kat:MealProduct}) musi być przypisany do dokładnie jednego posiłku (\ref{kat:Meal})
        \item Produkt posiłku (\ref{kat:MealProduct}) musi mieć przypisany dokładnie jeden produkt (\ref{kat:Product})
        \item Produkt posiłku (\ref{kat:MealProduct}) nie musi mieć przypisanej żadnej miary domowej (\ref{kat:HouseholdMeasure})
        \item Produkt posiłku (\ref{kat:MealProduct}) musi mieć przypisaną maksymalnie jedną miarę domową (\ref{kat:HouseholdMeasure})
        %CRUD
        \item Produkt posiłku (\ref{kat:MealProduct}) jest przedmiotem kompozycji ze strony posiłku (\ref{kat:Meal})
    \end{enumerate}
    \item\ref{kat:MealDefinition}
    \begin{enumerate}[label={\textbf{REG/\protect\threedigits{\arabic{enumi}}}}, wide, labelwidth=!, align=left, leftmargin=3cm, resume]
        %Relacje
        \item Definicja posiłku (\ref{kat:MealDefinition}) musi być przypisana do dokładnie jednego jadłospisu (\ref{kat:MealPlan})
        \item Definicja posiłku (\ref{kat:MealDefinition}) musi mieć przypisany dokładnie jeden typ posiłku (\ref{kat:MealType})
        %CRUD
        \item Definicja posiłku (\ref{kat:MealDefinition}) jest przedmiotem kompozycji ze strony jadłospisu (\ref{kat:MealPlan})
    \end{enumerate}
    \item\ref{kat:Appointment}
    \begin{enumerate}[label={\textbf{REG/\protect\threedigits{\arabic{enumi}}}}, wide, labelwidth=!, align=left, leftmargin=3cm, resume]
        %Relacje
        \item Wizyta (\ref{kat:Appointment}) musi być przypisana do dokładnie jednej karty pacjenta (\ref{kat:PatientCard})
        \item Wizyta (\ref{kat:Appointment}) nie musi mieć przypisanej żadnej ewaluacji (\ref{kat:AppointmentEvaluation})
        \item Wizyta (\ref{kat:Appointment}) może mieć przypisaną maksymalnie jedną ewaluację (\ref{kat:AppointmentEvaluation})
        \item Wizyta (\ref{kat:Appointment}) nie musi mieć przypisanego żadnych pomiarów ciała (\ref{kat:BodyMeasurement})
        \item Wizyta (\ref{kat:Appointment}) może mieć przypisane maksymalnie jedne pomiary ciała (\ref{kat:BodyMeasurement})
        \item Wizyta (\ref{kat:Appointment}) nie musi mieć przypisanego żadnego wywiadu żywieniowego (\ref{kat:NutritionalInterview})
        \item Wizyta (\ref{kat:Appointment}) może mieć przypisany maksymalnie jeden wywiad żywieniowy (\ref{kat:NutritionalInterview})
        \item Wizyta (\ref{kat:Appointment}) nie musi mieć przypisanego żadnego jadłospisu (\ref{kat:AssignedMealPlan})
        \item Wizyta (\ref{kat:Appointment}) może mieć przypisanych wiele jadłospisów (\ref{kat:AssignedMealPlan})
        %CRUD
        \item todo
    \end{enumerate}
    \item\ref{kat:PatientCard}
    \begin{enumerate}[label={\textbf{REG/\protect\threedigits{\arabic{enumi}}}}, wide, labelwidth=!, align=left, leftmargin=3cm, resume]
        %Relacje
        \item Karta pacjenta (\ref{kat:PatientCard}) nie musi mieć przypisanej żadnej wizyty (\ref{kat:Appointment})
        \item Karta pacjenta (\ref{kat:PatientCard}) może mieć przypisanych wiele wizyt (\ref{kat:Appointment})
        \item Karta pacjenta (\ref{kat:PatientCard}) musi mieć przypisanego dokładnie jednego pacjenta (\ref{kat:User})
        \item Karta pacjenta (\ref{kat:PatientCard}) musi mieć przypisanego dokładnie jednego dietetyka (\ref{kat:User})
        %CRUD
        \item todo
    \end{enumerate}
    \item\ref{kat:AppointmentEvaluation}
    \begin{enumerate}[label={\textbf{REG/\protect\threedigits{\arabic{enumi}}}}, wide, labelwidth=!, align=left, leftmargin=3cm, resume]
        %Relacje
        \item Ewaluacja wizyty (\ref{kat:AppointmentEvaluation}) musi być przypisana do dokładnie jednej wizyty (\ref{kat:Appointment})
        %CRUD
        \item todo
    \end{enumerate}
    \item\ref{kat:BodyMeasurement}
    \begin{enumerate}[label={\textbf{REG/\protect\threedigits{\arabic{enumi}}}}, wide, labelwidth=!, align=left, leftmargin=3cm, resume]
        %Relacje
        \item Pomiary ciała (\ref{kat:BodyMeasurement}) muszą być przypisane do dokładnie jednej wizyty (\ref{kat:Appointment})
        %CRUD
        \item Pomiary ciała (\ref{kat:BodyMeasurement}) są przedmiotem kompozycji ze strony wizyty (\ref{kat:Appointment})
    \end{enumerate}
    \item\ref{kat:NutritionalInterview}
    \begin{enumerate}[label={\textbf{REG/\protect\threedigits{\arabic{enumi}}}}, wide, labelwidth=!, align=left, leftmargin=3cm, resume]
        %Relacje
        \item Wywiad żywieniowy (\ref{kat:NutritionalInterview}) musi być przypisany do dokładnie jednej wizyty (\ref{kat:Appointment})
        \item Wywiad żywieniowy (\ref{kat:NutritionalInterview}) nie musi mieć przypisanego żadnego niestandardowego pytania (\ref{kat:CustomNutritionalInterviewQuestion})
        \item Wywiad żywiniowy (\ref{kat:NutritionalInterview}) może mieć przypisanych wiele niestandardowych pytań (\ref{kat:CustomNutritionalInterviewQuestion})
        \item Wywiad żywieniowy (\ref{kat:NutritionalInterview}) nie musi mieć przypisanych żadnych posiadanych sprzętów kuchennych (\ref{kat:KitchenAppliance})
        \item Wywiad żywieniowy (\ref{kat:NutritionalInterview}) może mieć przypisanych wiele posiadanych sprzętów kuchennych (\ref{kat:KitchenAppliance})
        %CRUD
        \item Wywiad żywieniowy (\ref{kat:NutritionalInterview}) jest przedmiotem kompozycji ze strony wizyty (\ref{kat:Appointment})
    \end{enumerate}
    \item\ref{kat:CustomNutritionalInterviewQuestion}
    \begin{enumerate}[label={\textbf{REG/\protect\threedigits{\arabic{enumi}}}}, wide, labelwidth=!, align=left, leftmargin=3cm, resume]
        %Relacje
        \item Niestandardowe pytanie żywieniowe (\ref{kat:CustomNutritionalInterviewQuestion}) musi być przypisane do dokładnie jednego wywiadu żywieniowego (\ref{kat:NutritionalInterview})
        %CRUD
        \item  Niestandardowe pytanie żywieniowe (\ref{kat:CustomNutritionalInterviewQuestion}) jest przedmiotem kompozycji ze strony wywiadu żywieniowego (\ref{kat:NutritionalInterview})
    \end{enumerate}
    \item\ref{kat:CustomNutritionalInterviewQuestionTemplate}
    \begin{enumerate}[label={\textbf{REG/\protect\threedigits{\arabic{enumi}}}}, wide, labelwidth=!, align=left, leftmargin=3cm, resume]
        %Relacje
        \item Szablon niestandardowego pytania żywieniowego (\ref{kat:CustomNutritionalInterviewQuestionTemplate}) musi mieć dokaldnie jednego autora (\ref{kat:User})
        %CRUD
        \item todo
    \end{enumerate}
    \item\ref{kat:AssignedMealPlan}
    \begin{enumerate}[label={\textbf{REG/\protect\threedigits{\arabic{enumi}}}}, wide, labelwidth=!, align=left, leftmargin=3cm, resume]
        %Relacje
        \item Przypisany jadłospis (\ref{kat:AssignedMealPlan}) musi mieć przydzieloną dokładnie jedną wizytę (\ref{kat:Appointment})
        \item Przypisany jadłospis (\ref{kat:AssignedMealPlan}) musi mieć przydzielony dokładnie jeden jadłospis (\ref{kat:MealPlan})
        %CRUD
        \item todo
    \end{enumerate}
\end{itemize}

\section{Ograniczenia dziedzinowe}\label{sec:restrictions}

\begin{itemize}[label={\textbf{Ograniczenia dla}}, wide, labelwidth=!, labelindent=0pt]
    \setlength\itemsep{1em}
    \item[\textbf{Ograniczenia}] \textbf{ogólne}
    \begin{enumerate}[label={\textbf{OGR/\protect\threedigits{\arabic{enumi}}}}, wide, labelwidth=!, align=left, leftmargin=3cm]
        \item Wszystkie \textbf{id} muszą być być unikalne
        \item Wszystkie \textbf{id} są wymagane
        \item Wszystkie \textbf{id} są liczbami całkowitymi dodatnimi tworzonymi przez SZBD za pomocą autonumerowania
        \item Wszystkie atrybuty \textbf{language} są wymagane
        \item Wszystkie \textbf{language} są ciągami znaków o długości 2 znaków spełniającymi normę ISO 639-1
        \item Wszystkie \textbf{stemple czasowe} są w formacie YYYY:MM:DD HH:MI:SS
        \item Wszystkie \textbf{daty} są w formacie YYYY:MM:DD
        \item Ciągi znaków bez dodatkowych ograniczeń mogą zawierać dowolne znaki dopuszczalne w systemie kodowania UTF-8
    \end{enumerate}
    \item\ref{kat:User}
    \begin{enumerate}[label={\textbf{OGR/\protect\threedigits{\arabic{enumi}}}}, wide, labelwidth=!, align=left, leftmargin=3cm, resume]
        %Required
        \item Atrybut \ref{kat:User:login} jest wymagany
        \item Atrybut \ref{kat:User:passwordHash} jest wymagany
        \item Atrybut flagę \ref{kat:User:activated} jest wymagany
        \item Atrybut \ref{kat:User:createdDate} jest wymagany
        %Unique
        \item Atrybut \ref{kat:User:login} ma unikalną wartość
        \item Atrybut \ref{kat:User:email} ma unikalną wartość
        %Type
        \item Atrybut \ref{kat:User:login} jest ciągiem znaków składającym się z liter, cyfr i dodatkowo mogącym zawierać znaki ".", "\_", "-", "@" o długości od 1 do 50 znaków
        \item Atrybut \ref{kat:User:passwordHash} jest ciągiem znaków o długości 60 znaków
        \item Atrybut \ref{kat:User:firstName} jest ciagiem znaków o długości do 50 znaków
        \item Atrybut \ref{kat:User:lastName} jest ciagiem znaków o długości do 50 znaków
        \item Atrybut \ref{kat:User:email} jest ciagiem znaków o długości od 5 do 254 znaków
        \item Atrybut \ref{kat:User:activated} jest typem logicznym
        \item Atrybut \ref{kat:User:image} jest ciągiem znaków o długości do 256 znaków tworzącym poprawny adres URL
        \item Atrybut \ref{kat:User:activationKey} jest ciągiem znaków o długości 20 znaków
        \item Atrybut \ref{kat:User:resetKey} jest ciągiem znaków o długości 20 znaków
        \item Atrybut \ref{kat:User:resetDate} jest stemplem czasowym
        \item Atrybut \ref{kat:User:createdDate} jest stemplem czasowym
        \item Atrybut \ref{kat:User:lastModifiedDate} jest stemplem czasowym
    \end{enumerate}
    \item\ref{kat:Authority}
    \begin{enumerate}[label={\textbf{OGR/\protect\threedigits{\arabic{enumi}}}}, wide, labelwidth=!, align=left, leftmargin=3cm, resume]
        \item Atrybut \ref{kat:Authority:name} jest wymagany
        \item Atrybut \ref{kat:Authority:name} ma unikalną wartość
        \item Atrybut \ref{kat:Authority:name} jest ciągiem znaków składającym się z liter i znaków "\_" o długości od 1 do 255 znaków
    \end{enumerate}
    \item\ref{kat:UserExtraInfo}
    \begin{enumerate}[label={\textbf{OGR/\protect\threedigits{\arabic{enumi}}}}, wide, labelwidth=!, align=left, leftmargin=3cm, resume]
        \item Atrybut \ref{kat:UserExtraInfo:gender} jest typu wyliczeniowego i może przyjmować wartości "FEMALE", "MALE", "OTHER"
        \item Atrybut \ref{kat:UserExtraInfo:dateOfBirth} jest datą
        \item Atrybut \ref{kat:UserExtraInfo:phoneNumber} jest ciągiem znaków o długości od 1 do 50 znaków
        \item Atrybut \ref{kat:UserExtraInfo:streetAddress} jest ciągiem znaków o długości od 1 do 255 znaków
        \item Atrybut \ref{kat:UserExtraInfo:postalCode} jest ciągiem znaków o długości od 1 do 20 znaków
        \item Atrybut \ref{kat:UserExtraInfo:city} jest ciągiem znaków o długości od 1 do 50 znaków
        \item Atrybut \ref{kat:UserExtraInfo:country} jest ciągiem znaków o długości od 1 do 50 znaków
        \item Atrybut \ref{kat:UserExtraInfo:personalDescription} jest ciągiem znaków
    \end{enumerate}

    \item\ref{kat:SiteContent}
    \begin{enumerate}[label={\textbf{OGR/\protect\threedigits{\arabic{enumi}}}}, wide, labelwidth=!, align=left, leftmargin=3cm, resume]
        \item Atrybut \ref{kat:SiteContent:ordinalNumber} jest wymagany
        \item Atrybut \ref{kat:SiteContent:siteContentType} jest wymagany
        \item Atrybut \ref{kat:SiteContent:description} jest wymagany

        \item Atrybut \ref{kat:SiteContent:ordinalNumber} jest liczbą całkowitą
        \item Atrybut \ref{kat:SiteContent:siteContentType} jest typu wyliczeniowego i może przyjmować wartości "LANDING\_PAGE\_CARD", "TERMS\_OF\_SERVICE", "PRIVACY\_POLICY", "FREQUENTLY\_ASKED\_QUESTION"
        \item Atrybut \ref{kat:SiteContent:title} jest ciągiem znaków o długości od 1 do 255 znaków
        \item Atrybut \ref{kat:SiteContent:description} jest ciągiem znaków
        \item Atrybut \ref{kat:SiteContent:image} jest zdjęciem o maksymalnym rozmiarze 5000000 bajtów
    \end{enumerate}

    \item\ref{kat:SiteContentTranslation}
    \begin{enumerate}[label={\textbf{OGR/\protect\threedigits{\arabic{enumi}}}}, wide, labelwidth=!, align=left, leftmargin=3cm, resume]
        \item Atrybut \ref{kat:SiteContentTranslation:description} jest wymagany

        \item Atrybut \ref{kat:SiteContentTranslation:title} jest ciągiem znaków o długości od 1 do 255 znaków
        \item Atrybut \ref{kat:SiteContentTranslation:description} jest ciągiem znaków
    \end{enumerate}

    \item\ref{kat:ContactInfo}
    \begin{enumerate}[label={\textbf{OGR/\protect\threedigits{\arabic{enumi}}}}, wide, labelwidth=!, align=left, leftmargin=3cm, resume]
        \item Atrybut \ref{kat:ContactInfo:contactInfoType} jest wymagany
        \item Atrybut \ref{kat:ContactInfo:description} jest wymagany

        \item Atrybut \ref{kat:ContactInfo:contactInfoType} jest typu wyliczeniowego i może przyjmować wartości "PHONE", "EMAIL", "ADDRESS", "FACEBOOK", "TWITTER", "INSTAGRAM", "ANDROID", "IOS", "WORKING\_HOURS", "OTHER"
        \item Atrybut \ref{kat:ContactInfo:description} jest ciągiem znaków o długości od 1 do 255 znaków
    \end{enumerate}

    \item\ref{kat:Pricing}
    \begin{enumerate}[label={\textbf{OGR/\protect\threedigits{\arabic{enumi}}}}, wide, labelwidth=!, align=left, leftmargin=3cm, resume]
        \item Atrybut \ref{kat:Pricing:ordinalNumber} jest wymagany
        \item Atrybut \ref{kat:Pricing:description} jest wymagany
        \item Atrybut \ref{kat:Pricing:price} jest wymagany
        \item Atrybut \ref{kat:Pricing:currency} jest wymagany

        \item Atrybut \ref{kat:Pricing:ordinalNumber} jest liczbą całkowitą większą od 0
        \item Atrybut \ref{kat:Pricing:title} jest ciągiem znaków o długości od 1 do 255 znaków
        \item Atrybut \ref{kat:Pricing:description} jest ciągiem znaków
        \item Atrybut \ref{kat:Pricing:price} jest ciągiem znaków o długości od 1 do 8 znaków
        \item Atrybut \ref{kat:Pricing:currency} jest ciągiem znaków o długości 3 znaków spełniającym normę ISO 4217
        \item Jeżeli w \ref{kat:Pricing:price} występuje znak "." to muszą po nim wystepować dokładnie 2 cyfry
    \end{enumerate}

    \item\ref{kat:PricingTranslation}
    \begin{enumerate}[label={\textbf{OGR/\protect\threedigits{\arabic{enumi}}}}, wide, labelwidth=!, align=left, leftmargin=3cm, resume]
        \item Atrybut \ref{kat:PricingTranslation:description} jest wymagany

        \item Atrybut \ref{kat:PricingTranslation:title} jest ciągiem znaków o długości od 1 do 255 znaków
        \item Atrybut \ref{kat:PricingTranslation:description} jest ciągiem znaków
    \end{enumerate}

    \item\ref{kat:Product}
    \begin{enumerate}[label={\textbf{OGR/\protect\threedigits{\arabic{enumi}}}}, wide, labelwidth=!, align=left, leftmargin=3cm, resume]
        \item Atrybut \ref{kat:Product:isPublic} jest wymagany

        \item Atrybut \ref{kat:Product:source} jest ciągiem znaków o długości od 1 do 255 znaków
        \item Atrybut \ref{kat:Product:isPublic} jest typu logicznego
    \end{enumerate}

    \item\ref{kat:ProductVersion}
    \begin{enumerate}[label={\textbf{OGR/\protect\threedigits{\arabic{enumi}}}}, wide, labelwidth=!, align=left, leftmargin=3cm, resume]
        \item Atrybut \ref{kat:ProductVersion:createdDate} jest wymagany
        \item Atrybut \ref{kat:ProductVersion:description} jest wymagany

        \item Atrybut \ref{kat:ProductVersion:createdDate} jest stemplem czasowym
        \item Atrybut \ref{kat:ProductVersion:description} jest ciągiem znaków o długości od 1 do 255 znaków
    \end{enumerate}

    \item\ref{kat:ProductBasicNutritionData}
    \begin{enumerate}[label={\textbf{OGR/\protect\threedigits{\arabic{enumi}}}}, wide, labelwidth=!, align=left, leftmargin=3cm, resume]
        \item Atrybut \ref{kat:ProductBasicNutritionData:energy} jest wymagany
        \item Atrybut \ref{kat:ProductBasicNutritionData:protein} jest wymagany
        \item Atrybut \ref{kat:ProductBasicNutritionData:fat} jest wymagany
        \item Atrybut \ref{kat:ProductBasicNutritionData:carbohydrates} jest wymagany

        \item Atrybut \ref{kat:ProductBasicNutritionData:energy} jest liczbą rzeczywistą nie mniejszą niż 0
        \item Atrybut \ref{kat:ProductBasicNutritionData:protein} jest liczbą rzeczywistą nie mniejszą niż 0
        \item Atrybut \ref{kat:ProductBasicNutritionData:fat} jest liczbą rzeczywistą nie mniejszą niż 0
        \item Atrybut \ref{kat:ProductBasicNutritionData:carbohydrates} jest liczbą rzeczywistą nie mniejszą niż 0
    \end{enumerate}

    \item\ref{kat:NutritionData}
    \begin{enumerate}[label={\textbf{OGR/\protect\threedigits{\arabic{enumi}}}}, wide, labelwidth=!, align=left, leftmargin=3cm, resume]
        \item Atrybut \ref{kat:NutritionData:nutritionValue} jest wymagany

        \item Atrybut \ref{kat:NutritionData:nutritionValue} jest liczbą rzeczywistą nie mniejszą niż 0
    \end{enumerate}

    \item\ref{kat:NutritionDefinition}
    \begin{enumerate}[label={\textbf{OGR/\protect\threedigits{\arabic{enumi}}}}, wide, labelwidth=!, align=left, leftmargin=3cm, resume]
        \item Atrybut \ref{kat:NutritionDefinition:tag} jest wymagany
        \item Atrybut \ref{kat:NutritionDefinition:description} jest wymagany
        \item Atrybut \ref{kat:NutritionDefinition:units} jest wymagany
        \item Atrybut \ref{kat:NutritionDefinition:decimalPlaces} jest wymagany

        \item Atrybut \ref{kat:NutritionDefinition:tag} ma unikalną wartość

        \item Atrybut \ref{kat:NutritionDefinition:tag} jest ciągiem znaków o długości od 1 do 20 znaków
        \item Atrybut \ref{kat:NutritionDefinition:description} jest ciągiem znaków o długości od 1 do 255 znaków
        \item Atrybut \ref{kat:NutritionDefinition:units} jest ciągiem znaków o długości od 1 do 10 znaków
        \item Atrybut \ref{kat:NutritionDefinition:decimalPlaces} jest liczbą całkowitą nie mniejszą niż 0
    \end{enumerate}

    \item\ref{kat:NutritionDefinitionTranslation}
    \begin{enumerate}[label={\textbf{OGR/\protect\threedigits{\arabic{enumi}}}}, wide, labelwidth=!, align=left, leftmargin=3cm, resume]
        \item Atrybut \ref{kat:NutritionDefinitionTranslation:translation} jest wymagany

        \item Atrybut \ref{kat:NutritionDefinitionTranslation:translation} jest ciągiem znaków o długości od 1 do 255 znaków
    \end{enumerate}

    \item\ref{kat:HouseholdMeasure}
    \begin{enumerate}[label={\textbf{OGR/\protect\threedigits{\arabic{enumi}}}}, wide, labelwidth=!, align=left, leftmargin=3cm, resume]
        \item Atrybut \ref{kat:HouseholdMeasure:description} jest wymagany
        \item Atrybut \ref{kat:HouseholdMeasure:gramsWeight} jest wymagany
        \item Atrybut \ref{kat:HouseholdMeasure:isVisible} jest wymagany

        \item Atrybut \ref{kat:HouseholdMeasure:description} jest ciągiem znaków o długości od 1 do 255 znaków
        \item Atrybut \ref{kat:HouseholdMeasure:gramsWeight} jest liczbą rzeczywistą nie mniejszą niż 0
        \item Atrybut \ref{kat:HouseholdMeasure:isVisible} jest typu logicznego
    \end{enumerate}

    \item\ref{kat:ProductSubcategory}
    \begin{enumerate}[label={\textbf{OGR/\protect\threedigits{\arabic{enumi}}}}, wide, labelwidth=!, align=left, leftmargin=3cm, resume]
        \item Atrybut \ref{kat:ProductSubcategory:description} jest wymagany

        \item Atrybut \ref{kat:ProductSubcategory:description} jest ciągiem znaków o długości od 1 do 255 znaków
    \end{enumerate}

    \item\ref{kat:ProductCategory}
    \begin{enumerate}[label={\textbf{OGR/\protect\threedigits{\arabic{enumi}}}}, wide, labelwidth=!, align=left, leftmargin=3cm, resume]
        \item Atrybut \ref{kat:ProductCategory:description} jest wymagany

        \item Atrybut \ref{kat:ProductCategory:description} ma unikalną wartość

        \item Atrybut \ref{kat:ProductCategory:description} jest ciągiem znaków o długości od 1 do 255 znaków
    \end{enumerate}

    \item\ref{kat:ProductCategoryTranslation}
    \begin{enumerate}[label={\textbf{OGR/\protect\threedigits{\arabic{enumi}}}}, wide, labelwidth=!, align=left, leftmargin=3cm, resume]
        \item Atrybut \ref{kat:ProductCategoryTranslation:translation} jest wymagany

        \item Atrybut \ref{kat:ProductCategoryTranslation:translation} jest ciągiem znaków o długości od 1 do 255 znaków
    \end{enumerate}

    \item\ref{kat:DietType}
    \begin{enumerate}[label={\textbf{OGR/\protect\threedigits{\arabic{enumi}}}}, wide, labelwidth=!, align=left, leftmargin=3cm, resume]
        \item Atrybut \ref{kat:DietType:name} jest wymagany

        \item Atrybut \ref{kat:DietType:name} ma unikalną wartość

        \item Atrybut \ref{kat:DietType:name} jest ciągiem znaków o długości od 1 do 255 znaków
    \end{enumerate}

    \item\ref{kat:DietTypeTranslation}
    \begin{enumerate}[label={\textbf{OGR/\protect\threedigits{\arabic{enumi}}}}, wide, labelwidth=!, align=left, leftmargin=3cm, resume]
        \item Atrybut \ref{kat:DietTypeTranslation:translation} jest wymagany

        \item Atrybut \ref{kat:DietTypeTranslation:translation} jest ciągiem znaków o długości od 1 do 255 znaków
    \end{enumerate}

    \item\ref{kat:Recipe}
    \begin{enumerate}[label={\textbf{OGR/\protect\threedigits{\arabic{enumi}}}}, wide, labelwidth=!, align=left, leftmargin=3cm, resume]
        \item Atrybut \ref{kat:Recipe:isPublic} jest wymagany

        \item Atrybut \ref{kat:Recipe:isPublic} jest typu logicznego
    \end{enumerate}

    \item\ref{kat:RecipeVersion}
    \begin{enumerate}[label={\textbf{OGR/\protect\threedigits{\arabic{enumi}}}}, wide, labelwidth=!, align=left, leftmargin=3cm, resume]
        \item Atrybut \ref{kat:RecipeVersion:name} jest wymagany
        \item Atrybut \ref{kat:RecipeVersion:preparationTimeMinutes} jest wymagany
        \item Atrybut \ref{kat:RecipeVersion:numberOfPortions} jest wymagany
        \item Atrybut \ref{kat:RecipeVersion:totalGramsWeight} jest wymagany

        \item Atrybut \ref{kat:RecipeVersion:editTimestamp} jest stemplem czasowym
        \item Atrybut \ref{kat:RecipeVersion:name} jest ciągiem znaków o długości od 1 do 255 znaków
        \item Atrybut \ref{kat:RecipeVersion:preparationTimeMinutes} jest liczbą całkowitą nie mniejszą niż 0
        \item Atrybut \ref{kat:RecipeVersion:numberOfPortions} jest liczbą rzeczywistą nie mniejszą niż 0
        \item Atrybut \ref{kat:RecipeVersion:image} jest zdjęciem o maksymalnym rozmiarze 5000000 bajtów
        \item Atrybut \ref{kat:RecipeVersion:totalGramsWeight} jest liczbą rzeczywistą nie mniejszą niż 0
    \end{enumerate}

    \item\ref{kat:RecipeBasicNutritionData}
    \begin{enumerate}[label={\textbf{OGR/\protect\threedigits{\arabic{enumi}}}}, wide, labelwidth=!, align=left, leftmargin=3cm, resume]
        \item Atrybut \ref{kat:RecipeBasicNutritionData:energy} jest wymagany
        \item Atrybut \ref{kat:RecipeBasicNutritionData:protein} jest wymagany
        \item Atrybut \ref{kat:RecipeBasicNutritionData:fat} jest wymagany
        \item Atrybut \ref{kat:RecipeBasicNutritionData:carbohydrates} jest wymagany

        \item Atrybut \ref{kat:RecipeBasicNutritionData:energy} jest liczbą całkowitą nie mniejszą niż 0
        \item Atrybut \ref{kat:RecipeBasicNutritionData:protein} jest liczbą całkowitą nie mniejszą niż 0
        \item Atrybut \ref{kat:RecipeBasicNutritionData:fat} jest liczbą całkowitą nie mniejszą niż 0
        \item Atrybut \ref{kat:RecipeBasicNutritionData:carbohydrates} jest liczbą całkowitą nie mniejszą niż 0
    \end{enumerate}

    \item\ref{kat:RecipeSection}
    \begin{enumerate}[label={\textbf{OGR/\protect\threedigits{\arabic{enumi}}}}, wide, labelwidth=!, align=left, leftmargin=3cm, resume]
        \item Atrybut \ref{kat:RecipeSection:sectionName} jest ciągiem znaków o długości od 1 do 255 znaków
    \end{enumerate}

    \item\ref{kat:ProductPortion}
    \begin{enumerate}[label={\textbf{OGR/\protect\threedigits{\arabic{enumi}}}}, wide, labelwidth=!, align=left, leftmargin=3cm, resume]
        \item Atrybut \ref{kat:ProductPortion:amount} jest wymagany

        \item Atrybut \ref{kat:ProductPortion:amount} jest liczbą rzeczywistą nie mniejszą niż 0
    \end{enumerate}

    \item\ref{kat:PreparationStep}
    \begin{enumerate}[label={\textbf{OGR/\protect\threedigits{\arabic{enumi}}}}, wide, labelwidth=!, align=left, leftmargin=3cm, resume]
        \item Atrybut \ref{kat:PreparationStep:ordinalNumber} jest wymagany

        \item Atrybut \ref{kat:PreparationStep:ordinalNumber} jest liczbą całkowitą nie mniejszą niż 1
        \item Atrybut \ref{kat:PreparationStep:stepDescription} jest ciągiem znaków
    \end{enumerate}

    \item\ref{kat:KitchenAppliance}
    \begin{enumerate}[label={\textbf{OGR/\protect\threedigits{\arabic{enumi}}}}, wide, labelwidth=!, align=left, leftmargin=3cm, resume]
        \item Atrybut \ref{kat:KitchenAppliance:name} jest wymagany

        \item Atrybut \ref{kat:KitchenAppliance:name} ma unikalną wartość

        \item Atrybut \ref{kat:KitchenAppliance:name} jest ciągiem znaków o długości od 1 do 255 znaków
    \end{enumerate}

    \item\ref{kat:KitchenApplianceTranslation}
    \begin{enumerate}[label={\textbf{OGR/\protect\threedigits{\arabic{enumi}}}}, wide, labelwidth=!, align=left, leftmargin=3cm, resume]
        \item Atrybut \ref{kat:KitchenApplianceTranslation:translation} jest wymagany

        \item Atrybut \ref{kat:KitchenApplianceTranslation:translation} jest ciągiem znaków o długości od 1 do 255 znaków
    \end{enumerate}

    \item\ref{kat:DishType}
    \begin{enumerate}[label={\textbf{OGR/\protect\threedigits{\arabic{enumi}}}}, wide, labelwidth=!, align=left, leftmargin=3cm, resume]
        \item Atrybut \ref{kat:DishType:description} jest wymagany

        \item Atrybut \ref{kat:DishType:description} ma unikalną wartość

        \item Atrybut \ref{kat:DishType:description} jest ciągiem znaków o długości od 1 do 255 znaków
    \end{enumerate}

    \item\ref{kat:DishTypeTranslation}
    \begin{enumerate}[label={\textbf{OGR/\protect\threedigits{\arabic{enumi}}}}, wide, labelwidth=!, align=left, leftmargin=3cm, resume]
        \item Atrybut \ref{kat:DishTypeTranslation:translation} jest wymagany

        \item Atrybut \ref{kat:DishTypeTranslation:translation} jest ciągiem znaków o długości od 1 do 255 znaków
    \end{enumerate}

    \item\ref{kat:MealType}
    \begin{enumerate}[label={\textbf{OGR/\protect\threedigits{\arabic{enumi}}}}, wide, labelwidth=!, align=left, leftmargin=3cm, resume]
        \item Atrybut \ref{kat:MealType:name} jest wymagany

        \item Atrybut \ref{kat:MealType:name} ma unikalną wartość

        \item Atrybut \ref{kat:MealType:name} jest ciągiem znaków o długości od 1 do 255 znaków
    \end{enumerate}

    \item\ref{kat:MealTypeTranslation}
    \begin{enumerate}[label={\textbf{OGR/\protect\threedigits{\arabic{enumi}}}}, wide, labelwidth=!, align=left, leftmargin=3cm, resume]
        \item Atrybut \ref{kat:MealTypeTranslation:translation} jest wymagany

        \item Atrybut \ref{kat:MealTypeTranslation:translation} jest ciągiem znaków o długości od 1 do 255 znaków
    \end{enumerate}

    \item\ref{kat:MealPlan}
    \begin{enumerate}[label={\textbf{OGR/\protect\threedigits{\arabic{enumi}}}}, wide, labelwidth=!, align=left, leftmargin=3cm, resume]
        \item Atrybut \ref{kat:MealPlan:creationTimestamp} jest wymagany
        \item Atrybut \ref{kat:MealPlan:editTimestamp} jest wymagany
        \item Atrybut \ref{kat:MealPlan:isVisible} jest wymagany
        \item Atrybut \ref{kat:MealPlan:numberOfDays}  jest wymagany
        \item Atrybut \ref{kat:MealPlan:numberOfMealsPerDay} jest wymagany
        \item Atrybut \ref{kat:MealPlan:totalDailyEnergy} jest wymagany
        \item Atrybut \ref{kat:MealPlan:percentOfProtein} jest wymagany
        \item Atrybut \ref{kat:MealPlan:percentOfFat} jest wymagany
        \item Atrybut \ref{kat:MealPlan:percentOfCarbohydrates} jest wymagany

        \item Atrybut \ref{kat:MealPlan:creationTimestamp} jest stemplem czasowym
        \item Atrybut \ref{kat:MealPlan:editTimestamp} jest stemplem czasowym
        \item Atrybut \ref{kat:MealPlan:name} jest ciągiem znaków o długości od 1 do 255 znaków
        \item Atrybut \ref{kat:MealPlan:isVisible} jest typu logicznego
        \item Atrybut \ref{kat:MealPlan:numberOfDays} jest liczbą całkowitą nie mniejszą niż 1 i nie większą niż 30
        \item Atrybut \ref{kat:MealPlan:numberOfMealsPerDay} jest liczbą całkowitą nie mniejszą niż 1 i nie większą niż 10
        \item Atrybut \ref{kat:MealPlan:totalDailyEnergy} jest liczbą całkowitą nie mniejszą niż 1
        \item Atrybut \ref{kat:MealPlan:percentOfProtein} jest liczbą całkowitą nie mniejszą niż 0 i nie większą niż 100
        \item Atrybut \ref{kat:MealPlan:percentOfFat} jest liczbą całkowitą nie mniejszą niż 0 i nie większą niż 100
        \item Atrybut \ref{kat:MealPlan:percentOfCarbohydrates} jest liczbą całkowitą nie mniejszą niż 0 i nie większą niż 100
        \item Suma wartości atrybutów \ref{kat:MealPlan:percentOfProtein}, \ref{kat:MealPlan:percentOfFat}, \ref{kat:MealPlan:percentOfCarbohydrates} nie może przekraczać 100
    \end{enumerate}

    \item\ref{kat:MealPlanDay}
    \begin{enumerate}[label={\textbf{OGR/\protect\threedigits{\arabic{enumi}}}}, wide, labelwidth=!, align=left, leftmargin=3cm, resume]
        \item Atrybut \ref{kat:MealPlanDay:ordinalNumber} jest wymagany

        \item Atrybut \ref{kat:MealPlanDay:ordinalNumber} jest liczbą całkowitą nie mniejszą niż 1
    \end{enumerate}

    \item\ref{kat:Meal}
    \begin{enumerate}[label={\textbf{OGR/\protect\threedigits{\arabic{enumi}}}}, wide, labelwidth=!, align=left, leftmargin=3cm, resume]
        \item Atrybut \ref{kat:Meal:ordinalNumber} jest wymagany

        \item Atrybut \ref{kat:Meal:ordinalNumber} jest liczbą całkowitą nie mniejszą niż 1
    \end{enumerate}

    \item\ref{kat:MealRecipe}
    \begin{enumerate}[label={\textbf{OGR/\protect\threedigits{\arabic{enumi}}}}, wide, labelwidth=!, align=left, leftmargin=3cm, resume]
        \item Atrybut \ref{kat:MealRecipe:amount} jest wymagany

        \item Atrybut \ref{kat:MealRecipe:amount} jest liczbą całkowitą nie mniejszą niż 0
    \end{enumerate}

    \item\ref{kat:MealProduct}
    \begin{enumerate}[label={\textbf{OGR/\protect\threedigits{\arabic{enumi}}}}, wide, labelwidth=!, align=left, leftmargin=3cm, resume]
        \item Atrybut \ref{kat:MealProduct:amount} jest wymagany

        \item Atrybut \ref{kat:MealProduct:amount} jest liczbą rzeczywistą nie mniejszą niż 0
    \end{enumerate}

    \item\ref{kat:MealDefinition}
    \begin{enumerate}[label={\textbf{OGR/\protect\threedigits{\arabic{enumi}}}}, wide, labelwidth=!, align=left, leftmargin=3cm, resume]
        \item Atrybut \ref{kat:MealDefinition:ordinalNumber} jest wymagany
        \item Atrybut \ref{kat:MealDefinition:timeOfMeal} jest wymagany
        \item Atrybut \ref{kat:MealDefinition:percentOfEnergy} jest wymagany

        \item Atrybut \ref{kat:MealDefinition:ordinalNumber} jest liczbą całkowitą nie mniejszą niż 1
        \item Atrybut \ref{kat:MealDefinition:timeOfMeal} jest ciągiem znaków w postaci HH:MI
        \item Atrybut \ref{kat:MealDefinition:percentOfEnergy} jest liczbą całkowitą nie mniejszą niż 0 i nie większą niż 100
        \item Suma wartości wszystkich atrybutów \ref{kat:MealDefinition:percentOfEnergy} w jednym jadłospisie musi być równa 100
    \end{enumerate}

    \item\ref{kat:Appointment}
    \begin{enumerate}[label={\textbf{OGR/\protect\threedigits{\arabic{enumi}}}}, wide, labelwidth=!, align=left, leftmargin=3cm, resume]
        \item Atrybut \ref{kat:Appointment:appointmentDate} jest wymagany
        \item Atrybut \ref{kat:Appointment:appointmentState} jest wymagany

        \item Atrybut \ref{kat:Appointment:appointmentDate} jest stemplem czasowym
        \item Atrybut \ref{kat:Appointment:appointmentState} jest typu wyliczeniowego i może przyjmować wartości "PLANNED", "CANCELED", "TOOK\_PLACE", "COMPLETED"
        \item Atrybut \ref{kat:Appointment:generalAdvice} jest ciągiem znaków
    \end{enumerate}

    \item\ref{kat:PatientCard}
    \begin{enumerate}[label={\textbf{OGR/\protect\threedigits{\arabic{enumi}}}}, wide, labelwidth=!, align=left, leftmargin=3cm, resume]
        \item Atrybut \ref{kat:PatientCard:creationDate} jest wymagany

        \item Atrybut \ref{kat:PatientCard:creationDate} jest stemplem czasowym
    \end{enumerate}

    \item\ref{kat:AppointmentEvaluation}
    \begin{enumerate}[label={\textbf{OGR/\protect\threedigits{\arabic{enumi}}}}, wide, labelwidth=!, align=left, leftmargin=3cm, resume]
        \item Atrybut \ref{kat:AppointmentEvaluation:overallSatisfaction} jest wymagany
        \item Atrybut \ref{kat:AppointmentEvaluation:dietitianServiceSatisfaction} jest wymagany
        \item Atrybut \ref{kat:AppointmentEvaluation:mealPlanOverallSatisfaction} jest wymagany
        \item Atrybut \ref{kat:AppointmentEvaluation:mealCostSatisfaction} jest wymagany
        \item Atrybut \ref{kat:AppointmentEvaluation:mealPreparationTimeSatisfaction} jest wymagany
        \item Atrybut \ref{kat:AppointmentEvaluation:mealComplexityLevelSatisfaction} jest wymagany
        \item Atrybut \ref{kat:AppointmentEvaluation:mealTastefulnessSatisfaction} jest wymagany
        \item Atrybut \ref{kat:AppointmentEvaluation:dietaryResultSatisfaction} jest wymagany

        \item Atrybut \ref{kat:AppointmentEvaluation:overallSatisfaction} jest typu wyliczeniowego i może przyjmować wartości "VERY\_DISSATISFIED", "DISSATISFIED", "NEUTRAL", "SATISFIED", "VERY\_SATISFIED"
        \item Atrybut \ref{kat:AppointmentEvaluation:dietitianServiceSatisfaction} jest typu wyliczeniowego i może przyjmować wartości "VERY\_DISSATISFIED", "DISSATISFIED", "NEUTRAL", "SATISFIED", "VERY\_SATISFIED"
        \item Atrybut \ref{kat:AppointmentEvaluation:mealPlanOverallSatisfaction} jest typu wyliczeniowego i może przyjmować wartości "VERY\_DISSATISFIED", "DISSATISFIED", "NEUTRAL", "SATISFIED", "VERY\_SATISFIED"
        \item Atrybut \ref{kat:AppointmentEvaluation:mealCostSatisfaction} jest typu wyliczeniowego i może przyjmować wartości "VERY\_DISSATISFIED", "DISSATISFIED", "NEUTRAL", "SATISFIED", "VERY\_SATISFIED"
        \item Atrybut \ref{kat:AppointmentEvaluation:mealPreparationTimeSatisfaction} jest typu wyliczeniowego i może przyjmować wartości "VERY\_DISSATISFIED", "DISSATISFIED", "NEUTRAL", "SATISFIED", "VERY\_SATISFIED"
        \item Atrybut \ref{kat:AppointmentEvaluation:mealComplexityLevelSatisfaction} jest typu wyliczeniowego i może przyjmować wartości "VERY\_DISSATISFIED", "DISSATISFIED", "NEUTRAL", "SATISFIED", "VERY\_SATISFIED"
        \item Atrybut \ref{kat:AppointmentEvaluation:mealTastefulnessSatisfaction} jest typu wyliczeniowego i może przyjmować wartości "VERY\_DISSATISFIED", "DISSATISFIED", "NEUTRAL", "SATISFIED", "VERY\_SATISFIED"
        \item Atrybut \ref{kat:AppointmentEvaluation:dietaryResultSatisfaction} jest typu wyliczeniowego i może przyjmować wartości "VERY\_DISSATISFIED", "DISSATISFIED", "NEUTRAL", "SATISFIED", "VERY\_SATISFIED"
        \item Atrybut \ref{kat:AppointmentEvaluation:comment} jest ciągiem znaków
    \end{enumerate}

    \item\ref{kat:BodyMeasurement}
    \begin{enumerate}[label={\textbf{OGR/\protect\threedigits{\arabic{enumi}}}}, wide, labelwidth=!, align=left, leftmargin=3cm, resume]
        \item Atrybut \ref{kat:BodyMeasurement:completionDate} jest wymagany
        \item Atrybut \ref{kat:BodyMeasurement:height} jest wymagany
        \item Atrybut \ref{kat:BodyMeasurement:weight} jest wymagany
        \item Atrybut \ref{kat:BodyMeasurement:waist} jest wymagany

        \item Atrybut \ref{kat:BodyMeasurement:completionDate} jest stemplem czasowym
        \item Atrybut \ref{kat:BodyMeasurement:height} jest liczbą całkowitą
        \item Atrybut \ref{kat:BodyMeasurement:weight} jest liczbą całkowitą
        \item Atrybut \ref{kat:BodyMeasurement:waist} jest liczbą rzeczywistą
        \item Atrybut \ref{kat:BodyMeasurement:percentOfFatTissue} jest liczbą rzeczywistą nie mniejszą niż 0 i nie większą niż 100
        \item Atrybut \ref{kat:BodyMeasurement:percentOfWater} jest liczbą rzeczywistą nie mniejszą niż 0 i nie większą niż 100
        \item Atrybut \ref{kat:BodyMeasurement:muscleMass} jest liczbą rzeczywistą
        \item Atrybut \ref{kat:BodyMeasurement:physicalMark} jest liczbą rzeczywistą
        \item Atrybut \ref{kat:BodyMeasurement:calciumInBones} jest liczbą rzeczywistą
        \item Atrybut \ref{kat:BodyMeasurement:basicMetabolism} jest liczbą całkowitą
        \item Atrybut \ref{kat:BodyMeasurement:metabolicAge} jest liczbą rzeczywistą
        \item Atrybut \ref{kat:BodyMeasurement:visceralFatLevel} jest liczbą rzeczywistą
    \end{enumerate}

    \item\ref{kat:NutritionalInterview}
    \begin{enumerate}[label={\textbf{OGR/\protect\threedigits{\arabic{enumi}}}}, wide, labelwidth=!, align=left, leftmargin=3cm, resume]
        \item Atrybut \ref{kat:NutritionalInterview:completionDate} jest wymagany
        \item Atrybut \ref{kat:NutritionalInterview:targetWeight} jest wymagany
        \item Atrybut \ref{kat:NutritionalInterview:advicePurpose} jest wymagany
        \item Atrybut \ref{kat:NutritionalInterview:physicalActivity} jest wymagany

        \item Atrybut \ref{kat:NutritionalInterview:completionDate} jest stemplem czasowym
        \item Atrybut \ref{kat:NutritionalInterview:targetWeight} jest liczbą całkowitą
        \item Atrybut \ref{kat:NutritionalInterview:advicePurpose} jest ciągiem znaków
        \item Atrybut \ref{kat:NutritionalInterview:physicalActivity} jest typu wyliczeniowego i może przyjmować wartości "EXTREMELY\_INACTIVE", "SEDENTARY", "MODERATELY\_ACTIVE", "VIGOROUSLY\_ACTIVE", "EXTREMELY\_ACTIVE"
        \item Atrybut \ref{kat:NutritionalInterview:diseases} jest ciągiem znaków
        \item Atrybut \ref{kat:NutritionalInterview:medicines} jest ciągiem znaków
        \item Atrybut \ref{kat:NutritionalInterview:jobType} jest typu wyliczeniowego i może przyjmować wartości "SITTING", "STANDING", "MIXED"
        \item Atrybut \ref{kat:NutritionalInterview:likedProducts} jest ciągiem znaków
        \item Atrybut \ref{kat:NutritionalInterview:dislikedProducts} jest ciągiem znaków
        \item Atrybut \ref{kat:NutritionalInterview:foodAllergies} jest ciągiem znaków
        \item Atrybut \ref{kat:NutritionalInterview:foodIntolerances} jest ciągiem znaków
    \end{enumerate}

    \item\ref{kat:CustomNutritionalInterviewQuestion}
    \begin{enumerate}[label={\textbf{OGR/\protect\threedigits{\arabic{enumi}}}}, wide, labelwidth=!, align=left, leftmargin=3cm, resume]
        \item Atrybut \ref{kat:CustomNutritionalInterviewQuestion:question} jest wymagany

        \item Atrybut \ref{kat:CustomNutritionalInterviewQuestion:ordinalNumber} jest liczbą całkowitą nie mniejszą niż 1
        \item Atrybut \ref{kat:CustomNutritionalInterviewQuestion:question} jest ciągiem znaków
        \item Atrybut \ref{kat:CustomNutritionalInterviewQuestion:answer} jest ciągiem znaków
    \end{enumerate}

    \item\ref{kat:CustomNutritionalInterviewQuestionTemplate}
    \begin{enumerate}[label={\textbf{OGR/\protect\threedigits{\arabic{enumi}}}}, wide, labelwidth=!, align=left, leftmargin=3cm, resume]
        \item Atrybut \ref{kat:CustomNutritionalInterviewQuestionTemplate:question} jest wymagany

        \item Atrybut \ref{kat:CustomNutritionalInterviewQuestionTemplate:question} jest ciągiem znaków
    \end{enumerate}

    \item\ref{kat:AssignedMealPlan}
    \begin{enumerate}[label={\textbf{OGR/\protect\threedigits{\arabic{enumi}}}}, wide, labelwidth=!, align=left, leftmargin=3cm, resume]
        \item Atrybut \ref{kat:AssignedMealPlan:assigmentTime} jest wymagany

        \item Atrybut \ref{kat:AssignedMealPlan:assigmentTime} jest stemplem czasowym
    \end{enumerate}
\end{itemize}

\section{Model domenowy}\label{sec:domainModel}
\todo{diagram klas}

\begin{minipage}{\textwidth}
    \begin{figure}[H]
        \centering\includegraphics[scale=0.7]{../uml/class_diagrams/dataTypes.png}
        \caption{Typy danych - diagram klas (opr.wł).}\label{rysunek:class-diagram-data-types}
    \end{figure}
\end{minipage}

\begin{minipage}{\textwidth}
    \begin{figure}[H]
        \centering\includegraphics[scale=0.7]{../uml/class_diagrams/gateway.png}
        \caption{Gateway - diagram klas (opr.wł).}\label{rysunek:class-diagram-gateway}
    \end{figure}
\end{minipage}

\begin{minipage}{\textwidth}
    \begin{figure}[H]
        \centering\includegraphics[scale=0.7]{../uml/class_diagrams/products.png}
        \caption{Produkty - diagram klas (opr.wł).}\label{rysunek:class-diagram-products}
    \end{figure}
\end{minipage}

\begin{minipage}{\textwidth}
    \begin{figure}[H]
        \centering\includegraphics[scale=0.7]{../uml/class_diagrams/recipes.png}
        \caption{Przepisy - diagram klas (opr.wł).}\label{rysunek:class-diagram-recipes}
    \end{figure}
\end{minipage}

\begin{minipage}{\textwidth}
    \begin{figure}[H]
        \centering\includegraphics[scale=0.7]{../uml/class_diagrams/mealplans.png}
        \caption{Jadłospisy - diagram klas (opr.wł).}\label{rysunek:class-diagram-mealplans}
    \end{figure}
\end{minipage}

\begin{minipage}{\textwidth}
    \begin{figure}[H]
        \centering\includegraphics[scale=0.7]{../uml/class_diagrams/appointments.png}
        \caption{Wizyty - diagram klas (opr.wł).}\label{rysunek:class-diagram-appointments}
    \end{figure}
\end{minipage}

\section{Opis podstawowej architektury systemu}\label{sec:basicArchitecture}
\todo{Opisać, że to aplikacja webowa w architekturze mikroserwisów
Wyszczególnienie modułów;
Diagram rozmieszczenia, wzorce projektowe}

%https://martinfowler.com/eaaDev/TemporalObject.html
\thispagestyle{normal}

    % !TeX spellcheck = pl_PL
\chapter{Implementacja}\label{ch:implementation}
\section{Wykorzystywane środowiska i~narzędzia programistyczne}\label{sec:dev-tools}

Podczas wyboru języków programowania, z~użyciem których miał zostać zaimplementowany system, postawiono następujące kryteria:

\begin{itemize}
    \item ścisła kontrola typów,
    \item dobre wsparcie dla paradygmatu programowania obiektowego,
    \item niezależność języka od platformy,
    \item bogaty ekosystem.
\end{itemize}

\par
Wybrane języki spełniające te kryteria to:

\begin{itemize}
    \item w~warstwie backendu Java\cite{tech:java}~- opracowany przez Sun Microsystems język kompilowalny do kodu bajtowego, który jest wykonywany na maszynie wirtualnej,
    \item w~warstwie frontendu Typescript\cite{tech:typescript}~- opracowany przez Microsoft język otwartoźródłowy kompilowalny do języka JavaScript\cite{tech:javascript}.
\end{itemize}

Powyższy wybór zaowocował decyzją o~zastosowaniu Angulara\cite{tech:angular} jako wiodącej frontendowej platformy programistycznej (ang. framework)
i~Springa\cite{tech:spring} jako wiodącej backendowej platformy programistycznej.
Wspomniane platformy cieszą się bardzo dużą popularnością, a~ich dojrzałość sprawia,
że znajdują zastosowanie zarówno w~niewielkich aplikacjach jak i~w~systemach klasy enterprise.

\par
Podczas projektu witryny internetowej kod, który jest wykonywany po stronie przeglądarki zwykle jest napisany w~technologiach HTML, CSS\cite{tech:html-css} i~JavaScript\cite{tech:javascript}.
Jak już wspomniano zamiast języka JavaScript wykorzystano TypeScript, natomiast zamiast CSS postanowiono wykorzystać SASS\cite{tech:sass}, który rozszerza funkcjonalność CSS.

\par
System został zaprojektowany tak, żeby wykorzystać cechy relacyjnych baz danych,
więc podczas wyboru systemu zarządzania bazą danych pod uwagę wzięto tylko relacyjne bazy danych.
Rozważano przede wszystkim systemy PostgreSQL\cite{tech:postgresql} i~MySQL\cite{tech:mysql}.
Z~punktu widzenia funkcjonalności potrzebnych w~implementowanej aplikacji oba systemy systemy zarządzania relacyjną bazą danych (ang. Relational Database Management System~- RDBMS) wypadają równie dobrze,
jednakże ostatecznie wybrano PostgreSQL ze względu na mniej restrykcyjną licencję wykorzystania systemu nawet w~rozwiązaniach komercyjnych o~zamkniętym kodzie.
Dodatkowo w~celu implementacji ewolucyjnego projektowania bazy danych\cite{url:evolutionary-database} postanowiono wykorzystać bibliotekę Liquibase\cite{tech:liquibase} do zarządzania zmianami schematu bazy.

\par
Do implementacji architektury mikroserwisów postanowiono wykorzystać stos technologii Netflix OSS\cite{tech:netflix-oss}.
Jest to dojrzały, prosty do wykorzystania w~implementacji stos technologii mikroserwisowych,
dla którego możliwa jest ścisła integracja ze Springiem poprzez wykorzystania projektu Spring Cloud Netflix\cite{tech:spring-cloud-netflix}.
W skład tego stosu wchodzą przede wszystkim:
\begin{itemize}
    \item Eureka\cite{tech:netflix-eureka}~- serwis typu discovery,
    \item Zuul\cite{tech:netflix-zuul}~- serwis zapewniający dynamiczne przekierowywanie żądań z~bramy aplikacji do poszczególnych mikroserwisów,
    \item Ribbon\cite{tech:netflix-ribbon}~- serwis zapewniający równoważenie obciążenia podczas wyboru mikroserwisu, który odpowie na żądanie.
\end{itemize}

\par
Jako serwis typu discovery postanowiono wykorzystać JHipster Registry\cite{tech:jhipster-registry}, który jest oparty na serwisie Eureka
i dodatkowo zapewnia metryki kompatybilne z~ekosystemem aplikacji tworzonych w~oparciu o~generator JHipster.

\par
Ze względu na wykorzystanie architektury mikroserwisowej zdecydowano się na uwierzytelnianie użytkowników z~użyciem tokenów JWT\cite{url:jwt},
ponieważ jest to bezstanowy mechanizm, który można bezproblemowo wykorzystywać w~środowisku rozproszonym.

\par
W celu przyspieszenia rozwoju aplikacji postanowiono wykorzystać generator szkieletu aplikacji JHipster\cite{tech:jhipster}.
Korzystając z~autorskiego języka domenowego JHipstera możliwe jest zdefiniowanie konfiguracji systemu mikroserwisowego w~oparciu o~stos technologii Netflix OSS oraz encji przypisanych do konkretnych serwisów.
Wykorzystaną konfigurację przedstawiono w~dodatku \ref{app:jdl}, jednakże w~celu zwiększenia czytelności postanowiono pominąć opis encji, które zostały bezpośrednio oparte na kategoriach opisanych w~rozdziale \ref{sec:database}.
Na podstawie zdefiniowanej konfiguracji wygenerowane zostały szkielety serwisów zapewniające zarządzanie infrastrukturą mikroserwisów, uwierzytelnianie i~autoryzację użytkowników oraz przeprowadzanie podstawowych operacji CRUD na encjach.

\par
Aby zapewnić możliwie wysoką jakość oprogramowania konieczne jest przetestowanie czy kod działa w~oczekiwany sposób.
Do implementacji testów jednostkowych i~integracyjnych po stronie backendu postanowiono wykorzystać bibliotekę JUnit\cite{tech:junit} i~platformę Mockito\cite{tech:mockito}
\par
Dodatkowo postanowiono wykorzystać następujące narzędzia niezwiązane bezpośrednio z~implementacją:
\begin{itemize}
    \item Docker\cite{tech:docker}~- system konteneryzacji pozwalający uprościć proces wdrażania aplikacji z~wykorzystaniem konfiguracji niezależnej od środowiska,
    \item Docker Compose\cite{tech:docker-compose}~- narzędzie upraszczające zarządzanie wielokontenerowym środowiskiem aplikacji skonteneryzowanych,
    \item Git\cite{tech:git}~- rozproszony system kontroli wersji wykorzystywany do zarządzania zmianami w~kodzie,
    \item Gitlab Pipelines\cite{tech:gitlab-pipelines}~- narzędzie wspomagające proces ciągłej integracji.
\end{itemize}

\section{Zakres implementacji}\label{sec:implementation-scope}

W wyniku analizy założeń projektowych przedstawionych w~rozdziale \ref{ch:design-assumptions} stwierdzono,
że osiągnięcie minimalnego funkcjonalnego stanu produktu (ang. Minimum Viable Product~- MVP)
wymaga zaimplementowania całego modelu domeny omówionego w~projekcie bazy danych w~rozdziale \ref{sec:database}.
Natomiast w~kwestii dostępu do systemu, w~celu osiągnięcia MVP wystarczy, żeby bezpośredni dostęp mieli tylko dietetycy i~administratorzy,
a pacjenci będą otrzymywali skomponowane diety na adres mailowy, który podali dietetykowi.

\par
Ze względu na przetwarzanie wrażliwych danych osobowych, konieczne jest zawarcie w~witrynie internetowej polityki prywatności.
Na rzecz osiągnięcia MVP w~przygotowywanej implementacji postanowiono wykorzystać szablon udostępniany przez firmę ogicom.pl\cite{url:ogicom-privacy-policy}.
Jednakże należy podkreślić, że autor niniejszej pracy inżynierskiej nie posiada wykształcenia prawniczego
i należy traktować przedstawioną politykę prywatności jedynie jako wersję roboczą,
a przed komercyjnym wdrożeniem systemu należałoby zasięgnąć porady w~kancelarii prawnej oferującej doradztwo prawne z~zakresu przetwarzania i~ochrony danych osobowych.
%Wykorzystany szablon przedstawiony został w~dodatku \ref{app:privacy-policy}.

\todo{usda-db}

\section{Architektura systemu}\label{sec:system-architecture}
\subsection{Architektura mikroserwisów}\label{subsec:system-architecture:microservices}

\noindent
\image{0.7}{../uml/deployment_diagrams/deployment.png}{Diagram rozmieszczenia}{deployment-diagram}

Na rysunku \ref{fig:deployment-diagram} przedstawiono diagram rozmieszczenia prezentujący fizyczne rozmieszczenie komponentów systemu.
Mikroserwisy mogą mieć wiele instancji, które po uruchomieniu rejestrują się w~serwisie JHipster Registry za pomocą protokołu HTTP.
Serwis Gateway z~pomocą mechanizmu równoważenia obciążenia komunikuje się z~innymi serwisami za pomocą protokołu HTTP.
Klient końcowy za pomocą przeglądarki internetowej łączy się z~systemem za pomocą protokołu HTTP.
Żądania klienta są obsługiwane przez jedną instancję serwisu Gateway, a~serwis Gateway może obsługiwać żądania wielu klientów.
Warto zauważyć, że serwis JHipster Registry jest pojedynczym punktem awarii (ang. Single point of failure), gdyż wszystkie mikroserwisy korzystają z~pojedynczej instancji tego serwisu.
Komunikacja z~serwerami bazodanowymi odbywa się za pomocą protokołu PostgreSQL.
W celu uniknięcia niespójności danych przyjęto założenie, że może istnieć tylko jedna instancja bazy danych dla konkretnego serwisu.

\subsection{Architektura backendu}\label{subsec:system-architecture:backend}

\noindent
\image{0.75}{../uml/package_diagrams/backend.png}{Podstawowy diagram pakietów warstwy biznesowej mikroserwisów}{package-diagram-backend}

Na rysunku \ref{fig:package-diagram-backend} przedstawiono diagram pakietów obrazujący konwencję zastosowaną we wszystkich mikroserwisach podczas implementowania kodu backendu.
Pakiet "REST" jest odpowiedzialny za odbieranie żądań przychodzących do aplikacji i~odpowiadanie na nie.
W celu wykonania akcji biznesowej wywoływane są funkcje z~klas pakietu "SERVICE", który agreguje całą logikę biznesową aplikacji.
Żeby ułatwić testowanie jednostkowe z~wykorzystaniem Mockito w~warstwie serwis postanowiono zaimplementować wzorzec projektowy Most\cite{book:wzorce-projektowe} i~oddzielić interfejsy serwisów od ich implementacji.
Aby realizować niektóre akcje biznesowe konieczne jest odwołanie się do bazy danych.
Za wszystkie operacje na bazie danych odpowiadają klasy z~pakietu "REPOSITORY".
We wszystkich wymienionych pakietach możliwe jest korzystanie z~pakietu "DOMAIN" zawierającego definicje encji i~typów wyliczeniowych.

\par
Dodatkowo zdefiniowane są dwa pakiety nie związane z~realizacją akcji biznesowych:
\begin{itemize}
    \item "SECURITY"~- odpowiedzialny za zabezpieczanie aplikacji,
    \item "CONFIG"~- odpowiedzialny za zapewnienie konfiguracji aplikacji.
\end{itemize}

\par
Klasy z~pakietów "REST", "SERVICE.IMPL", "REPOSITORY", "SECURITY" i~"CONFIG" zdefiniowane są jako komponenty Spring (ang. Spring Bean) przez co platforma Spring zarządza ich czasem życia i~zapewnia mechanizm wstrzykiwania zależności\cite{book:spring-w-akcji}.

\subsection{Architektura frontendu}\label{subsec:system-architecture:frontend}

\noindent
\image{0.75}{img/angular-architecture.png}{Ogólna architektura angulara}{angular-architecture}

Architektura frontendu jest ściśle związana a~architekturą promowaną na platformie Angular opartej na komponentach,
co przedstawiono na rysunku \ref{fig:angular-architecture}.
Podstawową jednostką budulcową są komponenty, które za pomocą wiązania właściwości i~zdarzeń oddziałują z~szablonami w~calu prezentacji treści użytkownikowi końcowemu.
Komponenty mogą korzystać z~serwisów dostępnych z~pomocą mechanizmu wstrzykiwania zależności.

\section{Dokumentacja kodu}\label{sec:code-documentation}

Podstawowa dokumentacja kodu została napisana przy użyciu komentarzy w~stylu kompatybilnym z~generatorem dokumentacji JavaDoc\cite{tech:javadoc}.
Przykładowy komentarz przedstawiono na listingu\ref{listing:javadoc}, a~fragment wygenerowanej dokumentacji na rysunku \ref{fig:javadoc}.

\begin{listing}[h!]
    \begin{minted}{bash}
        /**
        * Short description of measure in language of a~product, e.g. \"cup\" or \"tea spoon\"
        */
        @NotNull
        @Size(min = 1, max = 255)
        private String description;
    \end{minted}
    \centering\caption{Uruchamianie Gradle Wrapper (opr. wł.)}\label{listing:javadoc}
\end{listing}

\imagewide{img/example-javadoc.png}{Przykładowy fragment dokumentacji JavaDoc}{javadoc}

Dodatkowo wykorzystano narzędzie Swagger\cite{tech:swagger} do automatycznego generowania dokumentacji
punktów końcowych interfejsu programowania aplikacji (ang. Application Programming Interface Endpoints~- API Endpoints) na podstawie implementacji,
czego rezultat zaprezentowano na rysunku \ref{fig:swagger}.

\imagewide{img/swagger-example.png}{Przykładowy fragment dokumentacji Swagger}{swagger}

\section{Instalacja oprogramowania}\label{sec:software-installation}
\subsection{Wymagania wstępne}\label{subsec:prerequirements}
Przed przystąpieniem do wykonywania kolejnych kroków należy się upewnić, że następujące narzędzia są zainstalowane:
\begin{itemize}
    \item Open JDK 11 (https://adoptopenjdk.net/?variant=openjdk11),
    \item Node.js 10 lub nowsza wersja LTS (https://nodejs.org/en/),
    \item Docker 19.03 + Docker Compose 2 (https://docs.docker.com/install/).
\end{itemize}

\subsection{Instalacja}\label{subsec:installation}

Aby zbudować i~uruchomić projekt z~wykorzystaniem Dockera należy z~poziomu głównego katalogu projektu
wykonać polecenia przedstawione na listingu \ref{listing:komilacja-i-uruchomienie}.
\begin{listing}[h!]
    \begin{minted}{bash}
        cd gateway
        npm install
        sh gradlew bootJar -Pprod jibDockerBuild
        cd ../products
        sh gradlew bootJar -Pprod jibDockerBuild
        cd ../recipes
        sh gradlew bootJar -Pprod jibDockerBuild
        cd ../mealplans
        sh gradlew bootJar -Pprod jibDockerBuild
        cd ../appointments
        sh gradlew bootJar -Pprod jibDockerBuild
        cd ../docker-compose
        sh docker-compose up
    \end{minted}
    \centering\caption{Skrypt kompilujący wszystkie mikroserwisy i~uruchamiający aplikację na Dockerze (opr. wł.)}\label{listing:komilacja-i-uruchomienie}
\end{listing}

Alternatywnie, dla celów deweloperskich można zastosować uproszczony proces nie wykorzystujący Dockera.
W tym celu należy najpierw uruchomić JHipster Registry wykonując polecenie z~poziomu głównego katalogu projektu
wykonać polecenia przedstawione na listingu \ref{listing:service-discovery}.
\begin{listing}[h!]
    \begin{minted}{bash}
        cd service-discovery
        java -jar ./jhipster-registry-5.0.2.jar --spring.profiles.active=dev --spring.security.user.password=admin --spring.cloud.config.server.composite.0.type=git --spring.cloud.config.server.composite.0.uri= https://github.com/jhipster/jhipster-registry-sample-config
    \end{minted}
    \centering\caption{Uruchamianie JHipster Registry (opr. wł.)}\label{listing:service-discovery}
\end{listing}

Następnie z~poziomu katalogu każdego z~serwisów (gateway, products, recipes, mealplans, appointments)
należy wykonać polecenie uruchamiające Gradle Wrapper przedstawione na listingu \ref{listing:run-gradle-wrapper}.
\begin{listing}[h!]
    \begin{minted}{bash}
        ./gradlew
    \end{minted}
    \centering\caption{Uruchamianie Gradle Wrapper (opr. wł.)}\label{listing:run-gradle-wrapper}
\end{listing}

Po uruchomieniu wszystkich serwisów aplikacja będzie dostępna pod adresem \textit{localhost:8080}.

\section{Prezentacja aplikacji}\label{sec:app-presentation}
\todo{podstawowy opis poruszania się po aplikacji, zrzuty ekranu z~kilku najważniejszych widoków}
\todo{filmik na youtube'ie?}
%\todo{implementacja w~kodzie: obliczanie podstawowych wartości odżywczych w~przepisie, wyświetlanie spełnienia norm odżywczych w~jadłospisie, wersjonowanie produktów i~przepisów, generowanie listy zakupów i~jadłospisu do wydruku, wykres BMI}

%\section{Testy}\label{sec:tests}
%\subsection{Testy jednostkowe}\label{subsec:tests:unit}
%\todo{opisać}
%\subsection{Testy integracyjne}\label{subsec:tests:integration}
%\todo{opisać}
%\subsection{Testy użyteczności}\label{subsec:tests:usability}
%\todo{opisać}
\thispagestyle{normal}

    % !TeX spellcheck = pl_PL
\chapter{Testy}
\section{Wprowadzenie}
Tworząc oprogramowanie istotne jest żeby akcje biznesowe i~inne operacje wykonywane w~systemie były realizowane w~określony sposób i~przynosiły oczekiwane rezultaty.
Wiąże się z~tym pojęcie zapewniania jakości, czyli przede wszystkim "spełnienie lub przekroczenie wymagań klienta"\cite{book:jakosc-projektow-informatycznych}.
Podstawowym narzędziem pozwalającym na zapewnienie jakości oprogramowania jest przeprowadzanie testów.
Według sylabusa Międzynarodowej Rady Kwalifikacji Testów Oprogramowania (ang. International Software Testing Qualifications Board~- ISQB)\cite{url:istqb-syllabus} testowanie przeprowadza się w~celu:
\begin{itemize}
    \item znajdowania i~zapobiegania błędom,
    \item zdobywania pewności odnośnie poziomu jakości,
    \item sprawdzania czy akcje biznesowe realizowane są w~oczekiwany sposób,
    \item zapewniania informacji dotyczących bieżącego stanu implementacji.
\end{itemize}

\par
W ramach niniejszej pracy opisane zostaną następujące rodzaje testów:
\begin{itemize}
    \item \textbf{jednostkowe}~- testujące logikę biznesową,
    \item \textbf{integracyjne}~- testujące punkty końcowe API,
    \item \textbf{użyteczności}~- testy przeprowadzane z~użytkownikami końcowymi sprawdzające użyteczność systemu.
\end{itemize}

\par
Pierwsze dwa rodzaje testów mogą być przeprowadzane automatycznie, dlatego postanowiono wdrożyć rozwiązanie zapewniające ciągłą integrację (ang. Continuous Integration~- CI) z~wykorzystaniem platformy Gitlab CI\cite{tech:gitlab-pipelines}.
Dzięki temu w~momencie zwracania kodu do repozytorium automatycznie wykonywany jest zdefiniowany proces CI (ang. pipeline), zawierający następujące etapy:
\begin{itemize}
    \item \textbf{build}~- kompilacja kodu,
    \item \textbf{unit tests}~- wykonanie testów jednostkowych,
    \item \textbf{integration tests}~- wykonanie testów integracyjnych,
    \item \textbf{package}~- pakowanie skompilowanego kodu do plików *.jar.
\end{itemize}

\par
Rezultat wykonania procesu CI na platformie Gitlab CI został przedstawiony na rysunku \ref{fig:gitlab-pipeline}.

\todo{screenshoot pipeline z~gitlaba}

\section{Testy jednostkowe}

Robert C. Martin stwierdził, że dobrze napisane testy jednostkowe zwiększają elastyczność kodu produkcyjnego,
ułatwiają wprowadzanie zmian w~kodzie i~pozwalają szybko wykryć zaistniałe błędy\cite{book:czysty-kod}.

\par
Przedmiotem testów jednostkowych jest logika akcji biznesowych przeprowadzanych w~obrębie warstwy serwisów.
Należy jednak zauważyć, że wątpliwą wartość przynosi testowanie funkcji,
których jedyną odpowiedzialnością jest przekierowanie żądania z~warstwy punktów końcowych API do warstwy repozytoriów.

\par
Testy jednostkowe, jak nazwa wskazuje, powinny testować jednostkę taką jak funkcja i~powinny daną jednostkę testować w~izolacji od zależności zewnętrznych\cite{book:testy-jednostkowe}.
Aby osiągnąć taką izolację, zastosowano platformę Mockito dzięki czemu możliwe jest tworzenie makiet dla zależności zewnętrznych i~definiowanie rezultatów obcowania z~nimi.
W rezultacie osiągnięta może być ścisła kontrola nad przepływem informacji w~obrębie testowanej jednostki.

\par
Na listingu \ref{listing:unit-test} przedstawiony został przykładowy test jednostkowy.

\noindent\hspace{.075\textwidth}\begin{minipage}{.85\textwidth}
\begin{minted}{java}
@RunWith(MockitoJUnitRunner.class)
public class ExampleUnitTest {
  @Mock private UserService userService;
  @Mock private ProductRepository productRepository;
  @Mock private CacheManager cacheManager;
  @Mock private ProductSubcategoryService productSubcategoryService;
  @Mock EntityManager entityManager;
  @InjectMocks private ProductServiceImpl productService;
  private User user;
  private Product product;

  @Before
  public void setup() {
    long id = 1;
    this.user = UserCreator.createEntity();
    this.user.setId(id);
    this.product = ProductCreator.createEntity(entityManager);
    this.product.setId(id);
    this.product.setAuthor(user);
    when(userService.getCurrentUser())
      .thenReturn(Optional.of(this.user));
    when(productRepository.findOneWithEagerRelationships(any()))
      .thenReturn(Optional.of(this.product));
  }

  @Test
  public void authorShouldBeAbleToEditOwnProduct() {
    //when
    productService.save(product);
    //then
    Mockito.verify(productRepository, times(1)).saveAndFlush(this.product);
  }
}
\end{minted}
\begin{lstlisting}[caption={Przykładowy test jednostkowy \source{\ownwork}}, label={listing:unit-test}]
\end{lstlisting}
\end{minipage}

\section{Testy integracyjne}

W przeciwieństwie do testów jednostkowych testy integracyjne nie są wykonywane w~całkowitej izolacji\cite{book:testy-jednostkowe}.
Testowaniu podlegają punkty końcowe API z~wykorzystaniem rzeczywistego połączenia z~bazą danych.
Na potrzeby testów tworzona jest instancja osadzonej bazy danych H2\cite{tech:h2-db} oraz inicjalizowany jest kontekst aplikacji Spring Boot.

\par
Celem przeprowadzania tego typu testów jest sprawdzenie czy punkty końcowe API w~poprawny sposób obsługują przychodzące żądania
z uwzględnieniem przeprowadzenia akcji biznesowych w~warstwie serwisów i~perzystencji w~warstwie repozytoriów.

\par
Na listingu \ref{listing:integration-test} przedstawiony został przykładowy test integracyjny.

\noindent\hspace{.075\textwidth}\begin{minipage}{.85\textwidth}
\begin{minted}{java}
@SpringBootTest(classes = RecipesApp.class)
public class ExampleIntegrationTest {
  @Autowired private DishTypeRepository dishTypeRepository;
  @Autowired private DishTypeService dishTypeService;
  @Autowired private DishTypeSearchRepository mockDishTypeSearchRepository;
  @Autowired private MappingJackson2HttpMessageConverter jacksonMessageConverter;
  @Autowired private PageableHandlerMethodArgumentResolver pageableArgumentResolver;
  @Autowired private ExceptionTranslator exceptionTranslator;
  @Autowired private EntityManager em;
  @Autowired private Validator validator;

  private MockMvc restDishTypeMockMvc;

  @BeforeEach
  public void setup() {
    MockitoAnnotations.initMocks(this);
    final DishTypeResource dishTypeResource = new DishTypeResource(dishTypeService);
    this.restDishTypeMockMvc = MockMvcBuilders.standaloneSetup(dishTypeResource)
      .setCustomArgumentResolvers(pageableArgumentResolver)
      .setControllerAdvice(exceptionTranslator)
      .setConversionService(createFormattingConversionService())
      .setMessageConverters(jacksonMessageConverter)
      .setValidator(validator).build();
  }

  @Test
  @Transactional
  public void getNonExistingDishType() throws Exception {
    // Get the dishType
    restDishTypeMockMvc.perform(get("/api/dish-types/{id}", Long.MAX_VALUE))
      .andExpect(status().isNotFound());
  }
}
\end{minted}
\begin{lstlisting}[caption={Przykładowy test integracyjny \source{\ownwork}}, label={listing:integration-test}]
\end{lstlisting}
\end{minipage}

\section{Testy użyteczności}

Oprogramowanie zwykle nie jest tworzone, żeby istniało w~próżni.
Należy więc zbadać w~jakim stopniu aplikacja może być używana przez rzeczywistych użytkowników.
W tym celu przeprowadzane są testy użyteczności\cite{book:testowanie-i-jakosc-oprogramowania}.

\par
W ramach realizacji tej grupy testów, 7 potencjalnych użytkowników (tj. osoby studiujące dietetykę lub zajmujące się profesjonalnie dietetyką)
zostało poproszonych o~zasymulowanie przeprowadzania kompleksowej wizyty pacjenta
i wyrażenie opinii o~używalności systemu poprzez odpowiedź na pytania ustandaryzowanego formularza Skali Używalności Systemu (ang. System Usability Scale - SUS)\cite{url:sus}.

\par
Korzystając z~SUS, uczestnik badania udziela odpowiedzi na 10 pytań w~pięciostopniowej skali od "Bardzo się zgadzam" do "Bardzo się nie zgadzam".
Rezultaty następnie są konwertowane na wartość liczbową z~zakresu 0-4, a~suma otrzymanych punktów jest mnożona przez 2.5, żeby otrzymać wynik w~skali 0-100.
Wynik powyżej 68 punktów uznawany jest za ponadprzeciętny.
Na rysunku \ref{fig:sus-form} przedstawiony został wykorzystany kwestionariusz SUS.

\imagewide[\cite{url:sus-form}]{img/kwestionariusz-sus.png}{Kwestionariusz SUS}{sus-form}

\todo{rezultaty sus}

\thispagestyle{normal}


%    \input{instrukcja}
%    \chapter{Ala ma kota}

ĄĆĘŁŃÓŚŹŻ ąćęłńóśźż\footnote{Przykład użycia polskich znaków diakrytycznych oraz przypisu w miejscu}. \lipsum[1]

\section{Odniesienie do pozycji z literatury (strona WWW)}

% Odniesienie do rysunku i cytowanie dokumentu. Dokumenty są definiowane w pliku literatura.bib
Reszta dokumentacji znajduje się w \cite{docker_compose_reference}. \lipsum[3]

\section{Odniesienie do książki}

Jak pisze Harel w \cite{harel_rzecz_2008}: \lipsum[7]

\section{Rysunek}

% Rysunek
\begin{figure}
    \centering\includegraphics[width=.6\textwidth]{img/swarm-network}
    \caption{Docker ma sieć \cite{docker_compose_reference}.}  \label{rys:network}
    % Źródło rysunku i etykieta przez którą odwołujemy się do rysunku.
\end{figure}

Jak widać na rys. \ref{rys:network} Docker ma wewnętrzną sieć. \lipsum[1]


\subsection{Rysunek z kotem}

Jak widać na rys.\ref{rysunek:kot} Ala ma kota. \lipsum[9-10]

\begin{figure}[h!]
    \centering\includegraphics[width=.4\textwidth]{img/kotek}
    \caption{Ala ma kota (opr.wł).}\label{rysunek:kot}
\end{figure}

\subsection{Tabela}

Co uwzględniono w tabeli \ref{tabela:coktoma}. \lipsum[13-15]

% Tabela. Nazwa tabeli u góry.
\begin{table}[h!]
    \centering\caption{Co kto ma \cite{harel_rzecz_2008} (patrz też dodatek~\ref{Dod1}) \label{tabela:coktoma}}
    \begin{tabular}{|l|l|l|}% wyrównanie kolumn tabeli -> l c r - do lewej, środka, do prawej
        \hline
        Ala & ma & kota \\
        \hline
        Ola & ma & psa \\
        \hline
        Ula & ma & małpę\\
        \hline
    \end{tabular}
\end{table}

\lipsum[19-20] Warto wspomnieć, że w \cite{aizawa_groundwater_2009} rzecz przedstawiona jest zupełnie inaczej. Poniższy wzór:

\begin{equation}
    \sum_{i=1}^{\infty}a_i
    \label{eq:mojWzor}
\end{equation}

Wzór \ref{eq:mojWzor} wskazuje że dowód podany w \cite{kaleta_experimental_2005} może zostać podważony. \lipsum[9]

\section{Kod źródłowy}

% lub {java} albo {bash} albo {text}
\begin{listing}[h!]
    \begin{minted}{c}
        int main()
        {
        int a=2*3;
        printf("**Ala ma kota\n**");
        while(!I2C_CheckEvent(I2C1, I2C_EVENT_MASTER_MODE_SELECT)); /* EV5 */
        return 0;
        }
    \end{minted}
    \caption{Przykładowy algorytm w języku C (opr. wł.)} \label{listing:moj}
\end{listing}

W moim kodzie \ref{listing:moj} zrobiłem coś wspaniałego. \lipsum[4]

\begin{table}[h]
    \begin{tabularx}{\textwidth}{|>{\setlength\hsize{1.4\hsize}\setlength\linewidth{\hsize}}X|>{\setlength\hsize{.9\hsize}\setlength\linewidth{\hsize}}X|>{\setlength\hsize{.7\hsize}\setlength\linewidth{\hsize}}X|}
        \hline
        \multicolumn{3}{|c|}{Classification of the criticel point $(0,0)$ of $x'=Ax,|\mathbf{A}|\not=0$.}\\
        \hline
        Types & Type of Critical Point & Stability \\
        \hline
        1. Real unequal eigenvalues of same sign
        \begin{itemize}
            \item $\lambda_1 > \lambda_2 > 0$
            \item $\lambda_1 < \lambda_2 < 0$
        \end{itemize} &
        \vphantom{1. Real unequal eigenvalues of same sign}
        \begin{itemize}
            \item Improper Node/Node
            \item Improper Node/Node
        \end{itemize} &
        \vphantom{1. Real unequal eigenvalues of same sign}
        \begin{itemize}
            \item Unstable
            \item Asym. Stable
        \end{itemize}\\
        \hline
        2. Real unequal eigenvalues of opposite sign
        \begin{itemize}
            \item $\lambda_2 < 0 >\lambda_1$
        \end{itemize} &
        \vphantom{2. Real unequal eigenvalues of opposite sign}
        \begin{itemize}
            \item Saddle Point
        \end{itemize} &
        \vphantom{2. Real unequal eigenvalues of opposite sign}
        \begin{itemize}
            \item Unstable
        \end{itemize}\\
        \hline
        3. Equal eigenvalues \newline Subtype 1: Two Independent vectors
        \begin{itemize}
            \item $\lambda_1 = \lambda_2 > 0$
            \item $\lambda_1 = \lambda_2 < 0$
        \end{itemize} &
        \vphantom{3. Equal eigenvalues} \vphantom{ Subtype 1: Two Independent vectors}
        \begin{itemize}
            \item Proper Node
            \item Proper Node
        \end{itemize} &
        \vphantom{3. Equal eigenvalues} \vphantom{ Subtype 1: Two Independent vectors}
        \begin{itemize}
            \item Unstable
            \item Asym. Stable
        \end{itemize}\\
        \hline
    \end{tabularx}
\end{table}
\thispagestyle{normal}


    % !TeX spellcheck = pl_PL
\chapter*{Zakończenie}\label{ch:ending}
Rezultatem wykonanych prac jest platforma wspomagająca zarządzanie dietą spełniająca kryteria biznesowe postawione w~fazie analizy wymagań.
Dietetycy mogą wykorzystać opracowane rozwiązanie do ułatwienia układania i~udostępniania swoim pacjentom jadłospisów,
a także do kontroli rezultatów stosowania diety przez pacjentów.

\par
Należy jednak wziąć pod uwagę, że oprogramowanie to ma na celu przede wszystkim ułatwienie pracy dietetyka
i zakłada się, że użytkownicy korzystający z~aplikacji będą mieli specjalistyczną wiedzę w~dziedzinie dietetyki.

\par
Oprogramowanie zostało zaprojektowane w~sposób ułatwiający skalowanie,
a zastosowane wzorce i~dobre praktyki programistyczne sprawią, że wdrażanie nowych funkcjonalności w~systemie nie będzie stanowiło problemu.
Do funkcjonalności, które mogłyby zostać wdrożone w~systemie w~przyszłości można zaliczyć przede wszystkim:
\begin{itemize}
    \item Zapewnienie możliwości korzystania z~aplikacji bez połączenia z~internetem poprzez dodanie funkcjonalności progresywnej aplikacji webowej (ang. Progressive Web App~- PWA)\cite{url:pwa}.
    \item Zapewnienie możliwości instalacji aplikacji na urządzeniach mobilnych poprzez integrację z~platformą Ionic\cite{tech:ionic}.
    \item Umożliwienie pacjentom korzystania z~aplikacji.
    \item Zapewnienie możliwości komunikacji pacjenta z~dietetykiem z~wykorzystaniem komunikatora zintegrowanego z~opracowywanym systemem.
\end{itemize}

\par
Wdrożenie aplikacji komercyjnie i~szersza współpraca z~ekspertami domenowymi z~dziedziny dietetyki z~pewnością pokazałyby wiele innych możliwości rozwoju opracowanej platformy,
jednakże może to wymagać zainwestowania znacznych środków finansowych.
\thispagestyle{normal}


    % W pracy pojawią się tylko prace naprawdę cytowane.
    % \nocite{*}

    \bibliography{literatura}
    \bibliographystyle{dyplom}

    \newpage
    \addcontentsline{toc}{chapter}{\listfigurename}
    \listoffigures

    \newpage
    \addcontentsline{toc}{chapter}{\listtablename}
    \listoftables

    \newpage
    \listof{listing}{Spis kodów źródłowych}

    % !TeX spellcheck = pl_PL
% \appendixpage
% \addappheadtotoc

\appendix
\begin{appendices}
    \chapter{Konfiguracja mikroserwisów w~języku JDL}\label{app:jdl}

    \inputminted{text}{../../jdl/jdl-short.jh}
    \begin{listing}[h!]
        \caption{Definicja mikroserwisów w~języku JDL (źródło: opr.wł)} \label{listing:short-jdl}
    \end{listing}

\end{appendices}
\thispagestyle{normal}


\end{document}
